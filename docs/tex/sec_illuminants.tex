% SPDX-License-Identifier: Apache-2.0
% Copyright (c) Contributors to the PhysLight Project.

\chapter{Illuminants}\label{ch:illuminants}

\section{Spectral Distribution Functions}

\subsection{Black body radiator}

\begin{figure}
{
\small
\centering
\resizebox{\linewidth}{!}{\import{figures_built/}{blackbody.pgf}}
\caption{Spectral distributions of black body radiator at various temperatures}
\label{fig:blackbody}
}
\vskip 1mm
{\footnotesize\it Plots of spectral distribution of black body
radiator at temperatures of 
$2855.54$ \textcolor[rgb]{1,0.809228,0.455376}{\rule{1em}{1em}},
$3200$ \textcolor[rgb]{1,0.826374,0.531592}{\rule{1em}{1em}},
$5000$ \textcolor[rgb]{0.95935,0.865267,0.780853}{\rule{1em}{1em}},
$6500$ \textcolor[rgb]{0.897476,0.874411,0.894404}{\rule{1em}{1em}}
degrees Kelvin. 
In the rendition, the color of each curve is taken from the
sRGB coordinates for the corresponding color,
normalized so that the maximum sRGB linear value is $0.7$}
\end{figure}

Black body spectral radiance as per Planck's law~\cite{planck14}:
\begin{equation}
B_T(\lambda) = \frac{2 h c^2}{\lambda^5} \frac 1{e^{\frac{hc}{\lambda k
T}}-1}
\qquad \left[\unit{\watt\per\square\meter\per\steradian\per\meter}\right]
\end{equation}
where relevant constants and variables are:

\begin{center}
\begin{tabular}{c l l}
$h$                & Planck constant            & $6.62607015\times 10^{-34} \;[\unit{\joule\second}]$  \\
$k$                & Boltzmann constant         & $1.380649\times 10^{-23} \;[\unit{\joule\per\kelvin}]$  \\
$c$                & speed of light in vacuum   & $2.99792458\times 10^{8} \;[\unit{\meter\per\second}]$ \\
$T$                & temperature in kelvins     & $[\unit{\kelvin}]$    \\
$\lambda$          & wavelength in meters       & $[\unit{\meter}]$    \\
$T_{kK}$           & temperature in kilokelvins & $[\unit{\kilo\kelvin}]$ \\
$\lambda_{\mu m}$  & wavelength in micrometers  & $[\unit{\micro\meter}]$ \\
\end{tabular}
\end{center}
Note that the fundamental constants $h$, $k$, $c$, are defined to exactly the value above in~\cite{bipm:si.2019}.
It is possible to get a formulation with every-day ranged constants and values in play by expressing the temperature in kilokelvins $[\unit{\kilo\kelvin}]$ and the wavelength in micrometers $[\unit{\micro\meter}]$:

\begin{equation}
B_T(\lambda) \approx \frac{119.104}{\lambda_{\mu m}^5} \frac
1{e^{\frac{14.3878}{\lambda_{\mu m} T_{kK}}}-1}
\qquad \left[\unit{\watt\per\square\meter\per\steradian\per\meter}\right]
\end{equation}

\paragraph{Chromaticity}

\begin{figure}
{
\centering
\begin{minipage}{.4\linewidth}
\resizebox{\linewidth}{!}{\import{figures_built/}{chromas.pgf}}
\end{minipage}\qquad\qquad
\begin{minipage}{.4\linewidth}
\resizebox{\linewidth}{!}{\import{figures_built/}{chromas_enlarge.pgf}}
\end{minipage}

\caption{Comparison of CCTs for black body and Illuminant D}
\label{fig:blackbodychroma}
}
\vskip 1mm
{\footnotesize\it Red crosses show actual chromaticity for a black body
emitter, green crosses show the result from the approximation in
\cite{kang02}, blue crosses show the chromaticity of CIE Standard
Illuminant D. The plot on the right is an enlargement of the
central region of the plot on the left, the triangle in the left plot
is the sRGB gamut}
\end{figure}

White point for black body radiator. Black body's approximate chromaticity
coordinates in CIE $xy$ space can be computed following \cite{kang02} as:

\begin{equation}
\begin{array}{rcll}
w & = & 1/T_{kK} \qquad \text{inverse of temperature in kiloKelvin} & \\
x & = & \begin{cases}
     - .2991239\,w^3 -  .234358 \,w^2  +  .8776956\,w  + .17991  & T_{kK} \in [\frac{5}{3}, 4] \\
     -3.0258469\,w^3 + 2.1070379\,w^2  +  .2226347\,w  + .240390 & T_{kK} \in [4, 25]
    \end{cases} \\
y & = & \begin{cases}
     -1.1063814\,x^3 - 1.34811020\,x^2 + 2.18555832\,x - .20219683 & T_{kK} \in [\frac{5}{3}, \frac{20}{9}] \\
     - .9549476\,x^3 - 1.37418593\,x^2 + 2.09137015\,x - .16748867 & T_{kK} \in [\frac{20}{9}, 4] \\
      3.081758 \,x^3 - 5.8733867 \,x^2 + 3.75112997\,x - .37001483 & T_{kK} \in [4, 25]
      \end{cases}
\end{array}
\end{equation}

\paragraph{Luminance}

The luminance of a black body radiator $\|B_T\|_{\bar y} = \int B_T(\lambda)\bar y(\lambda) d\lambda$,
this can be approximately computed as

\begin{equation}\label{eq:luminance_blackbody}
\|B_T\|_{\bar y} \approx \frac{50471.2}{e^{\frac{14.3878}{0.0384919 +
0.541533\; T_{kK}}}-1}
\quad \left\{\begin{array}{cl}
\pm 0.375\% & T_{kK} \in [1.6, 2.5] \\
+0.23\% \pm 0.1\% & T_{kK} \in [2.5, 3.3] \\
\pm 0.155\% & T_{kK} \in [3.3, 8.5] \\
\pm 0.11\%  & T_{kK} \in [9, 25] \\
\end{array}\right.
\quad \left[\unit{\watt\per\square\meter\per\steradian}\right]
\end{equation}


\subsection{CIE Standard Illuminants}

A standard illuminant is a theoretical source of visible light with a profile
(its spectral power distribution) which is published. 
Standard illuminants provide a basis for comparing images or colors recorded 
under different lighting. Standard Illuminants A, B, and C were
introduced in 1931, with the intention of respectively representing average
incandescent light, direct sunlight, and average daylight. 
Illuminants D represent phases of daylight, Illuminant E is the equal-energy 
illuminant, while Illuminants F represent fluorescent lamps of various
composition. 
Over many years of work and many revision, data for illuminants 
A, D65 and D50 is available in \cite{isocie:11664-2:2022}.

\begin{figure}
{
\small
\centering
\resizebox{\linewidth}{!}{\import{figures_built/}{commonilluminants.pgf}}
\caption{Spectral distributions of common Standard Illuminants}
\label{fig:illumspectra}
}
\vskip 1mm
{\footnotesize\it Plots of spectral distribution of
Standard Illuminants $A$, $D_{50}$, $D_{65}$, 
normalized to equal luminance.
In the rendition, the color of each curve is taken from the
sRGB coordinates for the corresponding color,
normalized so that the maximum sRGB linear value is $0.7$.
Data in line with~\cite{isocie:11664-2:2022}
The CIE color matching function $\bar y (\lambda)$ is also plotted in
black dashed line for reference}
\end{figure}

\paragraph{Standard Illuminant A}

The standard illuminant A defined by the \gls{CIE} is intended to
represent typical, domestic, tungsten-filament lighting.
Its relative spectral power distribution is that of a black body radiator at a
temperature of approximately \num{2855.54}\unit{\kelvin}.
Standard illuminant A should be used in all applications of colorimetry
involving the use of incandescent lighting, unless there are specific reasons
for
using a different illuminant. Using our previous expression for black body radiation we have
\begin{equation}
S_{A}(\lambda_{\mu m}) = B_{2855.54}(\lambda_{\mu m}) \approx
\frac{119.104}{\lambda_{\mu m}^5} \frac 1{e^{\frac{5.03855}{\lambda_{\mu m}}}-1}
\qquad \left[\unit{\watt\per\square\meter\per\steradian\per\meter}\right]
\end{equation}
where $\lambda_{\mu m}$ is the wavelength in micrometers.

The spectrum of this illuminant is published in~\cite{isocie:11664-2:2022}.

\paragraph{Standard Illuminant D series}

The D series of illuminants were derived by Judd, Mac Adam and Wyszecki, from a
large number of measurements of natural daylight. Although they are rather
difficult sources to build accurately in real life, they are based on a simple
PCA decomposition of the data gathered during the measurements, making them
very convenient to work with numerically.

The spectral distribution $S_D(\lambda, T)$ is defined in terms of three tabulated spectral
distributions $S_0(\lambda)$, $S_1(\lambda)$ and $S_2(\lambda)$ (data is tabulated
and plotted in the appendix):
\begin{equation}
S_D(\lambda, T) = S_0(\lambda) + M_1(T) S_1(\lambda) + M_2(T) S_2(\lambda)
\end{equation}
where the weighting functions $M_1(T)$ and $M_2(T)$ are defined in terms of the
\gls{CIE} chromaticity coordinates $x_D, y_D$ as follows:
\begin{align*}
M   &= 0.0241 + 0.2562 x_D - 0.7341 y_D \\
M_1 &= \frac{-1.3515 - 1.7703 x_D + 5.9114 y_D}{M} \\
M_2 &= \frac{0.03 - 31.4424 x_D + 30.0717 y_D}{M} \\
\end{align*}

Due to changes in constants that have happened since 1931, Standard Illuminant $D_{65}$
corresponds to a temperature of \num{6504}\unit{\kelvin} and Standard Illuminant $D_{50}$
to a temperature of \num{5003}\unit{\kelvin}.
The spectra of these illuminants are published as tabulations in~\cite{isocie:11664-2:2022}.

\paragraph{Chromaticity}
For D-series standard illuminants, \gls{CIE} xy coordinates are:

\begin{equation}
\begin{array}{rcl}
w & = & 1/T_{kK} \qquad \text{inverse of temperature in kiloKelvin}  \\
x & = & \begin{cases}
         - 4.6070\,w^3 + 2.9678 \,w^2 + .09911\,w + .244063  & T_{kK} \in [4, 7] \\
         - 2.0064\,w^3 + 1.9018\,w^2  +  .24748\,w + .237040 & T_{kK} \in [7, 25]
    \end{cases} \\
y & = & -3\,x^2  + 2.87\,x - .275
\end{array}
\end{equation}


\paragraph{Standard Illuminant E}

Standard Illuminant E is simply $S_E(\lambda) = 1$, an equal energy radiator

\paragraph{Standard Illuminant F series}

\begin{figure}
\small
\centering
\resizebox{\linewidth}{!}{\import{figures_built/}{illuminantf1-6.pgf}}
\caption{Spectral distributions of CIE Standard Fluorescent Illuminants}
\label{fig:illumfspectrastd}
\vskip 1mm
{\footnotesize\it Plots of spectral distribution of
Standard Illuminants $F_1$ to $F_6$, standard spectra.
In the rendition, the color of each curve is taken from the
sRGB coordinates for the corresponding color,
normalized so that the maximum sRGB linear value is $0.85$.
}
\end{figure}

\begin{figure}
{
\small
\centering
\resizebox{\linewidth}{!}{\import{figures_built/}{illuminantf7-9.pgf}}
\caption{Spectral distributions of CIE Broadband Fluorescent Illuminants}
\label{fig:illumfspectrabroad}
}
\vskip 1mm
{\footnotesize\it Plots of spectral distribution of
Standard Illuminants $F_7$ to $F_9$, broadband spectra.
In the rendition, the color of each curve is taken from the
sRGB coordinates for the corresponding color,
normalized so that the maximum sRGB linear value is $0.85$.
}
\end{figure}

\begin{figure}
{
\small
\centering
\resizebox{\linewidth}{!}{\import{figures_built/}{illuminantf10-12.pgf}}
\caption{Spectral distributions of CIE Narrowband Fluorescent Illuminants}
\label{fig:illumfspectranarrow}
}
\vskip 1mm
{\footnotesize\it Plots of spectral distribution of
Standard Illuminants $F_{10}$ to $F_{12}$, narrowband spectra.
In the rendition, the color of each curve is taken from the
sRGB coordinates for the corresponding color,
normalized so that the maximum sRGB linear value is $0.85$.
}
\end{figure}

The F series of standard illuminants represent various types of fluorescent lighting.
$F_1$--$F_6$ ``standard'' fluorescent lamps consist of two semi-broadband
emissions of
antimony and manganese activations in calcium halophosphate phosphor, these are
plotted in \cref{fig:illumfspectrastd}. $F_4$ is of
particular interest since it was used for calibrating the \gls{CIE} \gls{CRI}
(the \gls{CRI} formula was chosen such that $F_4$ would have a \gls{CRI} of
$51$).
$F_7$--$F_9$ are ``broadband'' (full-spectrum light) fluorescent lamps with
multiple phosphors, and higher \glspl{CRI}, plotted in
\cref{fig:illumfspectrabroad}. Finally, $F_{10}$--$F_{12}$ are narrow
triband illuminants consisting of three ``narrowband'' emissions (caused by
ternary
compositions of rare-earth phosphors) in the R,G,B regions of the visible
spectrum, plotted in \cref{fig:illumfspectranarrow}.
The phosphor weights can be tuned to achieve the desired \gls{CCT}.


\section{Standard Illuminants}\label{ch:stdilldata}

\subsection{Illuminant D}

Data for CIE Standard Illuminant D, the three functions
are $S_0(\lambda)$ in red, $S_1(\lambda)$ in green, $S_2(\lambda)$ in blue

%\begin{table}
{
\small
\noindent\resizebox{\linewidth}{!}{\import{figures_built/}{illuminantd.pgf}}\\
\setlength{\tabcolsep}{.35em}
\begin{minipage}{.3\linewidth}%
\begin{tabular}{c|r@{.}l | r@{.}l | r@{.}l}
$\lambda$ & \multicolumn{2}{c|}{$S_0(\lambda)$} &  \multicolumn{2}{c|}{$S_1(\lambda)$} &  \multicolumn{2}{c}{$S_2(\lambda)$} \\
\hline
\smsl 300 & \smsl  0&\smsl 04 & \smsl  0&\smsl 02 & \smsl 0& \\
\smsl 310 & \smsl  6&         & \smsl  4&\smsl 50 & \smsl 2& \\
\smsl 320 & \smsl 29&\smsl 60 & \smsl 22&\smsl 40 & \smsl 4& \\
\smsl 330 & \smsl 55&\smsl 30 & \smsl 42&         & \smsl 8&\smsl 50 \\
\smsl 340 & \smsl 57&\smsl 30 & \smsl 40&\smsl 60 & \smsl 7&\smsl 80 \\
\smsl 350 & \smsl 61&\smsl 80 & \smsl 41&\smsl 60 & \smsl 6&\smsl 70 \\
\smsl 360 & \smsl 61&\smsl 50 & \smsl 38&         & \smsl 5&\smsl 30 \\
\smsl 370 & \smsl 68&\smsl 80 & \smsl 42&\smsl 40 & \smsl 6&\smsl 10 \\
380 &  63&40 & 38&50 &  3& \\
390 &  65&80 & 35&   &  1&20 \\
400 &  94&80 & 43&40 & -1&10 \\
410 & 104&80 & 46&30 & -0&50 \\
420 & 105&90 & 43&90 & -0&70 \\
430 &  96&80 & 37&10 & -1&20 \\
440 & 113&90 & 36&70 & -2&60 \\
450 & 125&60 & 35&90 & -2&90 \\
460 & 125&50 & 32&60 & -2&80 \\
470 & 121&30 & 27&90 & -2&60 \\
\end{tabular}
\end{minipage}\hfill
\begin{minipage}{.3\linewidth}
\begin{tabular}{c|r@{.}l | r@{.}l | r@{.}l}
$\lambda$ & \multicolumn{2}{c|}{$S_0(\lambda)$} &  \multicolumn{2}{c|}{$S_1(\lambda)$} &  \multicolumn{2}{c}{$S_2(\lambda)$} \\
\hline
480 & 121&30 &  24&30 & -2&60 \\
490 & 113&50 &  20&10 & -1&80 \\
500 & 113&10 &  16&20 & -1&50 \\
510 & 110&80 &  13&20 & -1&30 \\
520 & 106&50 &   8&60 & -1&20 \\
530 & 108&80 &   6&10 & -1&   \\
540 & 105&30 &   4&20 & -0&50 \\
550 & 104&40 &   1&90 & -0&30 \\
560 & 100&   &   0&   &  0&   \\
570 &  96&   &  -1&60 &  0&20 \\
580 &  95&10 &  -3&50 &  0&50 \\
590 &  89&10 &  -3&50 &  2&10 \\
600 &  90&50 &  -5&80 &  3&20 \\
610 &  90&30 &  -7&20 &  4&10 \\
620 &  88&40 &  -8&60 &  4&70 \\
630 &  84&00 &  -9&50 &  5&10 \\
640 &  85&10 & -10&90 &  6&70 \\
650 &  81&90 & -10&70 &  7&30 \\
\end{tabular}
\end{minipage}\hfill
\begin{minipage}{.3\linewidth}
\begin{tabular}{c|r@{.}l | r@{.}l | r@{.}l}
$\lambda$ & \multicolumn{2}{c|}{$S_0(\lambda)$} &  \multicolumn{2}{c|}{$S_1(\lambda)$} &  \multicolumn{2}{c}{$S_2(\lambda)$} \\
\hline
660 & 82&60 & -12&   &  8&60 \\
670 & 84&90 & -14&   &  9&80 \\
680 & 81&30 & -13&60 & 10&20 \\
690 & 71&90 & -12&   &  8&30 \\
700 & 74&30 & -13&30 &  9&60 \\
710 & 76&40 & -12&90 &  8&50 \\
720 & 63&30 & -10&60 &  7&00 \\
\smsl 730 & \smsl 71&\smsl 70 & \smsl -11&\smsl 60 & \smsl 7&\smsl 60 \\
\smsl 740 & \smsl 77& & \smsl -12&\smsl 20 & \smsl 8&\smsl 00 \\
\smsl 750 & \smsl 65&\smsl 20 & \smsl -10&\smsl 20 & \smsl 6&\smsl 70 \\
\smsl 760 & \smsl 47&\smsl 70 & \smsl -7&\smsl 80 & \smsl 5&\smsl 20 \\
\smsl 770 & \smsl 68&\smsl 60 & \smsl -11&\smsl 20 & \smsl 7&\smsl 40 \\
\smsl 780 & \smsl 65& & \smsl -10&\smsl 40 & \smsl 6&\smsl 80 \\
\smsl 790 & \smsl 66& & \smsl -10&\smsl 60 & \smsl 7&\smsl 00 \\
\smsl 800 & \smsl 61& & \smsl -9&\smsl 70 & \smsl 6&\smsl 40 \\
\smsl 810 & \smsl 53&\smsl 30 & \smsl -8&\smsl 30 & \smsl 5\smsl &50 \\
\smsl 820 & \smsl 58&\smsl 90 & \smsl -9&\smsl 30 & \smsl 6&\smsl 10 \\
\smsl 830 & \smsl 61&\smsl 90 & \smsl -9&\smsl 80 & \smsl 6&\smsl 50
\end{tabular}
\end{minipage}
%\caption{Spectral power distribution data for Standard Illuminant D}
\vskip 1mm
\label{tab:spdillD}
}
%



\section{Angular distributions}\label{sec:angular_norm}

\subsection{Powered cosine}

The angular norm $\|D\|$ of $\langle \omega, \hat n \rangle^n$ is
\begin{multline}\label{eq:powered_cosine}
\|D(\omega)\|
   = \int_{\Omega^+} \langle \omega, \hat n \rangle \cos\theta \d\omega
   = \int_{\phi = 0}^{2\pi} \int_{\theta = 0}^{\pi/2} \cos^n\theta \cos\theta
     \sin\theta \d\theta \d\phi \\
   = 2\pi \left.\frac{-\cos^{n+2}\theta}{n+2}\right|_0^{\pi/2}
   = \frac{2\pi}{n+2}
\end{multline}

\subsection{IES profiles}

The \gls{IES} has a standard file format called \texttt{LM-63} (available in
three revisions \texttt{LM-63-1986}, \texttt{LM-63-1991} and
\texttt{LM-63-1995}) for photometric data describing luminaires. These files
are often called \gls{IES} profiles and tend to be freely available from lighting fixture
manufacturers. For the purposes of this document, an \gls{IES} profile
is an (unnormalized) intensity distribution map $D(\omega) =
D_{IES}(\theta,\phi)$, intended to be a bilinear interpolation of the data in
the corresponding tabulation from the file. As such it can be integrated
numerically evaluating the angular norm integral
\begin{equation}
\|D(\omega)\|
   = \int_{\Omega^+}  D_{IES}(\theta,\phi) \cos\theta \d\omega
   = \int_{\phi = 0}^{2\pi} \int_{\theta = 0}^{\pi/2}
        D_{IES}(\theta,\phi) \cos\theta \sin\theta \d\theta \d\phi
\end{equation}
implementation details are in \cref{ch:implementation}.

\section{Tint functions}

\subsection{Texture maps}

\begin{inconstruction}
Explain texture maps, their primaries, how to use custom primaries.
Explain differences between textured emitters and transmittance/reflectance data
\end{inconstruction}

\subsection{Tabulated spectral data}
Sometimes tabulated data is available for various gels or filters, \cref{fig:roscolux}
includes plots of a subset of Rosco's \emph{RoscoLux} series of gels.

\begin{figure}
{
\centering
\noindent\resizebox{\linewidth}{!}{\import{figures_built/}{roscolux.pgf}}
\caption{Spectral transmittance of \emph{RoscoLux} gel series}
\label{fig:roscolux}
}
\vskip 1mm
{\footnotesize\it Transmittance of the \emph{RoscoLux} series of gels from
Rosco. The coloration of the curve is the color of the gel as seen in front
of a D65 illuminant}
\end{figure}


