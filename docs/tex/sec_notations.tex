% SPDX-License-Identifier: Apache-2.0
% Copyright (c) Contributors to the PhysLight Project.

\chapter{Notation and symbols}\label{ch:notation}
%\addcontentsline{toc}{chapter}{Notation}

\begin{description}
\item[ {$\square$} ] a placeholder to indicate the location of some symbol
\item[ {$[\square]$} ] square brackets indicate the contained symbol is a unit,
	brackets will be empty of \emph{pure numbers}
\item[ {$\square^\uparrow$} ] an arrow pointing up $\uparrow$ is used to indicate 
	exitant quantities, that is leaving the object or location
\item[ {$\square^\downarrow$} ] an arrow pointing down $\downarrow$ is used to indicate 
	incident quantities, that is arriving at the object or location

\item[ {$\delta\square$} ] a small, differential amount of $\square$

% our constants and names
\item[{$\Omega$}] a solid angle
\item[{$\omega$}] a direction, usually within the solid angle $\Omega$
\item[$S$] a light source, or the region where its light originates
\item[$R$] a light receiver, or the region where it receives light
\item[ {$t\;[\unit\second]$} ] a time period
\item[$k_i$] imaging constant: scale factor between luminous exposure and pixel values
\item[$p^{img}$] location on the filmback or image plane
\item[$W_{pos}(p^{img})$] local response of the filmback
\item[$W_{col}(\lambda)$] spectral response of the filmback
\item[$W(p^{img},\lambda)$] film response function

\item[ {$\lambda\;[\unit\meter]$} ] wavelength
\item[ {$\nu\;[\unit\hertz]$} ] frequency

% general photography quantities and constants
\item[{$C$}] incident-light meter calibration constant $312.5 = 100 \sfrac{25}8$ for this document
\item[{$f$}] focal length
\item[{$N$}] aperture number
\item[{$o$}] focus distance
\item[{$\Delta t$}] exposure time
\item[{$S$}] film speed
\item[{$T_{cp}$}] \gls{CCT} of a white point
\item[{$N$}] aperture number



% from lighting standard vocabulary
\item[ {$N_p\;[]$} ] number of photons
\item[ {$Q_p\;[\unit\joule]$} ] energy of one photon
\item[ {$Q_e\;[\unit\joule]$} ] radiant energy


% from SI
\item[ {$h$} ] Planck's constant (see~\cite{bipm:si.2019})
\item[ {$K_{cd}$} ] luminous efficacy of monochromatic radiation (see~\cite{bipm:si.2019})

\item[ {$\unit\joule$} ] unit: joule, measures amounts of energy
\item[ {$\unit\watt$} ] unit: watt, measures power, being energy per 
	unit of time $[\unit\watt] = [\unit{\joule\per\second}]$
\item[ {$\unit\hertz$} ] unit: watt, measures power, being energy per 
	unit of time $[\unit\hertz] = [\unit{1\per\second}]$
\item[ {$\unit\radian$} ] 	unit: radian, measures angles
\item[ {$\unit\steradian$} ] unit: steradian, measures solid angles
                   
\end{description}

\paragraph{Vector-valued functions and integrals}

\begin{inconstruction}
	Explain the meaning of this expression
	\begin{displaymath}
		\begin{pmatrix}X\\Y\\Z\end{pmatrix}
		= \int_\Lambda S(\lambda)\; 
		\begin{pmatrix}
			\bar x(\lambda)\\
			\bar y(\lambda)\\
			\bar z(\lambda)
		\end{pmatrix}
		\d\lambda
	\end{displaymath}
\end{inconstruction}


\paragraph{Scalar product}

Given two vectors $x,y \in R^n$ we will indicate their components using a
subscript index,
for example for $n=2$ we would have $x = (x_1, x_2)$.
We will write their \textsl{scalar product} using angle bracket notation
$\langle \cdot, \cdot \rangle$:
\begin{displaymath}
\langle x, y \rangle = \sum_{i=1}^n x_i y_i = xy^t,
\end{displaymath}
where the second equality signifies row-vector notation (vectors are thought
of as matrices of one row). Sometimes it's useful to inject an $n\times n$ matrix $M$ ``in the middle''
of a scalar product, defined as follows $\langle x, y \rangle_M := \langle M x, M y \rangle$.

An often useful construct is the following relation: given
three vectors $x,y,z \in \R^n$
\begin{displaymath}
\langle x,y \rangle \langle y,z \rangle
  = \left(x y^t\right)\left(y z^t\right)
  = x \big(y \otimes y\big) z^t
  = \langle x, z\rangle_y
\end{displaymath}
where the symbol $\otimes$ indicates the \textsl{outer product} defined as $x\otimes y :=
x^t y$. Here is an example in $\R^3$:
\begin{displaymath}
x\otimes y =
\left[
\begin{array}{ccc}
x_1 y_1 & x_1 y_2 & x_1 y_3 \\
x_2 y_1 & x_2 y_2 & x_2 y_3 \\
x_3 y_1 & x_3 y_2 & x_3 y_3 \\
\end{array}
\right].
\end{displaymath}


\paragraph{Scalar product of functions}
Given two functions $f,g:X\to\R$ we define their scalar product
$\langle f, g \rangle$ as
\begin{displaymath}
\langle f, g \rangle := \int_X f(x)g(x) \d x, \qquad \langle f, g \rangle_\mu := \int_X f(x)g(x)\mu(x) \d x.
\end{displaymath}

\begin{inconstruction}
	Discuss how scalar product and dot product are connected
	Discuss how scalar product and inner product are connected
	Explain how to integrate a vector-valued function like $W_{\XYZ}(\lambda)$ 
\end{inconstruction}

\paragraph{Norm of a function}
Given a function $f(x): X\to\R$ we will often speak about the integral norm over its
domain $X$, sometimes after weighting with a second positive valued function
$\mu(x): X\to\R^+$, this will be notated as follows
\begin{displaymath}
\|f\| := \int_X |f(x)| \d x, \qquad \|f\|_\mu := \int_X |f(x)| \mu(x) \d x
\end{displaymath}
it is sometimes useful to consider the $p$-norm of a function $f$:
\begin{displaymath}
\|f\|^p := \sqrt[p]{\int_X |f(x)|^p \d x}, \qquad \|f\|_\mu^p := \sqrt[p]{\int_X |f(x)|^p \mu(x)\d dx}.
\end{displaymath}

In the case of a function $g(x): X\to\R^n$ valued in $\R^n$ we apply the
relations above to
each component separately, for example, in the case of $n = 2$ we would have
$g(x) = (g_1(x), g_2(x))$, where $g_1,g_2:X\to\R$
\begin{displaymath}
\|g\|_\mu := \left(\|g_1\|_\mu, \|g_2\|_\mu\right) \in \R^2.
\end{displaymath}
