% SPDX-License-Identifier: Apache-2.0
% Copyright (c) Contributors to the PhysLight Project.

\chapter*{Foreword}\label{ch:foreword}
\addcontentsline{toc}{chapter}{Foreword}

\epigraph{%
	\emph{I constantly asked myself `What would it really be like to photograph
		this if it was a real event?' [... I wanted the] effect of making you feel
		as if you were witnessing something real, rather than \acronym{cg}}}
{\textsc{Jon Favreau}, 2008}

\noindent This document is the result of a prolonged phase of soul-searching 
started in the summer of 2014. As it often happens, it all started with the classic,
seemingly innocent question, in our case in an e-mail from our supervisor Joe Letteri
at \companyname{W\=et\=a FX} (called \companyname{Weta Digital} at the time).
Production on \cite{hobbit3} was in its final stretch, and while the artists at the studio
were tremendously busy with the visual effects work for the show, the research groups
started preparing for the next wave of work. 

In our opening quote\footnote{See \cite{duncan08}} director Jon Favreau summarizes one of the key 
tenets in the aesthetic posture of \gls{VFX} work in our time; as much as he is talking
about camera work and how a given fast-moving action ought to be portrayed in a movie, 
there is a strong feeling from his words that the expectation is also that the integration between 
live-action photography and \gls{CGI} should be completely seamless.
The implication is that \gls{VFX} production is an integral part of the movie-making process and toolkit, 
and as such it ultimately and rightfully falls under the control of the director, in charge of content, 
performance and the overall vision for the show, and the \gls{DOP} (also known as 
the \textsl{cinematographer}), in charge of all the photography aspects. 
These two being the key creators of the motion picture. 
This naturally presents a valuable opportunity to support and assist them in fulfilling the vision
they have for the picture by providing a set of controls that under a unified language would span the 
real world as well as the \gls{virtual} world.
It seems that the \gls{VFX} crew should be part of the movie-making process and community supporting,
expanding and complementing its established language and method, evolving it as appropriate.
In particular, established photography and cinematography practice, with its huge 
legacy captured in books such as~\cite{alton1995} and~\cite{lowell1992} as well as
the \emph{American Cinematographer Manual}~\cite{burum2007}, would be augmented into the \gls{virtual} world
providing software tools that felt and responded in the same way as their real-life counterparts.

There is an observation to be made here, that the aesthetic of a picture (moving or not) is born from
an act of combination where the skill and sensitivities of the artists come together with the 
finite possibilities of the medium in use, which in a traditional motion picture context comprises 
the characteristics of lenses, cameras, and film stock. 
It is usually the job of the \gls{DOP} on a show to make a choice in picking which lenses, which cameras 
and which film stocks to work with, as a result of serving the aim of the director while working with 
what is available at the time the movie is made\footnote{This is both because of external limits such as 
	budgetary constraints as well as inherent limits, like the availability of specific piece of equipment 
	at a given point in history. It's clear that Orson Welles, Stanley Kubrick and Christopher Nolan, 
	as well as their respective \glspl{DOP}, worked with very different technology, and budgets}. 
The advancement of technology provides the \gls{DOP} with some kind of a traveling 
window of opportunity in terms of what can be achieved: new lenses are invented that are maybe sharper,
or exhibit more clarity, new film stock with different grain characteristics or color capability goes into
production, and older technology is gradually retired.
This is an element behind the evolution of the aesthetics of motion pictures, and it exists alongside the
evolution of the taste for certain visual elements that audiences and creators go through in the course of
history.
So as the palette of the \gls{DOP} evolves, it is the job of the \gls{VFX} side of things to stay
aligned with its capabilities as closely as possible, including both what might be considered
quality elements of a given piece of equipment (for example, a lens that might be very sharp),
as well as what could be instead considered a defect or shortcoming of it: both sides must be
matched for the integration between live-action and digitally-generated material to be seamless. 

Maybe the most obvious effect of this phenomenon where ``defects'' are turned into creative opportunities is
the use of lens flares. 
During the 1960's what used to be a capital sin of photography (``flaring the lens'') was gradually
turned into a cinematic device, starting with Hiroshi Teshigahara's \cite{teshigahara64} and then through pictures 
such as \cite{graduate67}, \cite{easyrider69}, \cite{fiveeasypieces70}, 
establishing a new aesthetic that would then be adopted into special-effect shows such as \cite{spaceodyssey68}, 
\cite{starwars77}, \cite{closeencounters77} and ended up becoming a staple of modern effects-heavy 
cinematography, for example as seen in Spielberg's work such as \cite{raiders81}, \cite{et82} or even
\cite{goonies85}, and all the way to J. J. Abrams's \cite{startrek2013}.

There is certainly a tradition in photography to take what might appear to be shortcomings
or defects in an image and turn them into creative tools in the photographer's kit.
The classic example is working with portions of an image to be in more or less sharp focus,
and this is certainly part of an aesthetic discourse which has seen much evolution over time: 
in the early days of photography there was a general taste for an overall ``soft focus'' look
to photographs, then in the 1930's Ansel Adams, Paul Strand and other giants of early American 
still photography started making pictures that were in perfect focus over the whole frame 
(they used very small lens apertures to achieve this, forming a current called 
\emph{Group $\sfrac{f}{64}$} that was very influential in its time). 
The artistic possibilities resulting from a deliberate use of the limited \gls{depth of field} 
achievable with wider \glspl{aperture} started to be explored by more and more photographers,
as it became more common to use smaller-format cameras inherently less capable of producing the
ultra-sharp images typical of large-format photography.
Shallower depth of field is typically employed as a mean to direct the observer's eye towards 
a region or another of the picture (not necessarily the sharp one), and convey specific meaning to an image. 
A similar evolution in the tastes of photographers and viewers has happened with regard to \gls{motion blur}:
at first it was considered a defect and typical of amateur work, 
but then it found its way in higher and higher levels of artistic work.

For what concerns motion pictures, the cinematographer working with physical film stock has more limited 
options when compared to the still photographer: the film contains 24 frames per second, and the motion
picture camera needs a certain amount of time to advance the film strip to the position for the next frame,
so usually motion picture cameras are limited to a maximum exposure time of $\sfrac1{48}$ of a second.
Motion picture cameras mostly employ what's called a \gls{rolling shutter}, being a disk that spins 
at constant speed interposed between the lens and the film. 
On this disk two ``blades'' are mounted, each covering half of the circle and spanning an angle of $180$ degrees each. 
The camera technicians on-set have provisions to mount these two blades so they leave a 
wedge-shaped gap where light can pass through as the disk spins.
So if for example they were to leave a gap of $90$ degrees, the film would only get light for a 
duration of a quarter of each frame, being an exposure time of $\sfrac1{96}$ of a second.
Because of how ubiquitous a rolling shutter is for cinema work, the language established in the field
talks about exposure times in degrees rather than in seconds, using phrases like 
``this take was shot on a 90 degree shutter''.
The amount of motion blur in a shot has a specific look that audiences feel: in fact it's 
always a balance between images being sharp, while maybe \gls{strobing} a little, versus being 
softer with reduced \gls{strobing}. The strobing effect comes from our eyes seeing the moving objects
jump from a position to their next, while the motion blur ``trails'' help the eye connect the 
movement more smoothly, appeasing the visual sense of the viewer.
Once a \gls{shutter} angle has been picked for a shot, there is a second delicate balancing act
to have: the \gls{aperture} is chosen to allow a depth of field appropriate for the intent
of the \gls{DOP} and director, but limited by the constraints of available light on set.
There is certainly a possibility to pick film stock with sensitivity better matching the need
of the moment, but film stock is only made in relatively few speeds, and often times a large change
in sensitivity implies a change in look that may not be appropriate for the moment, as it could
induce an undesirable visual discontinuity in the edited scene.

These problems were reduced with the advent of digital movie postprocessing:
the early experiments in \cite{pleasantville98} and \cite{whatdreams98}, exploring the possibilities
opened by the new \gls{digital intermediate} process, solidified very quickly into a completely rebuilt
and reconceived post-production method, allowing entirely new looks as exemplified in \cite{brother2000}
first and then \cite{amelie2001} and \cite{moulin2001}. In fact soon after the digital intermediate process
took hold, increasingly advanced software methods for adding motion blur and aperture blur 
(ie. limiting the depth of field in the image) after the fact continued to appear.
It is interestingly to observe how the two enjoyed very different adoption in practice, the origin
of the phenomenon lying in movie-making practice. The observation is that motion blur is very
rarely corrected in post production, while depth of field is very often modified in the late
stages of compositing. And this comes directly from the facts that sharpness of focus is treated 
very specifically and with great attention by the \gls{DOP}, while  motion blur is more of a 
not-completely-welcome guest at the table. And in turn that descends from the former being
easily consciously observable by artist and audiences, while the latter necessarily 
passes by very quickly and remains (at least for the mainstream audience) at a less conscious
level of perception.



The important question as to whether this is, or should be, the \emph{one true way} of using \glsname{CGI} in 
movie-making is explored with great depth in the essay \emph{The empire of effects}~\cite{turnock22}. 
It turns out the conclusion is far from obvious, and Turnock's essay goes to great lengths to explore the
origin of this state of things and what would be possible alternatives.
When connecting these thoughts with the task of providing \gls{VFX} material for a movie,
the inevitable conclusion is then that it will be highly desirable to employ as accurate a model of the 
photographic process as possible, including the ``defects'', because in the movie-maker's hands really there
are no ``defects'' as such, rather there are \emph{traits} of a system used to make motion pictures,
and these are employed with conscious deliberation, much the same way as colors and their shades 
would be picked with a brush from the physical palette of a painter.




For what concerned our own quest at \companyname{W\=et\=a FX}, one of the recurring difficulties with 
producing \gls{VFX} work at the time of \cite{hobbit3} was to accurately reconcile 
the overall brightness observed in the images from the principal photography camera
with the survey data gathered by the on-set lighting crew and
the brightness of the lighting fixtures captured by our reference photographers.
At that time our reference photography crew would capture essentially two data sets
that would later be used for \gls{virtual} lighting: the first set of photographs would be 
images of the ``faces'' of all light fixtures used for each shot. 
These would later be used to improve the fidelity of reflections in high
gloss materials, such as car paint and other polished surfaces, as well as anything wet, 
including the very important case of eyes in digital character faces. 
The second set would be wide-angle panoramas of the set as it stood just before shooting began,
which would later be used to construct \gls{IBL} sources\footnote{ 
	A lighting technique pioneered by Joe Letteri, John Knoll and several colleagues from
	\gls{ILM} in the late 1990's and early 2000's on shows such as 
	\cite{casper95}, \cite{speed297}, \cite{pearlharbor2001}, \cite{lotr-tri}}. 
The use of \gls{IBL} sources was crucial to provide our rendered images with the finessed tones 
of the light from the principal photography, enormously enhancing the accuracy of the \gls{virtual}
lighting with respect to its original real-world counterpart.

So as much as we had high quality, high dynamic range captures of all these elements,
until that point it had never occurred to us (or for what we know, anyone else in our
line of work) to also pin down their brightness with an absolute metric. 
All the contrary: it was customary at that time to normalize these captured images so their 
brightest point was at a known pixel value, and let the rest of these images behave linearly 
with respect to that, in the sense that region of the real-life scene that were half as bright
would have pixel values half as high. This would be in line with the idea that we were after 
\gls{scene-referred} data, which is intended to be \emph{proportional} to real-world values from
the scene~\cite{iso:12231-2:2022,iso:22028-1:2016,iec:60050-845:2020,icc:glossary}.

Early in the life of a sequence, the \gls{VFX} lighting crew would set up some reference configurations
matching the live-action state and make numerous test renders to reconcile the look of rendered
digital elements with their corresponding counterparts from the principal photography plates. 
Source data for this process is often seen in images about \gls{VFX} workflows: three images are captured including
a ``grey ball'', being a matte, middle-gray sphere, a ``chrome ball'', being a mirror-reflecting
sphere, and a ``MacBeth Chart'', a color calibration target made of 24 colorful squares of known 
reflectance properties.
They are used to capture various aspects of the coloration and distribution of light on set,
and they are comparatively easy to build into \gls{virtual} assets, so that photography and renders 
can be compared with some hopes of accuracy.
However, due to the nature of on-set work, and the realities arising when several separate crews 
all need to work as quickly and efficiently as possible, 
this reconciliation process entails a substantial amount of 
judgment to achieve a match robust enough to cover the needs of entire sequences. 
This turns into a large amount of time spent on the task by the lighting crew, 
made more problematic by the fact that the delicacy of the task ends up requiring
senior crew, which would be better spent elsewhere.

At some point it finally dawned on us that if we could measure the brightness
of the lights themselves, we could simply roll the values into our survey data 
and we would then be able to tie down one more loose string, 
and do away with the lengthy process of manual light reconciliation. 
And so it began, with the innocent question: ``How should we capture light 
brightness in our surveys? Should it be watts, candelas, footlamberts?".


The renderer in use at \companyname{W\=et\=a FX}'s, \productname{Manuka}, is a spectral renderer
which gives it a very accurate color management pipeline. 
So we were starting this work sitting on an opportunity to substantially improve the fidelity of 
our \gls{virtual} lighting process, by transporting more directly information from our on-set 
surveys all the way into the \gls{virtual} scenes.
It felt like an important improvement to our process, because it would allow us to free up 
senior lighting crew, enabling them to spend their time on creative work directly contributing into
realizing the vision for the picture, rather than on what always felt like an annoying calibration step.
There was the further added bonus that using measured data would also improve our ability to match 
more subtle aspects of our images.

It was pleasing for us to see how well the process worked out in practice: 
after a couple years of work, this document reached its ``Version 1.0'' status, 
and the system was adopted for work on \cite{wfpota2017} and \cite{valerian2017}. 
The shows feature cinematography by Michael Seresin and Thierry Arbogast respectively and were shot 
in vastly different visual styles: Seresin's photography includes long outdoor sequences in 
snow-covered mountains featuring a very delicate, naturalistic look, while Arbogast's work 
in \emph{Valerian} is all about different worlds from all around the galaxy, full of 
overly saturated colors and bright, intense primaries everywhere.
Our reference photography crew on both productions recorded absolute brightness data and spectral profiles
for all the luminaires employed on the show, alongside the usual images of their faces, and this dataset
was then applied during the production of the \gls{VFX} work. 
The resulting images on both shows were produced with a more robust process with reduced guesswork, 
and naturally fell into alignment with the live-action photography right from the get go, as is quite
visible in the finished pieces.





Once we had finished building a software system capable of very accurately simulating 
all the aspects of a real-life camera, and having complemented it with a set of controls
that corresponded very accurately with their real-life counterparts, we had in our hands 
a virtual photographic camera and as much as this enabled us to match very
closely the images our colleagues from live-action disciplines created on-set,
we also had an opportunity to propose the use of this system for all sorts of predictive rendering applications,
as could be useful in virtual product design work (as famously exemplified in the IKEA catalogue, for example),
architectural rendering (especially for lighting design) and all manners of similar applications,
inspired by the increasingly famous \emph{digital twin} concept~\cite{gelernter91},
which over the last couple decades has been penetrating and revolutionizing the
industrial world~\cite{grieves02}.
Given the impact of these changes, we decided to share our findings with everybody else through an
open-source project.



\ifomit


color responsivity in XYZ or in camera RGB?
exposure vs density curve plots




test: try the siggraph font styles
test: try Libertine for body text

\fi