% SPDX-License-Identifier: Apache-2.0
% Copyright (c) Contributors to the PhysLight Project.

\chapter{Reference implementation}\label{ch:implementation}

Although light transport simulations are fundamentally spectral, at present
the majority of the commercially available image synthesis systems have been
based on tristimulus \gls{RGB} transport.
In recent times, especially as facilitated by the increasing availability of the
OpenColorIO library, these systems have evolved to implement an ability to receive
data in a number of different color spaces, transforming it all into a common
color space used for computation, and then being able to produce images in a
target color space specified by the user. Some of these systems are also
capable of transforming spectral data into tristimulus values.
In the spectral approach, all light transport computations happen in the
spectral domain and the transformation into the imaging color space
happens at the end of the pipeline. This requires a facility to lift \gls{RGB}
input data to the renderer to some spectral representation to handle inputs
that are \gls{RGB} to start with, such as common texture maps.

There are various reasons why one might choose an approach versus the other,
and image synthesis systems that have had a long history are most often
tristimulus-based. Further, as the computational and storage requirements
of a tristimulus approach tend to be somewhat lighter, this is convenient on
less powerful machines or systems that are otherwise constrained in compute
resources, or maybe need a very high frame rate and can
sacrifice some accuracy to gain performance. On the other
hand, a number of light phenomena such as dispersion or an accurate accounting
of metameric effects are expensive to simulate accurately in a tristimulus-based
system, while they simply happen as a part of the system in a spectral system.

\section{Tristimulus rendering}

\def\colspacefont#1{\texttt{#1}}
\def\camRGBl{\ensuremath{\colspacefont{cameraRGB}_\ell}}
\def\sRGBl{\ensuremath{\colspacefont{sRGB}_\ell}}
\def\sRGBg{\ensuremath{\colspacefont{sRGB}_\gamma}}
\def\RecSONl{\ensuremath{\colspacefont{Rec709}_\ell}}
\def\RecSONg{\ensuremath{\colspacefont{Rec709}_\gamma}}



All the following assumes the \gls{RGB} working space is \colspacefont{sRGB}
linear scene referred, which will be indicates \sRGBl. When gamma corrected
values are used, their color space will be subscripted with a $\gamma$ such as
\sRGBg. As \colspacefont{sRGB} and \colspacefont{Rec709} use the same primaries,
\sRGBl\ is effectively the same as \RecSONl, whereas \sRGBg\ and \RecSONg\
differ due to the different definition of their gamma correction formulation.


\subsection{Area lights}

\paragraph{Tint vector}

The tint vector is the average value of the texture map, computed in \sRGBl\
space. Assuming the file actually contains data \sRGBl\ space, this can be
obtained either with a \gls{RSL} call like

\begin{rslcode}
color tint = texture(
                filename,
                0.5, 0.5,         // access at center
                "blur", 1.0,      // filter the entire texture into the result
                "filter", "box"); // use box filtering (compute average)
\end{rslcode}
or the average value can be computed with code similar to the following, in
which the incremental average formulation from~\cite{knuth97} is used

\begin{c++code}
// assume type float3 supports standard vector-like arithmetic operators
float3* image = ...;
float3 tint = float3(0.f, 0.f, 0.f);

for (int i = 0; i < pixelcount; ++i)
{
    tint += (image[i] - tint) / float(i+1);
}
\end{c++code}

\paragraph{Reduced luminance vector}

Introduce sensitivity primaries here

Compute $\|col\cdot \hat L\|_{\bar y}$: we know that $col$ is
our \texttt{CameraRGB} space $col = (\bar r(\lambda), \bar g(\lambda), \bar b(\lambda))$,
so the integrals can be precomputed for the supported cases with
code like this (for example in the case of $\hat L(\lambda)$ being Illuminant D50):

\begin{pythoncode}
# load spectral sensitivity of the camera
# this is effectively $col$
camera = Camera.Camera("Red_Mysterium_X")
# get CIE 1931 Color Matching Functions
cmf = CIE.CMF_1931
# get Illuminant D50 data: Illuminant D at 5003 K
# this is $\hat{L}$
illuminant = SPD.IlluminantD.at(cmf.lnms, 5003)
# reduced luminance vector $\|col\cdot \hat{L}\|_{\bar{y}}$
redlum_camRGB = [0,0,0]

# tp\_integral() is the triple product integral
# $\int \bar{r}(\lambda) ill(\lambda) \bar{y}(\lambda) \d\lambda$
redlum_camRGB[0] = tp_integral(camera.rbar, illuminant, cmf.ybar)
redlum_camRGB[1] = tp_integral(camera.gbar, illuminant, cmf.ybar)
redlum_camRGB[2] = tp_integral(camera.bbar, illuminant, cmf.ybar)

# now transform into \sRGBl\ primaries, M is CameraRGB to \sRGBl
M = [[  1.369,  0.0137, -0.3068],
 [  -0.18,   1.326, -0.1541],
 [0.01454, -0.5008,   1.331]]
redlum_sRGB = M * redlum_camRGB

\end{pythoncode}


\paragraph{Angular norm}

From \cref{sec:angular_norm} we see that the angular norm $\|D\|$ of
$\langle \omega, \hat n \rangle^n$ is $2\pi / (n+2)$, which is easily
implemented as

\begin{c++code}
float cospower = /* from light parameters */
float angularnorm = (2.f * M_PI) / (2.f + cospower);
\end{c++code}

\Gls{IES} profiles can be integrated using 2-dimensional Newton-Cotes quadrature
of degree 1 (also called \textsl{trapezium rule}) because the error introduced
in the sampling of the luminaire dominates the error in the sampling of the
$\cos\theta\sin\theta$ factor in the integral. What's called ``vertical angle''
in the \gls{IES} profile specification corresponds to angle $\theta$ and what's
called ``horizontal angle'' corresponds to $\phi$, this means the rows in the
``candela values'' data in the file are constant-$\phi$ measurements of the
described luminaire.

\paragraph{Emission constant}

The emission constant \Verb|k_e| is computed as

\begin{c++code}
const float K_m = 683.002;

float Phi_v = /* luminous power from light parameters */;
float A     = /* area of the light from geometry */;

float  angularnorm = /* computed as above */;
float3 tint        = /* computed as above */;
float3 redlum_sRGB = /* computed as above */;

float k_e = Phi_v / (K_m * A * angularnorm * dot(tint, redlum_sRGB));
\end{c++code}


\paragraph{Reduced radiance}

To compute radiance we need to first compute the reduced radiance $\|col\cdot
\hat L\|$: we know that $col$ is our \camRGBl\ space $col = (\bar r(\lambda),
\bar g(\lambda), \bar b(\lambda))$, so the integrals can be precomputed for the
supported cases with code like this (for example in the case of $\hat
L(\lambda)$ being Illuminant D50):

\begin{pythoncode}
# load spectral sensitivity of the camera
camera = Camera.Camera("Red_Mysterium_X")
# get CIE 1931 Color Matching functions
cmf = CIE.CMF_1931
# get Illuminant D50 data, Illuminant D at 5003 K
illuminant = SPD.IlluminantD.at(cmf.lnms, 5003)
# reduced radiance vector, r, g, b components
redrad_camRGB = [0,0,0]

# p\_integral() is the product integral
redrad_camRGB[0] = p_integral(camera.rbar, illuminant)
redrad_camRGB[1] = p_integral(camera.gbar, illuminant)
redrad_camRGB[2] = p_integral(camera.bbar, illuminant)

# now transform into sRGB primaries, M is CameraRGB to sRGB
M = [[  1.369,  0.0137, -0.3068],
 [  -0.18,   1.326, -0.1541],
 [0.01454, -0.5008,   1.331]]
redrad_sRGB = M * redrad_camRGB

\end{pythoncode}

\paragraph{Radiance from the source}

At this point the radiance \Verb/radiance/ from this source is simply

\begin{rslcode}
float k_e = /* computed as above */;
color redrad_sRGB = /* computed as above */;

color radiance =
    k_e *
    texture(filename, s, t) *
    redrad_sRGB *
    pow(normalize(L).normalize(N), cospower);
\end{rslcode}

\subsection{Image Based Lighting sources}

\begin{inconstruction}
    Add code defining what is in \Verb|radiance|

    \paragraph{Irradiance coloration vector}

    Compute the average of the \gls{IBL} map times $\cos\theta$ in \sRGBl. Be
    careful with long sums, mipmapping can't be used. Use Knuth-style
    incremental averaging (probably in double for accuracy).

    \paragraph{Tint vector}

    Compute $\|T \cdot col\|_{\bar y}$
\end{inconstruction}

\subsection{Sun lights}

\begin{inconstruction}
Add code defining what is in \Verb|radiance|

    \paragraph{Angular norm}

    The norm of $D_{\omega_c}$ need to be computed, write down a few common
    cases

    \paragraph{Tint vector}

    Compute the average of the tint texture map in \gls{sRGB}.
    Be careful with long sums, either use mipmapping or Knuth-style
    incremental averaging (probably in double for accuracy).

    \paragraph{Reduced illuminance vector}

    Compute $\|col\cdot \hat L\|_{\bar y}$: we know that $col$ is
    \gls{CIE} \gls{XYZ} space, so the integrals can be precomputed
    for the supported cases, and stored somewhere.
    This needs some attention to detail
\end{inconstruction}


\subsection{Materials}

Materials in \gls{RGB} renderers operate as if there were only three wavelenghts in
the world, that is picking \sRGBl\ data from the textures, converting it to the
working color space and proceeding to multiply component by component such values
with the incoming radiance to produce outgoing radiance.

%\begin{inconstruction}
% Give an example showing how to compute \Verb|Ci| for a simple material
%\end{inconstruction}

\subsection{Imaging}

This section outlines how to apply the imaging parameters to the resulting value
\Verb|Ci| containing the \gls{RGB} radiance flowing towards the image place.
White balance is achieved diving out the whitepoint color in \camRGBl coordinates.
This is implemented as a matrix multiply for convenience, which does in one go
the whole operation of going from working space, to camera space, multiplying by
 the inverse of the white point, and transforming back. Such matrix is built in
 the helper method
\Verb|buildWhitebalanceTransform()| outlined below.

The binding in a \productname{RenderMan} system can be implemented by means of a
pixel-sample imager shader, bound with the call \Verb|RiPixelSampleImager()|

A minimal version of such a shader follows, built on the convention that the
incoming \Verb|Ci| contains radiance, as above. The code of the shader has been
separated into a few code sections for clarity.

\begin{rslcode}
class physLightImaging(
    // camera focal length, in mm
    float focalLength = 24;
    // focus distance, in scene units
    float focusDistance = 100;
    // aperture f-number
    float apertureNumber = 8;
    // exposure time, in seconds.
    // -1 picks it up from the RIB
    float exposureTime = -1;
    // film speed, ISO/ASA
    float filmSpeed = 100;
    // white point, in Kelvin
    float whitepoint_T = 6500;
)
{
    constant matrix c_whitebalance;
    constant float  c_imagingRatio;

    // return the color of a blackbody of temperature T
    // in cameraRGB\_l space
    color blackBody2CameraRGBl(float T) { /* ... */ }

    // return the size of one meter in scene units
    float getOneMeter() { /* ... */ }

    // get the exposure time from a shader parameter or from the RIB
    float getExposureTime() { /* ... */ }

    // get the aperture distance
    float getApertureDistance(float f; float o) { /* ... */ }

    // build a tranformation to apply whitebalance in cameraRGB\_l space
    matrix buildWhitebalanceTransform() { /* ... */ }

    // rebalance the color for a white point at the given temperature
    color apply_whitebalance(color C)
    {
        return (color) transform(c_whitebalance, (point) C);
    }

    public void construct() { /* ... */ }

    public void imager(output varying color Ci, Oi)
    {
        // apply whitebalance
        Ci = apply_whitebalance(Ci);
        // apply exposure parameters
        Ci *= c_imagingRatio;
    }
}
\end{rslcode}


The \Verb|construct()| method initializes the white balance transform and the
imaging ratio. These are then used in the \Verb|imager()| method to convert
incoming radiance into appropriately scaled irradiance at the filmback.

\begin{rslcode}
public void construct()
{
    c_whitebalance = buildWhitebalanceTransform();

    float oneMeter = getOneMeter();
    float exposureTime = getExposureTime();
    float f = focalLength / 1000;
    float o = focusDistance / oneMeter;
    float a = getApertureDistance(f, o);
    float r = f / (2 * apertureNumber);
    float apertureRatio = PI * r*r / (a*a);

    // the imaging constant, \cref{eqn:imaging_ki}
    float k_i = 4 * 683 / 312.5;

    // scale as per \cref{eqn:imaging_y2}
    // note the missing division by $K_m$ as here Ci is already in
    // radiometric units
    c_imagingRatio = c_exposureTime * k_i * filmSpeed * c_apertureRatio;
}
\end{rslcode}

The helper method \Verb|getOneMeter()| returns the number of scene units in a
meter using the \gls{RIB} option \Verb|units:length|

\begin{rslcode}
// get the size of one meter in scene units
float getOneMeter()
{
    float lengthscale = .01;
    option("units:length", lengthscale);
    return 1 / lengthscale;
}
\end{rslcode}


The helper method \Verb|getApertureDistance()| computes the distance of the
aperture from the filmback, in the notation of \cref{ch:imaging}

\begin{rslcode}
// get the aperture distance
// f - the focal length
// o - the focus distance, from the filmback
float getApertureDistance(float f; float o)
{
    o /= 2;
    return o - sqrt(o*o - 2 * f * o);
}
\end{rslcode}


The helper method \Verb|getExposureTime()| computes the exposure time using
either the \gls{RIB} option \Verb|units:time| or the shader parameter
\Verb|exposureTime| in case it's not the sentinel value \Verb|-1|

\begin{rslcode}
// get the exposure time from a shader parameter or from the RIB
float getExposureTime()
{
    float result = exposureTime;
    if (result == -1)
    {
        float timescale = 1/24;
        option("units:time", timescale);
        float shutter[2] = {0,0};
        option("Ri:Shutter", shutter);
        result = (shutter[1] - shutter[0]) * timescale;
    }

    return result;
}
\end{rslcode}


White balance is achieved transforming into \camRGBl, divide component by
component by color of illuminant (also in \camRGBl) and transforming back into
\sRGBl.

\begin{rslcode}
matrix buildWhitebalanceTransform()
{
    // this is M\_MysteriumX from Section B.2, transposed because
    // RenderMan is row-vectors
    uniform matrix cam2sRGB = {  1.369,  -0.18,   0.01454, 0,
                                 0.0137,  1.326, -0.5008,  0,
                                -0.3068, -0.1541, 1.331,   0,
                                 0,       0,      0,       1};

    uniform matrix sRGB2cam = 1 / cam2sRGB;

    // white point, in cameraRGB
    uniform color whitepoint = 1 / blackBody2CameraRGB_l(whitepoint_T);
    uniform matrix wpm = {  whitepoint[0], 0,  0,  0,
                            0,  whitepoint[1], 0,  0,
                            0,  0,  whitepoint[2], 0,
                            0,  0,  0,             1};

    c_whitebalance = cam2sRGB * wpm * sRGB2cam;
}
\end{rslcode}

The \Verb|blackBody2CameraRGB_l| method returns the color of a black body
radiator in \camRGBl\ space and is here in python:

\begin{pythoncode}
# return the color of a blackbody of temperature T
# in cameraRGB\_l
def blackBody2CameraRGB_l(T):
    camera = Camera.Camera("Red_Mysterium_X")

    illuminant = SPD.BlackBody.at(camera.lnms, 6500)
    illuminantE = SPD.IlluminantE.at(camera.lnms)
    illXYZ = ColorSpace.XYZ_lv(camera.lnms, illuminant)
    illEXYZ = ColorSpace.XYZ_lv(camera.lnms, illuminantE)

    illuminant /= illXYZ[1]
    illuminant *= illEXYZ[1]

    bbCameraRGB = ColorSpace.RGB_lv(camera.lnms, illuminant, camera)
    bbCameraRGB /=  ColorSpace.RGB_lv(camera.lnms, illuminantE, camera)

    return bbCameraRGB
\end{pythoncode}



\section{Spectral setting}

The typical assumption in the tristimulus setting is that transformations
between the various color spaces has no adverse effects, and that it is
appropriate to use the \gls{CIE} $\bar x(\lambda), \bar y(\lambda), \bar z(\lambda)$
color matching curves to map from spectral to tristimulus \gls{XYZ} space.

We introduce the use of \colspacefont{cameraRGB} color space, the color space
determined by spectral response curves coming from a physical camera, together
with a matrix to map this space into \gls{CIE} \gls{XYZ} (or \sRGBl). Examples
of such data are given in the appendix, \cref{sec:cameradata}.


\subsection{Lights}

\begin{inconstruction}

Integrals to store:
\begin{itemize}
\item $\|D\|$ angular norm
\item $\|T_*\|$ store tint with texture
\item precomputed integrals for common illuminants
\end{itemize}

\paragraph{Angular norm}
angular distributions should be normalized to integrate to 1.

\paragraph{Texture norms}
\begin{displaymath}
|T_\star| = \int_A T_\star(x) \d x
\end{displaymath}
where A is the area of the light. The integral is actually an average of the
pixel values scaled with the light area. we can precompute the average of the
per channel pixel values and store them in the header of the file.

\paragraph{Reduced quantities}

Precompute
\begin{align*}
|\phi_\star| = \int_\lambda \star(\lambda) E(\lambda) \bar y(\lambda) \d\lambda\\
|\phi| = \int_\lambda E(\lambda) \bar y(\lambda) \d\lambda\\
\end{align*}

for the most common radiators. (black body, illuminant $D_{65}$, ..)

\end{inconstruction}


\section{Color space conversion code}

\subsubsection{Color spaces}

Quick summary of useful color space conversions

\paragraph{xyY to XYZ}

If $Y = 0$ the result is $(0,0,0)$, otherwise

\begin{align*}
X &= Y \frac{x}{y} \\
Y &= Y \\
Z &= Y \frac{1-x-y}{y} \\
\end{align*}


\begin{pythoncode}
# Computes XYZ coordinates from the given xy coordinates
# and Y value, assumes xy to be not imaginary
def XYZ(xy, Y):
    x, y = xy
    X = Y * x / y
    Z = Y * (1 - x - y) / y
    return (X, Y, Z)
\end{pythoncode}


\paragraph{XYZ to xy}

\begin{align*}
x &= \frac {X}{X+Y+Z} \\
y &= \frac {Y}{X+Y+Z}
\end{align*}


\begin{pythoncode}
# Computes xy coordinates from the given XYZ coordinates
def xy(XYZ):
    X, Y, Z = XYZ
    x = X / (X+Y+Z)
    y = Y / (X+Y+Z)
    return (x,y)
\end{pythoncode}


\paragraph{XYZ to sRGB/Rec. 709 primaries, linear}

\begin{displaymath}
C_{Rec709} = M_{Rec709} C_{XYZ} = \left[\begin{array}{r@{.}lr@{.}lr@{.}l}
 3&2404542 & -1&5371385 & -0&4985314 \\
-0&9692660 &  1&8760108 &  0&0415560 \\
 0&0556434 & -0&2040259 &  1&0572252 \\
\end{array}\right] C_{XYZ}
\end{displaymath}

\ifomit
\paragraph{XYZ to LMS/CAT02}

\begin{figure}
\centering
% SPDX-License-Identifier: Apache-2.0
% Copyright (c) Contributors to the PhysLight Project.
%% Creator: Matplotlib, PGF backend
%%
%% To include the figure in your LaTeX document, write
%%   \input{<filename>.pgf}
%%
%% Make sure the required packages are loaded in your preamble
%%   \usepackage{pgf}
%%
%% Figures using additional raster images can only be included by \input if
%% they are in the same directory as the main LaTeX file. For loading figures
%% from other directories you can use the `import` package
%%   \usepackage{import}
%% and then include the figures with
%%   \import{<path to file>}{<filename>.pgf}
%%
%% Matplotlib used the following preamble
%%   \usepackage{fontspec}
%%   \setmainfont{Bitstream Vera Serif}
%%   \setsansfont{Bitstream Vera Sans}
%%   \setmonofont{Bitstream Vera Sans Mono}
%%
\begingroup%
\makeatletter%
\begin{pgfpicture}%
\pgfpathrectangle{\pgfpointorigin}{\pgfqpoint{6.000000in}{2.000000in}}%
\pgfusepath{use as bounding box}%
\begin{pgfscope}%
\pgfsetrectcap%
\pgfsetroundjoin%
\definecolor{currentfill}{rgb}{1.000000,1.000000,1.000000}%
\pgfsetfillcolor{currentfill}%
\pgfsetlinewidth{0.000000pt}%
\definecolor{currentstroke}{rgb}{1.000000,1.000000,1.000000}%
\pgfsetstrokecolor{currentstroke}%
\pgfsetdash{}{0pt}%
\pgfpathmoveto{\pgfqpoint{0.000000in}{0.000000in}}%
\pgfpathlineto{\pgfqpoint{6.000000in}{0.000000in}}%
\pgfpathlineto{\pgfqpoint{6.000000in}{2.000000in}}%
\pgfpathlineto{\pgfqpoint{0.000000in}{2.000000in}}%
\pgfpathclose%
\pgfusepath{fill}%
\end{pgfscope}%
\begin{pgfscope}%
\pgfsetrectcap%
\pgfsetroundjoin%
\definecolor{currentfill}{rgb}{1.000000,1.000000,1.000000}%
\pgfsetfillcolor{currentfill}%
\pgfsetlinewidth{0.000000pt}%
\definecolor{currentstroke}{rgb}{0.000000,0.000000,0.000000}%
\pgfsetstrokecolor{currentstroke}%
\pgfsetdash{}{0pt}%
\pgfpathmoveto{\pgfqpoint{0.750000in}{0.200000in}}%
\pgfpathlineto{\pgfqpoint{5.400000in}{0.200000in}}%
\pgfpathlineto{\pgfqpoint{5.400000in}{1.800000in}}%
\pgfpathlineto{\pgfqpoint{0.750000in}{1.800000in}}%
\pgfpathclose%
\pgfusepath{fill}%
\end{pgfscope}%
\begin{pgfscope}%
\pgfpathrectangle{\pgfqpoint{0.750000in}{0.200000in}}{\pgfqpoint{4.650000in}{1.600000in}} %
\pgfusepath{clip}%
\pgfsetrectcap%
\pgfsetroundjoin%
\pgfsetlinewidth{1.003750pt}%
\definecolor{currentstroke}{rgb}{0.930925,0.928333,0.000000}%
\pgfsetstrokecolor{currentstroke}%
\pgfsetdash{}{0pt}%
\pgfpathmoveto{\pgfqpoint{0.750000in}{0.200000in}}%
\pgfpathlineto{\pgfqpoint{0.799468in}{0.200000in}}%
\pgfpathlineto{\pgfqpoint{0.848936in}{0.200000in}}%
\pgfpathlineto{\pgfqpoint{0.898404in}{0.200000in}}%
\pgfpathlineto{\pgfqpoint{0.947872in}{0.200000in}}%
\pgfpathlineto{\pgfqpoint{0.997340in}{0.200000in}}%
\pgfpathlineto{\pgfqpoint{1.046809in}{0.200664in}}%
\pgfpathlineto{\pgfqpoint{1.096277in}{0.201683in}}%
\pgfpathlineto{\pgfqpoint{1.145745in}{0.203853in}}%
\pgfpathlineto{\pgfqpoint{1.195213in}{0.207733in}}%
\pgfpathlineto{\pgfqpoint{1.244681in}{0.213954in}}%
\pgfpathlineto{\pgfqpoint{1.294149in}{0.221414in}}%
\pgfpathlineto{\pgfqpoint{1.343617in}{0.229517in}}%
\pgfpathlineto{\pgfqpoint{1.393085in}{0.236691in}}%
\pgfpathlineto{\pgfqpoint{1.442553in}{0.245100in}}%
\pgfpathlineto{\pgfqpoint{1.492021in}{0.254569in}}%
\pgfpathlineto{\pgfqpoint{1.541489in}{0.264410in}}%
\pgfpathlineto{\pgfqpoint{1.590957in}{0.271901in}}%
\pgfpathlineto{\pgfqpoint{1.640426in}{0.279782in}}%
\pgfpathlineto{\pgfqpoint{1.689894in}{0.288547in}}%
\pgfpathlineto{\pgfqpoint{1.739362in}{0.303546in}}%
\pgfpathlineto{\pgfqpoint{1.788830in}{0.329103in}}%
\pgfpathlineto{\pgfqpoint{1.838298in}{0.359161in}}%
\pgfpathlineto{\pgfqpoint{1.887766in}{0.390083in}}%
\pgfpathlineto{\pgfqpoint{1.937234in}{0.424232in}}%
\pgfpathlineto{\pgfqpoint{1.986702in}{0.462323in}}%
\pgfpathlineto{\pgfqpoint{2.036170in}{0.506490in}}%
\pgfpathlineto{\pgfqpoint{2.085638in}{0.572682in}}%
\pgfpathlineto{\pgfqpoint{2.135106in}{0.662334in}}%
\pgfpathlineto{\pgfqpoint{2.184574in}{0.775546in}}%
\pgfpathlineto{\pgfqpoint{2.234043in}{0.909893in}}%
\pgfpathlineto{\pgfqpoint{2.283511in}{1.058390in}}%
\pgfpathlineto{\pgfqpoint{2.332979in}{1.205698in}}%
\pgfpathlineto{\pgfqpoint{2.382447in}{1.327552in}}%
\pgfpathlineto{\pgfqpoint{2.431915in}{1.433008in}}%
\pgfpathlineto{\pgfqpoint{2.481383in}{1.521138in}}%
\pgfpathlineto{\pgfqpoint{2.530851in}{1.609618in}}%
\pgfpathlineto{\pgfqpoint{2.580319in}{1.670507in}}%
\pgfpathlineto{\pgfqpoint{2.629787in}{1.704317in}}%
\pgfpathlineto{\pgfqpoint{2.679255in}{1.745173in}}%
\pgfpathlineto{\pgfqpoint{2.728723in}{1.770312in}}%
\pgfpathlineto{\pgfqpoint{2.778191in}{1.791178in}}%
\pgfpathlineto{\pgfqpoint{2.827660in}{1.799989in}}%
\pgfpathlineto{\pgfqpoint{2.877128in}{1.787696in}}%
\pgfpathlineto{\pgfqpoint{2.926596in}{1.751086in}}%
\pgfpathlineto{\pgfqpoint{2.976064in}{1.728963in}}%
\pgfpathlineto{\pgfqpoint{3.025532in}{1.684277in}}%
\pgfpathlineto{\pgfqpoint{3.075000in}{1.617550in}}%
\pgfpathlineto{\pgfqpoint{3.124468in}{1.534371in}}%
\pgfpathlineto{\pgfqpoint{3.173936in}{1.440165in}}%
\pgfpathlineto{\pgfqpoint{3.223404in}{1.329141in}}%
\pgfpathlineto{\pgfqpoint{3.272872in}{1.209237in}}%
\pgfpathlineto{\pgfqpoint{3.322340in}{1.086758in}}%
\pgfpathlineto{\pgfqpoint{3.371809in}{0.967906in}}%
\pgfpathlineto{\pgfqpoint{3.421277in}{0.841138in}}%
\pgfpathlineto{\pgfqpoint{3.470745in}{0.724582in}}%
\pgfpathlineto{\pgfqpoint{3.520213in}{0.625254in}}%
\pgfpathlineto{\pgfqpoint{3.569681in}{0.541254in}}%
\pgfpathlineto{\pgfqpoint{3.619149in}{0.464226in}}%
\pgfpathlineto{\pgfqpoint{3.668617in}{0.399598in}}%
\pgfpathlineto{\pgfqpoint{3.718085in}{0.348814in}}%
\pgfpathlineto{\pgfqpoint{3.767553in}{0.309616in}}%
\pgfpathlineto{\pgfqpoint{3.817021in}{0.279786in}}%
\pgfpathlineto{\pgfqpoint{3.866489in}{0.257317in}}%
\pgfpathlineto{\pgfqpoint{3.915957in}{0.240606in}}%
\pgfpathlineto{\pgfqpoint{3.965426in}{0.228352in}}%
\pgfpathlineto{\pgfqpoint{4.014894in}{0.219472in}}%
\pgfpathlineto{\pgfqpoint{4.064362in}{0.213555in}}%
\pgfpathlineto{\pgfqpoint{4.113830in}{0.209436in}}%
\pgfpathlineto{\pgfqpoint{4.163298in}{0.206546in}}%
\pgfpathlineto{\pgfqpoint{4.212766in}{0.204487in}}%
\pgfpathlineto{\pgfqpoint{4.262234in}{0.203073in}}%
\pgfpathlineto{\pgfqpoint{4.311702in}{0.202123in}}%
\pgfpathlineto{\pgfqpoint{4.361170in}{0.201468in}}%
\pgfpathlineto{\pgfqpoint{4.410638in}{0.201023in}}%
\pgfpathlineto{\pgfqpoint{4.460106in}{0.200714in}}%
\pgfpathlineto{\pgfqpoint{4.509574in}{0.200497in}}%
\pgfpathlineto{\pgfqpoint{4.559043in}{0.200351in}}%
\pgfpathlineto{\pgfqpoint{4.608511in}{0.200247in}}%
\pgfpathlineto{\pgfqpoint{4.657979in}{0.200175in}}%
\pgfpathlineto{\pgfqpoint{4.707447in}{0.200125in}}%
\pgfpathlineto{\pgfqpoint{4.756915in}{0.200089in}}%
\pgfpathlineto{\pgfqpoint{4.806383in}{0.200064in}}%
\pgfpathlineto{\pgfqpoint{4.855851in}{0.200046in}}%
\pgfpathlineto{\pgfqpoint{4.905319in}{0.200033in}}%
\pgfpathlineto{\pgfqpoint{4.954787in}{0.200024in}}%
\pgfpathlineto{\pgfqpoint{5.004255in}{0.200018in}}%
\pgfpathlineto{\pgfqpoint{5.053723in}{0.200013in}}%
\pgfpathlineto{\pgfqpoint{5.103191in}{0.200009in}}%
\pgfpathlineto{\pgfqpoint{5.152660in}{0.200007in}}%
\pgfpathlineto{\pgfqpoint{5.202128in}{0.200005in}}%
\pgfpathlineto{\pgfqpoint{5.251596in}{0.200004in}}%
\pgfpathlineto{\pgfqpoint{5.301064in}{0.200003in}}%
\pgfpathlineto{\pgfqpoint{5.350532in}{0.200002in}}%
\pgfpathlineto{\pgfqpoint{5.400000in}{0.200002in}}%
\pgfusepath{stroke}%
\end{pgfscope}%
\begin{pgfscope}%
\pgfpathrectangle{\pgfqpoint{0.750000in}{0.200000in}}{\pgfqpoint{4.650000in}{1.600000in}} %
\pgfusepath{clip}%
\pgfsetbuttcap%
\pgfsetroundjoin%
\definecolor{currentfill}{rgb}{0.930925,0.928333,0.000000}%
\pgfsetfillcolor{currentfill}%
\pgfsetlinewidth{0.501875pt}%
\definecolor{currentstroke}{rgb}{0.930925,0.928333,0.000000}%
\pgfsetstrokecolor{currentstroke}%
\pgfsetdash{}{0pt}%
\pgfsys@defobject{currentmarker}{\pgfqpoint{-0.034722in}{-0.034722in}}{\pgfqpoint{0.034722in}{0.034722in}}{%
\pgfpathmoveto{\pgfqpoint{-0.034722in}{-0.034722in}}%
\pgfpathlineto{\pgfqpoint{0.034722in}{0.034722in}}%
\pgfpathmoveto{\pgfqpoint{-0.034722in}{0.034722in}}%
\pgfpathlineto{\pgfqpoint{0.034722in}{-0.034722in}}%
\pgfusepath{stroke,fill}%
}%
\begin{pgfscope}%
\pgfsys@transformshift{0.750000in}{0.200000in}%
\pgfsys@useobject{currentmarker}{}%
\end{pgfscope}%
\begin{pgfscope}%
\pgfsys@transformshift{0.799468in}{0.200000in}%
\pgfsys@useobject{currentmarker}{}%
\end{pgfscope}%
\begin{pgfscope}%
\pgfsys@transformshift{0.848936in}{0.200000in}%
\pgfsys@useobject{currentmarker}{}%
\end{pgfscope}%
\begin{pgfscope}%
\pgfsys@transformshift{0.898404in}{0.200000in}%
\pgfsys@useobject{currentmarker}{}%
\end{pgfscope}%
\begin{pgfscope}%
\pgfsys@transformshift{0.947872in}{0.200000in}%
\pgfsys@useobject{currentmarker}{}%
\end{pgfscope}%
\begin{pgfscope}%
\pgfsys@transformshift{0.997340in}{0.200000in}%
\pgfsys@useobject{currentmarker}{}%
\end{pgfscope}%
\begin{pgfscope}%
\pgfsys@transformshift{1.046809in}{0.200664in}%
\pgfsys@useobject{currentmarker}{}%
\end{pgfscope}%
\begin{pgfscope}%
\pgfsys@transformshift{1.096277in}{0.201683in}%
\pgfsys@useobject{currentmarker}{}%
\end{pgfscope}%
\begin{pgfscope}%
\pgfsys@transformshift{1.145745in}{0.203853in}%
\pgfsys@useobject{currentmarker}{}%
\end{pgfscope}%
\begin{pgfscope}%
\pgfsys@transformshift{1.195213in}{0.207733in}%
\pgfsys@useobject{currentmarker}{}%
\end{pgfscope}%
\begin{pgfscope}%
\pgfsys@transformshift{1.244681in}{0.213954in}%
\pgfsys@useobject{currentmarker}{}%
\end{pgfscope}%
\begin{pgfscope}%
\pgfsys@transformshift{1.294149in}{0.221414in}%
\pgfsys@useobject{currentmarker}{}%
\end{pgfscope}%
\begin{pgfscope}%
\pgfsys@transformshift{1.343617in}{0.229517in}%
\pgfsys@useobject{currentmarker}{}%
\end{pgfscope}%
\begin{pgfscope}%
\pgfsys@transformshift{1.393085in}{0.236691in}%
\pgfsys@useobject{currentmarker}{}%
\end{pgfscope}%
\begin{pgfscope}%
\pgfsys@transformshift{1.442553in}{0.245100in}%
\pgfsys@useobject{currentmarker}{}%
\end{pgfscope}%
\begin{pgfscope}%
\pgfsys@transformshift{1.492021in}{0.254569in}%
\pgfsys@useobject{currentmarker}{}%
\end{pgfscope}%
\begin{pgfscope}%
\pgfsys@transformshift{1.541489in}{0.264410in}%
\pgfsys@useobject{currentmarker}{}%
\end{pgfscope}%
\begin{pgfscope}%
\pgfsys@transformshift{1.590957in}{0.271901in}%
\pgfsys@useobject{currentmarker}{}%
\end{pgfscope}%
\begin{pgfscope}%
\pgfsys@transformshift{1.640426in}{0.279782in}%
\pgfsys@useobject{currentmarker}{}%
\end{pgfscope}%
\begin{pgfscope}%
\pgfsys@transformshift{1.689894in}{0.288547in}%
\pgfsys@useobject{currentmarker}{}%
\end{pgfscope}%
\begin{pgfscope}%
\pgfsys@transformshift{1.739362in}{0.303546in}%
\pgfsys@useobject{currentmarker}{}%
\end{pgfscope}%
\begin{pgfscope}%
\pgfsys@transformshift{1.788830in}{0.329103in}%
\pgfsys@useobject{currentmarker}{}%
\end{pgfscope}%
\begin{pgfscope}%
\pgfsys@transformshift{1.838298in}{0.359161in}%
\pgfsys@useobject{currentmarker}{}%
\end{pgfscope}%
\begin{pgfscope}%
\pgfsys@transformshift{1.887766in}{0.390083in}%
\pgfsys@useobject{currentmarker}{}%
\end{pgfscope}%
\begin{pgfscope}%
\pgfsys@transformshift{1.937234in}{0.424232in}%
\pgfsys@useobject{currentmarker}{}%
\end{pgfscope}%
\begin{pgfscope}%
\pgfsys@transformshift{1.986702in}{0.462323in}%
\pgfsys@useobject{currentmarker}{}%
\end{pgfscope}%
\begin{pgfscope}%
\pgfsys@transformshift{2.036170in}{0.506490in}%
\pgfsys@useobject{currentmarker}{}%
\end{pgfscope}%
\begin{pgfscope}%
\pgfsys@transformshift{2.085638in}{0.572682in}%
\pgfsys@useobject{currentmarker}{}%
\end{pgfscope}%
\begin{pgfscope}%
\pgfsys@transformshift{2.135106in}{0.662334in}%
\pgfsys@useobject{currentmarker}{}%
\end{pgfscope}%
\begin{pgfscope}%
\pgfsys@transformshift{2.184574in}{0.775546in}%
\pgfsys@useobject{currentmarker}{}%
\end{pgfscope}%
\begin{pgfscope}%
\pgfsys@transformshift{2.234043in}{0.909893in}%
\pgfsys@useobject{currentmarker}{}%
\end{pgfscope}%
\begin{pgfscope}%
\pgfsys@transformshift{2.283511in}{1.058390in}%
\pgfsys@useobject{currentmarker}{}%
\end{pgfscope}%
\begin{pgfscope}%
\pgfsys@transformshift{2.332979in}{1.205698in}%
\pgfsys@useobject{currentmarker}{}%
\end{pgfscope}%
\begin{pgfscope}%
\pgfsys@transformshift{2.382447in}{1.327552in}%
\pgfsys@useobject{currentmarker}{}%
\end{pgfscope}%
\begin{pgfscope}%
\pgfsys@transformshift{2.431915in}{1.433008in}%
\pgfsys@useobject{currentmarker}{}%
\end{pgfscope}%
\begin{pgfscope}%
\pgfsys@transformshift{2.481383in}{1.521138in}%
\pgfsys@useobject{currentmarker}{}%
\end{pgfscope}%
\begin{pgfscope}%
\pgfsys@transformshift{2.530851in}{1.609618in}%
\pgfsys@useobject{currentmarker}{}%
\end{pgfscope}%
\begin{pgfscope}%
\pgfsys@transformshift{2.580319in}{1.670507in}%
\pgfsys@useobject{currentmarker}{}%
\end{pgfscope}%
\begin{pgfscope}%
\pgfsys@transformshift{2.629787in}{1.704317in}%
\pgfsys@useobject{currentmarker}{}%
\end{pgfscope}%
\begin{pgfscope}%
\pgfsys@transformshift{2.679255in}{1.745173in}%
\pgfsys@useobject{currentmarker}{}%
\end{pgfscope}%
\begin{pgfscope}%
\pgfsys@transformshift{2.728723in}{1.770312in}%
\pgfsys@useobject{currentmarker}{}%
\end{pgfscope}%
\begin{pgfscope}%
\pgfsys@transformshift{2.778191in}{1.791178in}%
\pgfsys@useobject{currentmarker}{}%
\end{pgfscope}%
\begin{pgfscope}%
\pgfsys@transformshift{2.827660in}{1.799989in}%
\pgfsys@useobject{currentmarker}{}%
\end{pgfscope}%
\begin{pgfscope}%
\pgfsys@transformshift{2.877128in}{1.787696in}%
\pgfsys@useobject{currentmarker}{}%
\end{pgfscope}%
\begin{pgfscope}%
\pgfsys@transformshift{2.926596in}{1.751086in}%
\pgfsys@useobject{currentmarker}{}%
\end{pgfscope}%
\begin{pgfscope}%
\pgfsys@transformshift{2.976064in}{1.728963in}%
\pgfsys@useobject{currentmarker}{}%
\end{pgfscope}%
\begin{pgfscope}%
\pgfsys@transformshift{3.025532in}{1.684277in}%
\pgfsys@useobject{currentmarker}{}%
\end{pgfscope}%
\begin{pgfscope}%
\pgfsys@transformshift{3.075000in}{1.617550in}%
\pgfsys@useobject{currentmarker}{}%
\end{pgfscope}%
\begin{pgfscope}%
\pgfsys@transformshift{3.124468in}{1.534371in}%
\pgfsys@useobject{currentmarker}{}%
\end{pgfscope}%
\begin{pgfscope}%
\pgfsys@transformshift{3.173936in}{1.440165in}%
\pgfsys@useobject{currentmarker}{}%
\end{pgfscope}%
\begin{pgfscope}%
\pgfsys@transformshift{3.223404in}{1.329141in}%
\pgfsys@useobject{currentmarker}{}%
\end{pgfscope}%
\begin{pgfscope}%
\pgfsys@transformshift{3.272872in}{1.209237in}%
\pgfsys@useobject{currentmarker}{}%
\end{pgfscope}%
\begin{pgfscope}%
\pgfsys@transformshift{3.322340in}{1.086758in}%
\pgfsys@useobject{currentmarker}{}%
\end{pgfscope}%
\begin{pgfscope}%
\pgfsys@transformshift{3.371809in}{0.967906in}%
\pgfsys@useobject{currentmarker}{}%
\end{pgfscope}%
\begin{pgfscope}%
\pgfsys@transformshift{3.421277in}{0.841138in}%
\pgfsys@useobject{currentmarker}{}%
\end{pgfscope}%
\begin{pgfscope}%
\pgfsys@transformshift{3.470745in}{0.724582in}%
\pgfsys@useobject{currentmarker}{}%
\end{pgfscope}%
\begin{pgfscope}%
\pgfsys@transformshift{3.520213in}{0.625254in}%
\pgfsys@useobject{currentmarker}{}%
\end{pgfscope}%
\begin{pgfscope}%
\pgfsys@transformshift{3.569681in}{0.541254in}%
\pgfsys@useobject{currentmarker}{}%
\end{pgfscope}%
\begin{pgfscope}%
\pgfsys@transformshift{3.619149in}{0.464226in}%
\pgfsys@useobject{currentmarker}{}%
\end{pgfscope}%
\begin{pgfscope}%
\pgfsys@transformshift{3.668617in}{0.399598in}%
\pgfsys@useobject{currentmarker}{}%
\end{pgfscope}%
\begin{pgfscope}%
\pgfsys@transformshift{3.718085in}{0.348814in}%
\pgfsys@useobject{currentmarker}{}%
\end{pgfscope}%
\begin{pgfscope}%
\pgfsys@transformshift{3.767553in}{0.309616in}%
\pgfsys@useobject{currentmarker}{}%
\end{pgfscope}%
\begin{pgfscope}%
\pgfsys@transformshift{3.817021in}{0.279786in}%
\pgfsys@useobject{currentmarker}{}%
\end{pgfscope}%
\begin{pgfscope}%
\pgfsys@transformshift{3.866489in}{0.257317in}%
\pgfsys@useobject{currentmarker}{}%
\end{pgfscope}%
\begin{pgfscope}%
\pgfsys@transformshift{3.915957in}{0.240606in}%
\pgfsys@useobject{currentmarker}{}%
\end{pgfscope}%
\begin{pgfscope}%
\pgfsys@transformshift{3.965426in}{0.228352in}%
\pgfsys@useobject{currentmarker}{}%
\end{pgfscope}%
\begin{pgfscope}%
\pgfsys@transformshift{4.014894in}{0.219472in}%
\pgfsys@useobject{currentmarker}{}%
\end{pgfscope}%
\begin{pgfscope}%
\pgfsys@transformshift{4.064362in}{0.213555in}%
\pgfsys@useobject{currentmarker}{}%
\end{pgfscope}%
\begin{pgfscope}%
\pgfsys@transformshift{4.113830in}{0.209436in}%
\pgfsys@useobject{currentmarker}{}%
\end{pgfscope}%
\begin{pgfscope}%
\pgfsys@transformshift{4.163298in}{0.206546in}%
\pgfsys@useobject{currentmarker}{}%
\end{pgfscope}%
\begin{pgfscope}%
\pgfsys@transformshift{4.212766in}{0.204487in}%
\pgfsys@useobject{currentmarker}{}%
\end{pgfscope}%
\begin{pgfscope}%
\pgfsys@transformshift{4.262234in}{0.203073in}%
\pgfsys@useobject{currentmarker}{}%
\end{pgfscope}%
\begin{pgfscope}%
\pgfsys@transformshift{4.311702in}{0.202123in}%
\pgfsys@useobject{currentmarker}{}%
\end{pgfscope}%
\begin{pgfscope}%
\pgfsys@transformshift{4.361170in}{0.201468in}%
\pgfsys@useobject{currentmarker}{}%
\end{pgfscope}%
\begin{pgfscope}%
\pgfsys@transformshift{4.410638in}{0.201023in}%
\pgfsys@useobject{currentmarker}{}%
\end{pgfscope}%
\begin{pgfscope}%
\pgfsys@transformshift{4.460106in}{0.200714in}%
\pgfsys@useobject{currentmarker}{}%
\end{pgfscope}%
\begin{pgfscope}%
\pgfsys@transformshift{4.509574in}{0.200497in}%
\pgfsys@useobject{currentmarker}{}%
\end{pgfscope}%
\begin{pgfscope}%
\pgfsys@transformshift{4.559043in}{0.200351in}%
\pgfsys@useobject{currentmarker}{}%
\end{pgfscope}%
\begin{pgfscope}%
\pgfsys@transformshift{4.608511in}{0.200247in}%
\pgfsys@useobject{currentmarker}{}%
\end{pgfscope}%
\begin{pgfscope}%
\pgfsys@transformshift{4.657979in}{0.200175in}%
\pgfsys@useobject{currentmarker}{}%
\end{pgfscope}%
\begin{pgfscope}%
\pgfsys@transformshift{4.707447in}{0.200125in}%
\pgfsys@useobject{currentmarker}{}%
\end{pgfscope}%
\begin{pgfscope}%
\pgfsys@transformshift{4.756915in}{0.200089in}%
\pgfsys@useobject{currentmarker}{}%
\end{pgfscope}%
\begin{pgfscope}%
\pgfsys@transformshift{4.806383in}{0.200064in}%
\pgfsys@useobject{currentmarker}{}%
\end{pgfscope}%
\begin{pgfscope}%
\pgfsys@transformshift{4.855851in}{0.200046in}%
\pgfsys@useobject{currentmarker}{}%
\end{pgfscope}%
\begin{pgfscope}%
\pgfsys@transformshift{4.905319in}{0.200033in}%
\pgfsys@useobject{currentmarker}{}%
\end{pgfscope}%
\begin{pgfscope}%
\pgfsys@transformshift{4.954787in}{0.200024in}%
\pgfsys@useobject{currentmarker}{}%
\end{pgfscope}%
\begin{pgfscope}%
\pgfsys@transformshift{5.004255in}{0.200018in}%
\pgfsys@useobject{currentmarker}{}%
\end{pgfscope}%
\begin{pgfscope}%
\pgfsys@transformshift{5.053723in}{0.200013in}%
\pgfsys@useobject{currentmarker}{}%
\end{pgfscope}%
\begin{pgfscope}%
\pgfsys@transformshift{5.103191in}{0.200009in}%
\pgfsys@useobject{currentmarker}{}%
\end{pgfscope}%
\begin{pgfscope}%
\pgfsys@transformshift{5.152660in}{0.200007in}%
\pgfsys@useobject{currentmarker}{}%
\end{pgfscope}%
\begin{pgfscope}%
\pgfsys@transformshift{5.202128in}{0.200005in}%
\pgfsys@useobject{currentmarker}{}%
\end{pgfscope}%
\begin{pgfscope}%
\pgfsys@transformshift{5.251596in}{0.200004in}%
\pgfsys@useobject{currentmarker}{}%
\end{pgfscope}%
\begin{pgfscope}%
\pgfsys@transformshift{5.301064in}{0.200003in}%
\pgfsys@useobject{currentmarker}{}%
\end{pgfscope}%
\begin{pgfscope}%
\pgfsys@transformshift{5.350532in}{0.200002in}%
\pgfsys@useobject{currentmarker}{}%
\end{pgfscope}%
\begin{pgfscope}%
\pgfsys@transformshift{5.400000in}{0.200002in}%
\pgfsys@useobject{currentmarker}{}%
\end{pgfscope}%
\end{pgfscope}%
\begin{pgfscope}%
\pgfpathrectangle{\pgfqpoint{0.750000in}{0.200000in}}{\pgfqpoint{4.650000in}{1.600000in}} %
\pgfusepath{clip}%
\pgfsetrectcap%
\pgfsetroundjoin%
\pgfsetlinewidth{1.003750pt}%
\definecolor{currentstroke}{rgb}{0.462725,0.930925,0.199283}%
\pgfsetstrokecolor{currentstroke}%
\pgfsetdash{}{0pt}%
\pgfpathmoveto{\pgfqpoint{0.750000in}{0.200000in}}%
\pgfpathlineto{\pgfqpoint{0.799468in}{0.200000in}}%
\pgfpathlineto{\pgfqpoint{0.848936in}{0.200000in}}%
\pgfpathlineto{\pgfqpoint{0.898404in}{0.200000in}}%
\pgfpathlineto{\pgfqpoint{0.947872in}{0.200000in}}%
\pgfpathlineto{\pgfqpoint{0.997340in}{0.200000in}}%
\pgfpathlineto{\pgfqpoint{1.046809in}{0.200589in}}%
\pgfpathlineto{\pgfqpoint{1.096277in}{0.201534in}}%
\pgfpathlineto{\pgfqpoint{1.145745in}{0.203632in}}%
\pgfpathlineto{\pgfqpoint{1.195213in}{0.207520in}}%
\pgfpathlineto{\pgfqpoint{1.244681in}{0.214070in}}%
\pgfpathlineto{\pgfqpoint{1.294149in}{0.223244in}}%
\pgfpathlineto{\pgfqpoint{1.343617in}{0.234664in}}%
\pgfpathlineto{\pgfqpoint{1.393085in}{0.247314in}}%
\pgfpathlineto{\pgfqpoint{1.442553in}{0.263131in}}%
\pgfpathlineto{\pgfqpoint{1.492021in}{0.282912in}}%
\pgfpathlineto{\pgfqpoint{1.541489in}{0.303645in}}%
\pgfpathlineto{\pgfqpoint{1.590957in}{0.321410in}}%
\pgfpathlineto{\pgfqpoint{1.640426in}{0.339284in}}%
\pgfpathlineto{\pgfqpoint{1.689894in}{0.357109in}}%
\pgfpathlineto{\pgfqpoint{1.739362in}{0.386035in}}%
\pgfpathlineto{\pgfqpoint{1.788830in}{0.431266in}}%
\pgfpathlineto{\pgfqpoint{1.838298in}{0.481429in}}%
\pgfpathlineto{\pgfqpoint{1.887766in}{0.528637in}}%
\pgfpathlineto{\pgfqpoint{1.937234in}{0.577206in}}%
\pgfpathlineto{\pgfqpoint{1.986702in}{0.628901in}}%
\pgfpathlineto{\pgfqpoint{2.036170in}{0.685808in}}%
\pgfpathlineto{\pgfqpoint{2.085638in}{0.771298in}}%
\pgfpathlineto{\pgfqpoint{2.135106in}{0.884422in}}%
\pgfpathlineto{\pgfqpoint{2.184574in}{1.024939in}}%
\pgfpathlineto{\pgfqpoint{2.234043in}{1.184832in}}%
\pgfpathlineto{\pgfqpoint{2.283511in}{1.350646in}}%
\pgfpathlineto{\pgfqpoint{2.332979in}{1.506576in}}%
\pgfpathlineto{\pgfqpoint{2.382447in}{1.616880in}}%
\pgfpathlineto{\pgfqpoint{2.431915in}{1.697099in}}%
\pgfpathlineto{\pgfqpoint{2.481383in}{1.750173in}}%
\pgfpathlineto{\pgfqpoint{2.530851in}{1.792347in}}%
\pgfpathlineto{\pgfqpoint{2.580319in}{1.795509in}}%
\pgfpathlineto{\pgfqpoint{2.629787in}{1.763509in}}%
\pgfpathlineto{\pgfqpoint{2.679255in}{1.730533in}}%
\pgfpathlineto{\pgfqpoint{2.728723in}{1.668400in}}%
\pgfpathlineto{\pgfqpoint{2.778191in}{1.597128in}}%
\pgfpathlineto{\pgfqpoint{2.827660in}{1.501614in}}%
\pgfpathlineto{\pgfqpoint{2.877128in}{1.384466in}}%
\pgfpathlineto{\pgfqpoint{2.926596in}{1.245238in}}%
\pgfpathlineto{\pgfqpoint{2.976064in}{1.116155in}}%
\pgfpathlineto{\pgfqpoint{3.025532in}{0.988158in}}%
\pgfpathlineto{\pgfqpoint{3.075000in}{0.857994in}}%
\pgfpathlineto{\pgfqpoint{3.124468in}{0.735086in}}%
\pgfpathlineto{\pgfqpoint{3.173936in}{0.623795in}}%
\pgfpathlineto{\pgfqpoint{3.223404in}{0.528437in}}%
\pgfpathlineto{\pgfqpoint{3.272872in}{0.449989in}}%
\pgfpathlineto{\pgfqpoint{3.322340in}{0.386626in}}%
\pgfpathlineto{\pgfqpoint{3.371809in}{0.336940in}}%
\pgfpathlineto{\pgfqpoint{3.421277in}{0.299379in}}%
\pgfpathlineto{\pgfqpoint{3.470745in}{0.271181in}}%
\pgfpathlineto{\pgfqpoint{3.520213in}{0.250285in}}%
\pgfpathlineto{\pgfqpoint{3.569681in}{0.234886in}}%
\pgfpathlineto{\pgfqpoint{3.619149in}{0.224717in}}%
\pgfpathlineto{\pgfqpoint{3.668617in}{0.217139in}}%
\pgfpathlineto{\pgfqpoint{3.718085in}{0.211684in}}%
\pgfpathlineto{\pgfqpoint{3.767553in}{0.207955in}}%
\pgfpathlineto{\pgfqpoint{3.817021in}{0.205499in}}%
\pgfpathlineto{\pgfqpoint{3.866489in}{0.203802in}}%
\pgfpathlineto{\pgfqpoint{3.915957in}{0.202620in}}%
\pgfpathlineto{\pgfqpoint{3.965426in}{0.201794in}}%
\pgfpathlineto{\pgfqpoint{4.014894in}{0.201218in}}%
\pgfpathlineto{\pgfqpoint{4.064362in}{0.200841in}}%
\pgfpathlineto{\pgfqpoint{4.113830in}{0.200585in}}%
\pgfpathlineto{\pgfqpoint{4.163298in}{0.200405in}}%
\pgfpathlineto{\pgfqpoint{4.212766in}{0.200279in}}%
\pgfpathlineto{\pgfqpoint{4.262234in}{0.200193in}}%
\pgfpathlineto{\pgfqpoint{4.311702in}{0.200135in}}%
\pgfpathlineto{\pgfqpoint{4.361170in}{0.200094in}}%
\pgfpathlineto{\pgfqpoint{4.410638in}{0.200067in}}%
\pgfpathlineto{\pgfqpoint{4.460106in}{0.200047in}}%
\pgfpathlineto{\pgfqpoint{4.509574in}{0.200033in}}%
\pgfpathlineto{\pgfqpoint{4.559043in}{0.200024in}}%
\pgfpathlineto{\pgfqpoint{4.608511in}{0.200017in}}%
\pgfpathlineto{\pgfqpoint{4.657979in}{0.200013in}}%
\pgfpathlineto{\pgfqpoint{4.707447in}{0.200009in}}%
\pgfpathlineto{\pgfqpoint{4.756915in}{0.200007in}}%
\pgfpathlineto{\pgfqpoint{4.806383in}{0.200005in}}%
\pgfpathlineto{\pgfqpoint{4.855851in}{0.200004in}}%
\pgfpathlineto{\pgfqpoint{4.905319in}{0.200003in}}%
\pgfpathlineto{\pgfqpoint{4.954787in}{0.200002in}}%
\pgfpathlineto{\pgfqpoint{5.004255in}{0.200001in}}%
\pgfpathlineto{\pgfqpoint{5.053723in}{0.200001in}}%
\pgfpathlineto{\pgfqpoint{5.103191in}{0.200001in}}%
\pgfpathlineto{\pgfqpoint{5.152660in}{0.200001in}}%
\pgfpathlineto{\pgfqpoint{5.202128in}{0.200000in}}%
\pgfpathlineto{\pgfqpoint{5.251596in}{0.200000in}}%
\pgfpathlineto{\pgfqpoint{5.301064in}{0.200000in}}%
\pgfpathlineto{\pgfqpoint{5.350532in}{0.200000in}}%
\pgfpathlineto{\pgfqpoint{5.400000in}{0.200000in}}%
\pgfusepath{stroke}%
\end{pgfscope}%
\begin{pgfscope}%
\pgfpathrectangle{\pgfqpoint{0.750000in}{0.200000in}}{\pgfqpoint{4.650000in}{1.600000in}} %
\pgfusepath{clip}%
\pgfsetbuttcap%
\pgfsetroundjoin%
\definecolor{currentfill}{rgb}{0.462725,0.930925,0.199283}%
\pgfsetfillcolor{currentfill}%
\pgfsetlinewidth{0.501875pt}%
\definecolor{currentstroke}{rgb}{0.462725,0.930925,0.199283}%
\pgfsetstrokecolor{currentstroke}%
\pgfsetdash{}{0pt}%
\pgfsys@defobject{currentmarker}{\pgfqpoint{-0.034722in}{-0.034722in}}{\pgfqpoint{0.034722in}{0.034722in}}{%
\pgfpathmoveto{\pgfqpoint{-0.034722in}{-0.034722in}}%
\pgfpathlineto{\pgfqpoint{0.034722in}{0.034722in}}%
\pgfpathmoveto{\pgfqpoint{-0.034722in}{0.034722in}}%
\pgfpathlineto{\pgfqpoint{0.034722in}{-0.034722in}}%
\pgfusepath{stroke,fill}%
}%
\begin{pgfscope}%
\pgfsys@transformshift{0.750000in}{0.200000in}%
\pgfsys@useobject{currentmarker}{}%
\end{pgfscope}%
\begin{pgfscope}%
\pgfsys@transformshift{0.799468in}{0.200000in}%
\pgfsys@useobject{currentmarker}{}%
\end{pgfscope}%
\begin{pgfscope}%
\pgfsys@transformshift{0.848936in}{0.200000in}%
\pgfsys@useobject{currentmarker}{}%
\end{pgfscope}%
\begin{pgfscope}%
\pgfsys@transformshift{0.898404in}{0.200000in}%
\pgfsys@useobject{currentmarker}{}%
\end{pgfscope}%
\begin{pgfscope}%
\pgfsys@transformshift{0.947872in}{0.200000in}%
\pgfsys@useobject{currentmarker}{}%
\end{pgfscope}%
\begin{pgfscope}%
\pgfsys@transformshift{0.997340in}{0.200000in}%
\pgfsys@useobject{currentmarker}{}%
\end{pgfscope}%
\begin{pgfscope}%
\pgfsys@transformshift{1.046809in}{0.200589in}%
\pgfsys@useobject{currentmarker}{}%
\end{pgfscope}%
\begin{pgfscope}%
\pgfsys@transformshift{1.096277in}{0.201534in}%
\pgfsys@useobject{currentmarker}{}%
\end{pgfscope}%
\begin{pgfscope}%
\pgfsys@transformshift{1.145745in}{0.203632in}%
\pgfsys@useobject{currentmarker}{}%
\end{pgfscope}%
\begin{pgfscope}%
\pgfsys@transformshift{1.195213in}{0.207520in}%
\pgfsys@useobject{currentmarker}{}%
\end{pgfscope}%
\begin{pgfscope}%
\pgfsys@transformshift{1.244681in}{0.214070in}%
\pgfsys@useobject{currentmarker}{}%
\end{pgfscope}%
\begin{pgfscope}%
\pgfsys@transformshift{1.294149in}{0.223244in}%
\pgfsys@useobject{currentmarker}{}%
\end{pgfscope}%
\begin{pgfscope}%
\pgfsys@transformshift{1.343617in}{0.234664in}%
\pgfsys@useobject{currentmarker}{}%
\end{pgfscope}%
\begin{pgfscope}%
\pgfsys@transformshift{1.393085in}{0.247314in}%
\pgfsys@useobject{currentmarker}{}%
\end{pgfscope}%
\begin{pgfscope}%
\pgfsys@transformshift{1.442553in}{0.263131in}%
\pgfsys@useobject{currentmarker}{}%
\end{pgfscope}%
\begin{pgfscope}%
\pgfsys@transformshift{1.492021in}{0.282912in}%
\pgfsys@useobject{currentmarker}{}%
\end{pgfscope}%
\begin{pgfscope}%
\pgfsys@transformshift{1.541489in}{0.303645in}%
\pgfsys@useobject{currentmarker}{}%
\end{pgfscope}%
\begin{pgfscope}%
\pgfsys@transformshift{1.590957in}{0.321410in}%
\pgfsys@useobject{currentmarker}{}%
\end{pgfscope}%
\begin{pgfscope}%
\pgfsys@transformshift{1.640426in}{0.339284in}%
\pgfsys@useobject{currentmarker}{}%
\end{pgfscope}%
\begin{pgfscope}%
\pgfsys@transformshift{1.689894in}{0.357109in}%
\pgfsys@useobject{currentmarker}{}%
\end{pgfscope}%
\begin{pgfscope}%
\pgfsys@transformshift{1.739362in}{0.386035in}%
\pgfsys@useobject{currentmarker}{}%
\end{pgfscope}%
\begin{pgfscope}%
\pgfsys@transformshift{1.788830in}{0.431266in}%
\pgfsys@useobject{currentmarker}{}%
\end{pgfscope}%
\begin{pgfscope}%
\pgfsys@transformshift{1.838298in}{0.481429in}%
\pgfsys@useobject{currentmarker}{}%
\end{pgfscope}%
\begin{pgfscope}%
\pgfsys@transformshift{1.887766in}{0.528637in}%
\pgfsys@useobject{currentmarker}{}%
\end{pgfscope}%
\begin{pgfscope}%
\pgfsys@transformshift{1.937234in}{0.577206in}%
\pgfsys@useobject{currentmarker}{}%
\end{pgfscope}%
\begin{pgfscope}%
\pgfsys@transformshift{1.986702in}{0.628901in}%
\pgfsys@useobject{currentmarker}{}%
\end{pgfscope}%
\begin{pgfscope}%
\pgfsys@transformshift{2.036170in}{0.685808in}%
\pgfsys@useobject{currentmarker}{}%
\end{pgfscope}%
\begin{pgfscope}%
\pgfsys@transformshift{2.085638in}{0.771298in}%
\pgfsys@useobject{currentmarker}{}%
\end{pgfscope}%
\begin{pgfscope}%
\pgfsys@transformshift{2.135106in}{0.884422in}%
\pgfsys@useobject{currentmarker}{}%
\end{pgfscope}%
\begin{pgfscope}%
\pgfsys@transformshift{2.184574in}{1.024939in}%
\pgfsys@useobject{currentmarker}{}%
\end{pgfscope}%
\begin{pgfscope}%
\pgfsys@transformshift{2.234043in}{1.184832in}%
\pgfsys@useobject{currentmarker}{}%
\end{pgfscope}%
\begin{pgfscope}%
\pgfsys@transformshift{2.283511in}{1.350646in}%
\pgfsys@useobject{currentmarker}{}%
\end{pgfscope}%
\begin{pgfscope}%
\pgfsys@transformshift{2.332979in}{1.506576in}%
\pgfsys@useobject{currentmarker}{}%
\end{pgfscope}%
\begin{pgfscope}%
\pgfsys@transformshift{2.382447in}{1.616880in}%
\pgfsys@useobject{currentmarker}{}%
\end{pgfscope}%
\begin{pgfscope}%
\pgfsys@transformshift{2.431915in}{1.697099in}%
\pgfsys@useobject{currentmarker}{}%
\end{pgfscope}%
\begin{pgfscope}%
\pgfsys@transformshift{2.481383in}{1.750173in}%
\pgfsys@useobject{currentmarker}{}%
\end{pgfscope}%
\begin{pgfscope}%
\pgfsys@transformshift{2.530851in}{1.792347in}%
\pgfsys@useobject{currentmarker}{}%
\end{pgfscope}%
\begin{pgfscope}%
\pgfsys@transformshift{2.580319in}{1.795509in}%
\pgfsys@useobject{currentmarker}{}%
\end{pgfscope}%
\begin{pgfscope}%
\pgfsys@transformshift{2.629787in}{1.763509in}%
\pgfsys@useobject{currentmarker}{}%
\end{pgfscope}%
\begin{pgfscope}%
\pgfsys@transformshift{2.679255in}{1.730533in}%
\pgfsys@useobject{currentmarker}{}%
\end{pgfscope}%
\begin{pgfscope}%
\pgfsys@transformshift{2.728723in}{1.668400in}%
\pgfsys@useobject{currentmarker}{}%
\end{pgfscope}%
\begin{pgfscope}%
\pgfsys@transformshift{2.778191in}{1.597128in}%
\pgfsys@useobject{currentmarker}{}%
\end{pgfscope}%
\begin{pgfscope}%
\pgfsys@transformshift{2.827660in}{1.501614in}%
\pgfsys@useobject{currentmarker}{}%
\end{pgfscope}%
\begin{pgfscope}%
\pgfsys@transformshift{2.877128in}{1.384466in}%
\pgfsys@useobject{currentmarker}{}%
\end{pgfscope}%
\begin{pgfscope}%
\pgfsys@transformshift{2.926596in}{1.245238in}%
\pgfsys@useobject{currentmarker}{}%
\end{pgfscope}%
\begin{pgfscope}%
\pgfsys@transformshift{2.976064in}{1.116155in}%
\pgfsys@useobject{currentmarker}{}%
\end{pgfscope}%
\begin{pgfscope}%
\pgfsys@transformshift{3.025532in}{0.988158in}%
\pgfsys@useobject{currentmarker}{}%
\end{pgfscope}%
\begin{pgfscope}%
\pgfsys@transformshift{3.075000in}{0.857994in}%
\pgfsys@useobject{currentmarker}{}%
\end{pgfscope}%
\begin{pgfscope}%
\pgfsys@transformshift{3.124468in}{0.735086in}%
\pgfsys@useobject{currentmarker}{}%
\end{pgfscope}%
\begin{pgfscope}%
\pgfsys@transformshift{3.173936in}{0.623795in}%
\pgfsys@useobject{currentmarker}{}%
\end{pgfscope}%
\begin{pgfscope}%
\pgfsys@transformshift{3.223404in}{0.528437in}%
\pgfsys@useobject{currentmarker}{}%
\end{pgfscope}%
\begin{pgfscope}%
\pgfsys@transformshift{3.272872in}{0.449989in}%
\pgfsys@useobject{currentmarker}{}%
\end{pgfscope}%
\begin{pgfscope}%
\pgfsys@transformshift{3.322340in}{0.386626in}%
\pgfsys@useobject{currentmarker}{}%
\end{pgfscope}%
\begin{pgfscope}%
\pgfsys@transformshift{3.371809in}{0.336940in}%
\pgfsys@useobject{currentmarker}{}%
\end{pgfscope}%
\begin{pgfscope}%
\pgfsys@transformshift{3.421277in}{0.299379in}%
\pgfsys@useobject{currentmarker}{}%
\end{pgfscope}%
\begin{pgfscope}%
\pgfsys@transformshift{3.470745in}{0.271181in}%
\pgfsys@useobject{currentmarker}{}%
\end{pgfscope}%
\begin{pgfscope}%
\pgfsys@transformshift{3.520213in}{0.250285in}%
\pgfsys@useobject{currentmarker}{}%
\end{pgfscope}%
\begin{pgfscope}%
\pgfsys@transformshift{3.569681in}{0.234886in}%
\pgfsys@useobject{currentmarker}{}%
\end{pgfscope}%
\begin{pgfscope}%
\pgfsys@transformshift{3.619149in}{0.224717in}%
\pgfsys@useobject{currentmarker}{}%
\end{pgfscope}%
\begin{pgfscope}%
\pgfsys@transformshift{3.668617in}{0.217139in}%
\pgfsys@useobject{currentmarker}{}%
\end{pgfscope}%
\begin{pgfscope}%
\pgfsys@transformshift{3.718085in}{0.211684in}%
\pgfsys@useobject{currentmarker}{}%
\end{pgfscope}%
\begin{pgfscope}%
\pgfsys@transformshift{3.767553in}{0.207955in}%
\pgfsys@useobject{currentmarker}{}%
\end{pgfscope}%
\begin{pgfscope}%
\pgfsys@transformshift{3.817021in}{0.205499in}%
\pgfsys@useobject{currentmarker}{}%
\end{pgfscope}%
\begin{pgfscope}%
\pgfsys@transformshift{3.866489in}{0.203802in}%
\pgfsys@useobject{currentmarker}{}%
\end{pgfscope}%
\begin{pgfscope}%
\pgfsys@transformshift{3.915957in}{0.202620in}%
\pgfsys@useobject{currentmarker}{}%
\end{pgfscope}%
\begin{pgfscope}%
\pgfsys@transformshift{3.965426in}{0.201794in}%
\pgfsys@useobject{currentmarker}{}%
\end{pgfscope}%
\begin{pgfscope}%
\pgfsys@transformshift{4.014894in}{0.201218in}%
\pgfsys@useobject{currentmarker}{}%
\end{pgfscope}%
\begin{pgfscope}%
\pgfsys@transformshift{4.064362in}{0.200841in}%
\pgfsys@useobject{currentmarker}{}%
\end{pgfscope}%
\begin{pgfscope}%
\pgfsys@transformshift{4.113830in}{0.200585in}%
\pgfsys@useobject{currentmarker}{}%
\end{pgfscope}%
\begin{pgfscope}%
\pgfsys@transformshift{4.163298in}{0.200405in}%
\pgfsys@useobject{currentmarker}{}%
\end{pgfscope}%
\begin{pgfscope}%
\pgfsys@transformshift{4.212766in}{0.200279in}%
\pgfsys@useobject{currentmarker}{}%
\end{pgfscope}%
\begin{pgfscope}%
\pgfsys@transformshift{4.262234in}{0.200193in}%
\pgfsys@useobject{currentmarker}{}%
\end{pgfscope}%
\begin{pgfscope}%
\pgfsys@transformshift{4.311702in}{0.200135in}%
\pgfsys@useobject{currentmarker}{}%
\end{pgfscope}%
\begin{pgfscope}%
\pgfsys@transformshift{4.361170in}{0.200094in}%
\pgfsys@useobject{currentmarker}{}%
\end{pgfscope}%
\begin{pgfscope}%
\pgfsys@transformshift{4.410638in}{0.200067in}%
\pgfsys@useobject{currentmarker}{}%
\end{pgfscope}%
\begin{pgfscope}%
\pgfsys@transformshift{4.460106in}{0.200047in}%
\pgfsys@useobject{currentmarker}{}%
\end{pgfscope}%
\begin{pgfscope}%
\pgfsys@transformshift{4.509574in}{0.200033in}%
\pgfsys@useobject{currentmarker}{}%
\end{pgfscope}%
\begin{pgfscope}%
\pgfsys@transformshift{4.559043in}{0.200024in}%
\pgfsys@useobject{currentmarker}{}%
\end{pgfscope}%
\begin{pgfscope}%
\pgfsys@transformshift{4.608511in}{0.200017in}%
\pgfsys@useobject{currentmarker}{}%
\end{pgfscope}%
\begin{pgfscope}%
\pgfsys@transformshift{4.657979in}{0.200013in}%
\pgfsys@useobject{currentmarker}{}%
\end{pgfscope}%
\begin{pgfscope}%
\pgfsys@transformshift{4.707447in}{0.200009in}%
\pgfsys@useobject{currentmarker}{}%
\end{pgfscope}%
\begin{pgfscope}%
\pgfsys@transformshift{4.756915in}{0.200007in}%
\pgfsys@useobject{currentmarker}{}%
\end{pgfscope}%
\begin{pgfscope}%
\pgfsys@transformshift{4.806383in}{0.200005in}%
\pgfsys@useobject{currentmarker}{}%
\end{pgfscope}%
\begin{pgfscope}%
\pgfsys@transformshift{4.855851in}{0.200004in}%
\pgfsys@useobject{currentmarker}{}%
\end{pgfscope}%
\begin{pgfscope}%
\pgfsys@transformshift{4.905319in}{0.200003in}%
\pgfsys@useobject{currentmarker}{}%
\end{pgfscope}%
\begin{pgfscope}%
\pgfsys@transformshift{4.954787in}{0.200002in}%
\pgfsys@useobject{currentmarker}{}%
\end{pgfscope}%
\begin{pgfscope}%
\pgfsys@transformshift{5.004255in}{0.200001in}%
\pgfsys@useobject{currentmarker}{}%
\end{pgfscope}%
\begin{pgfscope}%
\pgfsys@transformshift{5.053723in}{0.200001in}%
\pgfsys@useobject{currentmarker}{}%
\end{pgfscope}%
\begin{pgfscope}%
\pgfsys@transformshift{5.103191in}{0.200001in}%
\pgfsys@useobject{currentmarker}{}%
\end{pgfscope}%
\begin{pgfscope}%
\pgfsys@transformshift{5.152660in}{0.200001in}%
\pgfsys@useobject{currentmarker}{}%
\end{pgfscope}%
\begin{pgfscope}%
\pgfsys@transformshift{5.202128in}{0.200000in}%
\pgfsys@useobject{currentmarker}{}%
\end{pgfscope}%
\begin{pgfscope}%
\pgfsys@transformshift{5.251596in}{0.200000in}%
\pgfsys@useobject{currentmarker}{}%
\end{pgfscope}%
\begin{pgfscope}%
\pgfsys@transformshift{5.301064in}{0.200000in}%
\pgfsys@useobject{currentmarker}{}%
\end{pgfscope}%
\begin{pgfscope}%
\pgfsys@transformshift{5.350532in}{0.200000in}%
\pgfsys@useobject{currentmarker}{}%
\end{pgfscope}%
\begin{pgfscope}%
\pgfsys@transformshift{5.400000in}{0.200000in}%
\pgfsys@useobject{currentmarker}{}%
\end{pgfscope}%
\end{pgfscope}%
\begin{pgfscope}%
\pgfpathrectangle{\pgfqpoint{0.750000in}{0.200000in}}{\pgfqpoint{4.650000in}{1.600000in}} %
\pgfusepath{clip}%
\pgfsetrectcap%
\pgfsetroundjoin%
\pgfsetlinewidth{1.003750pt}%
\definecolor{currentstroke}{rgb}{0.123949,0.000000,0.930925}%
\pgfsetstrokecolor{currentstroke}%
\pgfsetdash{}{0pt}%
\pgfpathmoveto{\pgfqpoint{0.750000in}{0.200000in}}%
\pgfpathlineto{\pgfqpoint{0.799468in}{0.200000in}}%
\pgfpathlineto{\pgfqpoint{0.848936in}{0.200000in}}%
\pgfpathlineto{\pgfqpoint{0.898404in}{0.200000in}}%
\pgfpathlineto{\pgfqpoint{0.947872in}{0.200000in}}%
\pgfpathlineto{\pgfqpoint{0.997340in}{0.200000in}}%
\pgfpathlineto{\pgfqpoint{1.046809in}{0.215276in}}%
\pgfpathlineto{\pgfqpoint{1.096277in}{0.238120in}}%
\pgfpathlineto{\pgfqpoint{1.145745in}{0.290640in}}%
\pgfpathlineto{\pgfqpoint{1.195213in}{0.395922in}}%
\pgfpathlineto{\pgfqpoint{1.244681in}{0.572813in}}%
\pgfpathlineto{\pgfqpoint{1.294149in}{0.810181in}}%
\pgfpathlineto{\pgfqpoint{1.343617in}{1.069789in}}%
\pgfpathlineto{\pgfqpoint{1.393085in}{1.279158in}}%
\pgfpathlineto{\pgfqpoint{1.442553in}{1.484088in}}%
\pgfpathlineto{\pgfqpoint{1.492021in}{1.645717in}}%
\pgfpathlineto{\pgfqpoint{1.541489in}{1.785632in}}%
\pgfpathlineto{\pgfqpoint{1.590957in}{1.786424in}}%
\pgfpathlineto{\pgfqpoint{1.640426in}{1.728629in}}%
\pgfpathlineto{\pgfqpoint{1.689894in}{1.576384in}}%
\pgfpathlineto{\pgfqpoint{1.739362in}{1.458726in}}%
\pgfpathlineto{\pgfqpoint{1.788830in}{1.381229in}}%
\pgfpathlineto{\pgfqpoint{1.838298in}{1.234174in}}%
\pgfpathlineto{\pgfqpoint{1.887766in}{1.026258in}}%
\pgfpathlineto{\pgfqpoint{1.937234in}{0.824533in}}%
\pgfpathlineto{\pgfqpoint{1.986702in}{0.664515in}}%
\pgfpathlineto{\pgfqpoint{2.036170in}{0.538987in}}%
\pgfpathlineto{\pgfqpoint{2.085638in}{0.456842in}}%
\pgfpathlineto{\pgfqpoint{2.135106in}{0.396542in}}%
\pgfpathlineto{\pgfqpoint{2.184574in}{0.342234in}}%
\pgfpathlineto{\pgfqpoint{2.234043in}{0.297314in}}%
\pgfpathlineto{\pgfqpoint{2.283511in}{0.268500in}}%
\pgfpathlineto{\pgfqpoint{2.332979in}{0.246725in}}%
\pgfpathlineto{\pgfqpoint{2.382447in}{0.231026in}}%
\pgfpathlineto{\pgfqpoint{2.431915in}{0.220162in}}%
\pgfpathlineto{\pgfqpoint{2.481383in}{0.212951in}}%
\pgfpathlineto{\pgfqpoint{2.530851in}{0.208142in}}%
\pgfpathlineto{\pgfqpoint{2.580319in}{0.205070in}}%
\pgfpathlineto{\pgfqpoint{2.629787in}{0.203134in}}%
\pgfpathlineto{\pgfqpoint{2.679255in}{0.201924in}}%
\pgfpathlineto{\pgfqpoint{2.728723in}{0.201184in}}%
\pgfpathlineto{\pgfqpoint{2.778191in}{0.200730in}}%
\pgfpathlineto{\pgfqpoint{2.827660in}{0.200451in}}%
\pgfpathlineto{\pgfqpoint{2.877128in}{0.200280in}}%
\pgfpathlineto{\pgfqpoint{2.926596in}{0.200175in}}%
\pgfpathlineto{\pgfqpoint{2.976064in}{0.200110in}}%
\pgfpathlineto{\pgfqpoint{3.025532in}{0.200070in}}%
\pgfpathlineto{\pgfqpoint{3.075000in}{0.200045in}}%
\pgfpathlineto{\pgfqpoint{3.124468in}{0.200029in}}%
\pgfpathlineto{\pgfqpoint{3.173936in}{0.200019in}}%
\pgfpathlineto{\pgfqpoint{3.223404in}{0.200013in}}%
\pgfpathlineto{\pgfqpoint{3.272872in}{0.200009in}}%
\pgfpathlineto{\pgfqpoint{3.322340in}{0.200000in}}%
\pgfpathlineto{\pgfqpoint{3.371809in}{0.200000in}}%
\pgfpathlineto{\pgfqpoint{3.421277in}{0.200000in}}%
\pgfpathlineto{\pgfqpoint{3.470745in}{0.200000in}}%
\pgfpathlineto{\pgfqpoint{3.520213in}{0.200000in}}%
\pgfpathlineto{\pgfqpoint{3.569681in}{0.200000in}}%
\pgfpathlineto{\pgfqpoint{3.619149in}{0.200000in}}%
\pgfpathlineto{\pgfqpoint{3.668617in}{0.200000in}}%
\pgfpathlineto{\pgfqpoint{3.718085in}{0.200000in}}%
\pgfpathlineto{\pgfqpoint{3.767553in}{0.200000in}}%
\pgfpathlineto{\pgfqpoint{3.817021in}{0.200000in}}%
\pgfpathlineto{\pgfqpoint{3.866489in}{0.200000in}}%
\pgfpathlineto{\pgfqpoint{3.915957in}{0.200000in}}%
\pgfpathlineto{\pgfqpoint{3.965426in}{0.200000in}}%
\pgfpathlineto{\pgfqpoint{4.014894in}{0.200000in}}%
\pgfpathlineto{\pgfqpoint{4.064362in}{0.200000in}}%
\pgfpathlineto{\pgfqpoint{4.113830in}{0.200000in}}%
\pgfpathlineto{\pgfqpoint{4.163298in}{0.200000in}}%
\pgfpathlineto{\pgfqpoint{4.212766in}{0.200000in}}%
\pgfpathlineto{\pgfqpoint{4.262234in}{0.200000in}}%
\pgfpathlineto{\pgfqpoint{4.311702in}{0.200000in}}%
\pgfpathlineto{\pgfqpoint{4.361170in}{0.200000in}}%
\pgfpathlineto{\pgfqpoint{4.410638in}{0.200000in}}%
\pgfpathlineto{\pgfqpoint{4.460106in}{0.200000in}}%
\pgfpathlineto{\pgfqpoint{4.509574in}{0.200000in}}%
\pgfpathlineto{\pgfqpoint{4.559043in}{0.200000in}}%
\pgfpathlineto{\pgfqpoint{4.608511in}{0.200000in}}%
\pgfpathlineto{\pgfqpoint{4.657979in}{0.200000in}}%
\pgfpathlineto{\pgfqpoint{4.707447in}{0.200000in}}%
\pgfpathlineto{\pgfqpoint{4.756915in}{0.200000in}}%
\pgfpathlineto{\pgfqpoint{4.806383in}{0.200000in}}%
\pgfpathlineto{\pgfqpoint{4.855851in}{0.200000in}}%
\pgfpathlineto{\pgfqpoint{4.905319in}{0.200000in}}%
\pgfpathlineto{\pgfqpoint{4.954787in}{0.200000in}}%
\pgfpathlineto{\pgfqpoint{5.004255in}{0.200000in}}%
\pgfpathlineto{\pgfqpoint{5.053723in}{0.200000in}}%
\pgfpathlineto{\pgfqpoint{5.103191in}{0.200000in}}%
\pgfpathlineto{\pgfqpoint{5.152660in}{0.200000in}}%
\pgfpathlineto{\pgfqpoint{5.202128in}{0.200000in}}%
\pgfpathlineto{\pgfqpoint{5.251596in}{0.200000in}}%
\pgfpathlineto{\pgfqpoint{5.301064in}{0.200000in}}%
\pgfpathlineto{\pgfqpoint{5.350532in}{0.200000in}}%
\pgfpathlineto{\pgfqpoint{5.400000in}{0.200000in}}%
\pgfusepath{stroke}%
\end{pgfscope}%
\begin{pgfscope}%
\pgfpathrectangle{\pgfqpoint{0.750000in}{0.200000in}}{\pgfqpoint{4.650000in}{1.600000in}} %
\pgfusepath{clip}%
\pgfsetbuttcap%
\pgfsetroundjoin%
\definecolor{currentfill}{rgb}{0.123949,0.000000,0.930925}%
\pgfsetfillcolor{currentfill}%
\pgfsetlinewidth{0.501875pt}%
\definecolor{currentstroke}{rgb}{0.123949,0.000000,0.930925}%
\pgfsetstrokecolor{currentstroke}%
\pgfsetdash{}{0pt}%
\pgfsys@defobject{currentmarker}{\pgfqpoint{-0.034722in}{-0.034722in}}{\pgfqpoint{0.034722in}{0.034722in}}{%
\pgfpathmoveto{\pgfqpoint{-0.034722in}{-0.034722in}}%
\pgfpathlineto{\pgfqpoint{0.034722in}{0.034722in}}%
\pgfpathmoveto{\pgfqpoint{-0.034722in}{0.034722in}}%
\pgfpathlineto{\pgfqpoint{0.034722in}{-0.034722in}}%
\pgfusepath{stroke,fill}%
}%
\begin{pgfscope}%
\pgfsys@transformshift{0.750000in}{0.200000in}%
\pgfsys@useobject{currentmarker}{}%
\end{pgfscope}%
\begin{pgfscope}%
\pgfsys@transformshift{0.799468in}{0.200000in}%
\pgfsys@useobject{currentmarker}{}%
\end{pgfscope}%
\begin{pgfscope}%
\pgfsys@transformshift{0.848936in}{0.200000in}%
\pgfsys@useobject{currentmarker}{}%
\end{pgfscope}%
\begin{pgfscope}%
\pgfsys@transformshift{0.898404in}{0.200000in}%
\pgfsys@useobject{currentmarker}{}%
\end{pgfscope}%
\begin{pgfscope}%
\pgfsys@transformshift{0.947872in}{0.200000in}%
\pgfsys@useobject{currentmarker}{}%
\end{pgfscope}%
\begin{pgfscope}%
\pgfsys@transformshift{0.997340in}{0.200000in}%
\pgfsys@useobject{currentmarker}{}%
\end{pgfscope}%
\begin{pgfscope}%
\pgfsys@transformshift{1.046809in}{0.215276in}%
\pgfsys@useobject{currentmarker}{}%
\end{pgfscope}%
\begin{pgfscope}%
\pgfsys@transformshift{1.096277in}{0.238120in}%
\pgfsys@useobject{currentmarker}{}%
\end{pgfscope}%
\begin{pgfscope}%
\pgfsys@transformshift{1.145745in}{0.290640in}%
\pgfsys@useobject{currentmarker}{}%
\end{pgfscope}%
\begin{pgfscope}%
\pgfsys@transformshift{1.195213in}{0.395922in}%
\pgfsys@useobject{currentmarker}{}%
\end{pgfscope}%
\begin{pgfscope}%
\pgfsys@transformshift{1.244681in}{0.572813in}%
\pgfsys@useobject{currentmarker}{}%
\end{pgfscope}%
\begin{pgfscope}%
\pgfsys@transformshift{1.294149in}{0.810181in}%
\pgfsys@useobject{currentmarker}{}%
\end{pgfscope}%
\begin{pgfscope}%
\pgfsys@transformshift{1.343617in}{1.069789in}%
\pgfsys@useobject{currentmarker}{}%
\end{pgfscope}%
\begin{pgfscope}%
\pgfsys@transformshift{1.393085in}{1.279158in}%
\pgfsys@useobject{currentmarker}{}%
\end{pgfscope}%
\begin{pgfscope}%
\pgfsys@transformshift{1.442553in}{1.484088in}%
\pgfsys@useobject{currentmarker}{}%
\end{pgfscope}%
\begin{pgfscope}%
\pgfsys@transformshift{1.492021in}{1.645717in}%
\pgfsys@useobject{currentmarker}{}%
\end{pgfscope}%
\begin{pgfscope}%
\pgfsys@transformshift{1.541489in}{1.785632in}%
\pgfsys@useobject{currentmarker}{}%
\end{pgfscope}%
\begin{pgfscope}%
\pgfsys@transformshift{1.590957in}{1.786424in}%
\pgfsys@useobject{currentmarker}{}%
\end{pgfscope}%
\begin{pgfscope}%
\pgfsys@transformshift{1.640426in}{1.728629in}%
\pgfsys@useobject{currentmarker}{}%
\end{pgfscope}%
\begin{pgfscope}%
\pgfsys@transformshift{1.689894in}{1.576384in}%
\pgfsys@useobject{currentmarker}{}%
\end{pgfscope}%
\begin{pgfscope}%
\pgfsys@transformshift{1.739362in}{1.458726in}%
\pgfsys@useobject{currentmarker}{}%
\end{pgfscope}%
\begin{pgfscope}%
\pgfsys@transformshift{1.788830in}{1.381229in}%
\pgfsys@useobject{currentmarker}{}%
\end{pgfscope}%
\begin{pgfscope}%
\pgfsys@transformshift{1.838298in}{1.234174in}%
\pgfsys@useobject{currentmarker}{}%
\end{pgfscope}%
\begin{pgfscope}%
\pgfsys@transformshift{1.887766in}{1.026258in}%
\pgfsys@useobject{currentmarker}{}%
\end{pgfscope}%
\begin{pgfscope}%
\pgfsys@transformshift{1.937234in}{0.824533in}%
\pgfsys@useobject{currentmarker}{}%
\end{pgfscope}%
\begin{pgfscope}%
\pgfsys@transformshift{1.986702in}{0.664515in}%
\pgfsys@useobject{currentmarker}{}%
\end{pgfscope}%
\begin{pgfscope}%
\pgfsys@transformshift{2.036170in}{0.538987in}%
\pgfsys@useobject{currentmarker}{}%
\end{pgfscope}%
\begin{pgfscope}%
\pgfsys@transformshift{2.085638in}{0.456842in}%
\pgfsys@useobject{currentmarker}{}%
\end{pgfscope}%
\begin{pgfscope}%
\pgfsys@transformshift{2.135106in}{0.396542in}%
\pgfsys@useobject{currentmarker}{}%
\end{pgfscope}%
\begin{pgfscope}%
\pgfsys@transformshift{2.184574in}{0.342234in}%
\pgfsys@useobject{currentmarker}{}%
\end{pgfscope}%
\begin{pgfscope}%
\pgfsys@transformshift{2.234043in}{0.297314in}%
\pgfsys@useobject{currentmarker}{}%
\end{pgfscope}%
\begin{pgfscope}%
\pgfsys@transformshift{2.283511in}{0.268500in}%
\pgfsys@useobject{currentmarker}{}%
\end{pgfscope}%
\begin{pgfscope}%
\pgfsys@transformshift{2.332979in}{0.246725in}%
\pgfsys@useobject{currentmarker}{}%
\end{pgfscope}%
\begin{pgfscope}%
\pgfsys@transformshift{2.382447in}{0.231026in}%
\pgfsys@useobject{currentmarker}{}%
\end{pgfscope}%
\begin{pgfscope}%
\pgfsys@transformshift{2.431915in}{0.220162in}%
\pgfsys@useobject{currentmarker}{}%
\end{pgfscope}%
\begin{pgfscope}%
\pgfsys@transformshift{2.481383in}{0.212951in}%
\pgfsys@useobject{currentmarker}{}%
\end{pgfscope}%
\begin{pgfscope}%
\pgfsys@transformshift{2.530851in}{0.208142in}%
\pgfsys@useobject{currentmarker}{}%
\end{pgfscope}%
\begin{pgfscope}%
\pgfsys@transformshift{2.580319in}{0.205070in}%
\pgfsys@useobject{currentmarker}{}%
\end{pgfscope}%
\begin{pgfscope}%
\pgfsys@transformshift{2.629787in}{0.203134in}%
\pgfsys@useobject{currentmarker}{}%
\end{pgfscope}%
\begin{pgfscope}%
\pgfsys@transformshift{2.679255in}{0.201924in}%
\pgfsys@useobject{currentmarker}{}%
\end{pgfscope}%
\begin{pgfscope}%
\pgfsys@transformshift{2.728723in}{0.201184in}%
\pgfsys@useobject{currentmarker}{}%
\end{pgfscope}%
\begin{pgfscope}%
\pgfsys@transformshift{2.778191in}{0.200730in}%
\pgfsys@useobject{currentmarker}{}%
\end{pgfscope}%
\begin{pgfscope}%
\pgfsys@transformshift{2.827660in}{0.200451in}%
\pgfsys@useobject{currentmarker}{}%
\end{pgfscope}%
\begin{pgfscope}%
\pgfsys@transformshift{2.877128in}{0.200280in}%
\pgfsys@useobject{currentmarker}{}%
\end{pgfscope}%
\begin{pgfscope}%
\pgfsys@transformshift{2.926596in}{0.200175in}%
\pgfsys@useobject{currentmarker}{}%
\end{pgfscope}%
\begin{pgfscope}%
\pgfsys@transformshift{2.976064in}{0.200110in}%
\pgfsys@useobject{currentmarker}{}%
\end{pgfscope}%
\begin{pgfscope}%
\pgfsys@transformshift{3.025532in}{0.200070in}%
\pgfsys@useobject{currentmarker}{}%
\end{pgfscope}%
\begin{pgfscope}%
\pgfsys@transformshift{3.075000in}{0.200045in}%
\pgfsys@useobject{currentmarker}{}%
\end{pgfscope}%
\begin{pgfscope}%
\pgfsys@transformshift{3.124468in}{0.200029in}%
\pgfsys@useobject{currentmarker}{}%
\end{pgfscope}%
\begin{pgfscope}%
\pgfsys@transformshift{3.173936in}{0.200019in}%
\pgfsys@useobject{currentmarker}{}%
\end{pgfscope}%
\begin{pgfscope}%
\pgfsys@transformshift{3.223404in}{0.200013in}%
\pgfsys@useobject{currentmarker}{}%
\end{pgfscope}%
\begin{pgfscope}%
\pgfsys@transformshift{3.272872in}{0.200009in}%
\pgfsys@useobject{currentmarker}{}%
\end{pgfscope}%
\begin{pgfscope}%
\pgfsys@transformshift{3.322340in}{0.200000in}%
\pgfsys@useobject{currentmarker}{}%
\end{pgfscope}%
\begin{pgfscope}%
\pgfsys@transformshift{3.371809in}{0.200000in}%
\pgfsys@useobject{currentmarker}{}%
\end{pgfscope}%
\begin{pgfscope}%
\pgfsys@transformshift{3.421277in}{0.200000in}%
\pgfsys@useobject{currentmarker}{}%
\end{pgfscope}%
\begin{pgfscope}%
\pgfsys@transformshift{3.470745in}{0.200000in}%
\pgfsys@useobject{currentmarker}{}%
\end{pgfscope}%
\begin{pgfscope}%
\pgfsys@transformshift{3.520213in}{0.200000in}%
\pgfsys@useobject{currentmarker}{}%
\end{pgfscope}%
\begin{pgfscope}%
\pgfsys@transformshift{3.569681in}{0.200000in}%
\pgfsys@useobject{currentmarker}{}%
\end{pgfscope}%
\begin{pgfscope}%
\pgfsys@transformshift{3.619149in}{0.200000in}%
\pgfsys@useobject{currentmarker}{}%
\end{pgfscope}%
\begin{pgfscope}%
\pgfsys@transformshift{3.668617in}{0.200000in}%
\pgfsys@useobject{currentmarker}{}%
\end{pgfscope}%
\begin{pgfscope}%
\pgfsys@transformshift{3.718085in}{0.200000in}%
\pgfsys@useobject{currentmarker}{}%
\end{pgfscope}%
\begin{pgfscope}%
\pgfsys@transformshift{3.767553in}{0.200000in}%
\pgfsys@useobject{currentmarker}{}%
\end{pgfscope}%
\begin{pgfscope}%
\pgfsys@transformshift{3.817021in}{0.200000in}%
\pgfsys@useobject{currentmarker}{}%
\end{pgfscope}%
\begin{pgfscope}%
\pgfsys@transformshift{3.866489in}{0.200000in}%
\pgfsys@useobject{currentmarker}{}%
\end{pgfscope}%
\begin{pgfscope}%
\pgfsys@transformshift{3.915957in}{0.200000in}%
\pgfsys@useobject{currentmarker}{}%
\end{pgfscope}%
\begin{pgfscope}%
\pgfsys@transformshift{3.965426in}{0.200000in}%
\pgfsys@useobject{currentmarker}{}%
\end{pgfscope}%
\begin{pgfscope}%
\pgfsys@transformshift{4.014894in}{0.200000in}%
\pgfsys@useobject{currentmarker}{}%
\end{pgfscope}%
\begin{pgfscope}%
\pgfsys@transformshift{4.064362in}{0.200000in}%
\pgfsys@useobject{currentmarker}{}%
\end{pgfscope}%
\begin{pgfscope}%
\pgfsys@transformshift{4.113830in}{0.200000in}%
\pgfsys@useobject{currentmarker}{}%
\end{pgfscope}%
\begin{pgfscope}%
\pgfsys@transformshift{4.163298in}{0.200000in}%
\pgfsys@useobject{currentmarker}{}%
\end{pgfscope}%
\begin{pgfscope}%
\pgfsys@transformshift{4.212766in}{0.200000in}%
\pgfsys@useobject{currentmarker}{}%
\end{pgfscope}%
\begin{pgfscope}%
\pgfsys@transformshift{4.262234in}{0.200000in}%
\pgfsys@useobject{currentmarker}{}%
\end{pgfscope}%
\begin{pgfscope}%
\pgfsys@transformshift{4.311702in}{0.200000in}%
\pgfsys@useobject{currentmarker}{}%
\end{pgfscope}%
\begin{pgfscope}%
\pgfsys@transformshift{4.361170in}{0.200000in}%
\pgfsys@useobject{currentmarker}{}%
\end{pgfscope}%
\begin{pgfscope}%
\pgfsys@transformshift{4.410638in}{0.200000in}%
\pgfsys@useobject{currentmarker}{}%
\end{pgfscope}%
\begin{pgfscope}%
\pgfsys@transformshift{4.460106in}{0.200000in}%
\pgfsys@useobject{currentmarker}{}%
\end{pgfscope}%
\begin{pgfscope}%
\pgfsys@transformshift{4.509574in}{0.200000in}%
\pgfsys@useobject{currentmarker}{}%
\end{pgfscope}%
\begin{pgfscope}%
\pgfsys@transformshift{4.559043in}{0.200000in}%
\pgfsys@useobject{currentmarker}{}%
\end{pgfscope}%
\begin{pgfscope}%
\pgfsys@transformshift{4.608511in}{0.200000in}%
\pgfsys@useobject{currentmarker}{}%
\end{pgfscope}%
\begin{pgfscope}%
\pgfsys@transformshift{4.657979in}{0.200000in}%
\pgfsys@useobject{currentmarker}{}%
\end{pgfscope}%
\begin{pgfscope}%
\pgfsys@transformshift{4.707447in}{0.200000in}%
\pgfsys@useobject{currentmarker}{}%
\end{pgfscope}%
\begin{pgfscope}%
\pgfsys@transformshift{4.756915in}{0.200000in}%
\pgfsys@useobject{currentmarker}{}%
\end{pgfscope}%
\begin{pgfscope}%
\pgfsys@transformshift{4.806383in}{0.200000in}%
\pgfsys@useobject{currentmarker}{}%
\end{pgfscope}%
\begin{pgfscope}%
\pgfsys@transformshift{4.855851in}{0.200000in}%
\pgfsys@useobject{currentmarker}{}%
\end{pgfscope}%
\begin{pgfscope}%
\pgfsys@transformshift{4.905319in}{0.200000in}%
\pgfsys@useobject{currentmarker}{}%
\end{pgfscope}%
\begin{pgfscope}%
\pgfsys@transformshift{4.954787in}{0.200000in}%
\pgfsys@useobject{currentmarker}{}%
\end{pgfscope}%
\begin{pgfscope}%
\pgfsys@transformshift{5.004255in}{0.200000in}%
\pgfsys@useobject{currentmarker}{}%
\end{pgfscope}%
\begin{pgfscope}%
\pgfsys@transformshift{5.053723in}{0.200000in}%
\pgfsys@useobject{currentmarker}{}%
\end{pgfscope}%
\begin{pgfscope}%
\pgfsys@transformshift{5.103191in}{0.200000in}%
\pgfsys@useobject{currentmarker}{}%
\end{pgfscope}%
\begin{pgfscope}%
\pgfsys@transformshift{5.152660in}{0.200000in}%
\pgfsys@useobject{currentmarker}{}%
\end{pgfscope}%
\begin{pgfscope}%
\pgfsys@transformshift{5.202128in}{0.200000in}%
\pgfsys@useobject{currentmarker}{}%
\end{pgfscope}%
\begin{pgfscope}%
\pgfsys@transformshift{5.251596in}{0.200000in}%
\pgfsys@useobject{currentmarker}{}%
\end{pgfscope}%
\begin{pgfscope}%
\pgfsys@transformshift{5.301064in}{0.200000in}%
\pgfsys@useobject{currentmarker}{}%
\end{pgfscope}%
\begin{pgfscope}%
\pgfsys@transformshift{5.350532in}{0.200000in}%
\pgfsys@useobject{currentmarker}{}%
\end{pgfscope}%
\begin{pgfscope}%
\pgfsys@transformshift{5.400000in}{0.200000in}%
\pgfsys@useobject{currentmarker}{}%
\end{pgfscope}%
\end{pgfscope}%
\begin{pgfscope}%
\pgfsetbuttcap%
\pgfsetroundjoin%
\definecolor{currentfill}{rgb}{0.000000,0.000000,0.000000}%
\pgfsetfillcolor{currentfill}%
\pgfsetlinewidth{0.501875pt}%
\definecolor{currentstroke}{rgb}{0.000000,0.000000,0.000000}%
\pgfsetstrokecolor{currentstroke}%
\pgfsetdash{}{0pt}%
\pgfsys@defobject{currentmarker}{\pgfqpoint{0.000000in}{0.000000in}}{\pgfqpoint{0.000000in}{0.055556in}}{%
\pgfpathmoveto{\pgfqpoint{0.000000in}{0.000000in}}%
\pgfpathlineto{\pgfqpoint{0.000000in}{0.055556in}}%
\pgfusepath{stroke,fill}%
}%
\begin{pgfscope}%
\pgfsys@transformshift{1.145745in}{0.200000in}%
\pgfsys@useobject{currentmarker}{}%
\end{pgfscope}%
\end{pgfscope}%
\begin{pgfscope}%
\pgfsetbuttcap%
\pgfsetroundjoin%
\definecolor{currentfill}{rgb}{0.000000,0.000000,0.000000}%
\pgfsetfillcolor{currentfill}%
\pgfsetlinewidth{0.501875pt}%
\definecolor{currentstroke}{rgb}{0.000000,0.000000,0.000000}%
\pgfsetstrokecolor{currentstroke}%
\pgfsetdash{}{0pt}%
\pgfsys@defobject{currentmarker}{\pgfqpoint{0.000000in}{-0.055556in}}{\pgfqpoint{0.000000in}{0.000000in}}{%
\pgfpathmoveto{\pgfqpoint{0.000000in}{0.000000in}}%
\pgfpathlineto{\pgfqpoint{0.000000in}{-0.055556in}}%
\pgfusepath{stroke,fill}%
}%
\begin{pgfscope}%
\pgfsys@transformshift{1.145745in}{1.800000in}%
\pgfsys@useobject{currentmarker}{}%
\end{pgfscope}%
\end{pgfscope}%
\begin{pgfscope}%
\pgftext[left,bottom,x=1.013197in,y=0.039396in,rotate=0.000000]{{\fontsize{10.000000}{12.000000}\selectfont 400}}
%
\end{pgfscope}%
\begin{pgfscope}%
\pgfsetbuttcap%
\pgfsetroundjoin%
\definecolor{currentfill}{rgb}{0.000000,0.000000,0.000000}%
\pgfsetfillcolor{currentfill}%
\pgfsetlinewidth{0.501875pt}%
\definecolor{currentstroke}{rgb}{0.000000,0.000000,0.000000}%
\pgfsetstrokecolor{currentstroke}%
\pgfsetdash{}{0pt}%
\pgfsys@defobject{currentmarker}{\pgfqpoint{0.000000in}{0.000000in}}{\pgfqpoint{0.000000in}{0.055556in}}{%
\pgfpathmoveto{\pgfqpoint{0.000000in}{0.000000in}}%
\pgfpathlineto{\pgfqpoint{0.000000in}{0.055556in}}%
\pgfusepath{stroke,fill}%
}%
\begin{pgfscope}%
\pgfsys@transformshift{2.135106in}{0.200000in}%
\pgfsys@useobject{currentmarker}{}%
\end{pgfscope}%
\end{pgfscope}%
\begin{pgfscope}%
\pgfsetbuttcap%
\pgfsetroundjoin%
\definecolor{currentfill}{rgb}{0.000000,0.000000,0.000000}%
\pgfsetfillcolor{currentfill}%
\pgfsetlinewidth{0.501875pt}%
\definecolor{currentstroke}{rgb}{0.000000,0.000000,0.000000}%
\pgfsetstrokecolor{currentstroke}%
\pgfsetdash{}{0pt}%
\pgfsys@defobject{currentmarker}{\pgfqpoint{0.000000in}{-0.055556in}}{\pgfqpoint{0.000000in}{0.000000in}}{%
\pgfpathmoveto{\pgfqpoint{0.000000in}{0.000000in}}%
\pgfpathlineto{\pgfqpoint{0.000000in}{-0.055556in}}%
\pgfusepath{stroke,fill}%
}%
\begin{pgfscope}%
\pgfsys@transformshift{2.135106in}{1.800000in}%
\pgfsys@useobject{currentmarker}{}%
\end{pgfscope}%
\end{pgfscope}%
\begin{pgfscope}%
\pgftext[left,bottom,x=2.002558in,y=0.039396in,rotate=0.000000]{{\fontsize{10.000000}{12.000000}\selectfont 500}}
%
\end{pgfscope}%
\begin{pgfscope}%
\pgfsetbuttcap%
\pgfsetroundjoin%
\definecolor{currentfill}{rgb}{0.000000,0.000000,0.000000}%
\pgfsetfillcolor{currentfill}%
\pgfsetlinewidth{0.501875pt}%
\definecolor{currentstroke}{rgb}{0.000000,0.000000,0.000000}%
\pgfsetstrokecolor{currentstroke}%
\pgfsetdash{}{0pt}%
\pgfsys@defobject{currentmarker}{\pgfqpoint{0.000000in}{0.000000in}}{\pgfqpoint{0.000000in}{0.055556in}}{%
\pgfpathmoveto{\pgfqpoint{0.000000in}{0.000000in}}%
\pgfpathlineto{\pgfqpoint{0.000000in}{0.055556in}}%
\pgfusepath{stroke,fill}%
}%
\begin{pgfscope}%
\pgfsys@transformshift{3.124468in}{0.200000in}%
\pgfsys@useobject{currentmarker}{}%
\end{pgfscope}%
\end{pgfscope}%
\begin{pgfscope}%
\pgfsetbuttcap%
\pgfsetroundjoin%
\definecolor{currentfill}{rgb}{0.000000,0.000000,0.000000}%
\pgfsetfillcolor{currentfill}%
\pgfsetlinewidth{0.501875pt}%
\definecolor{currentstroke}{rgb}{0.000000,0.000000,0.000000}%
\pgfsetstrokecolor{currentstroke}%
\pgfsetdash{}{0pt}%
\pgfsys@defobject{currentmarker}{\pgfqpoint{0.000000in}{-0.055556in}}{\pgfqpoint{0.000000in}{0.000000in}}{%
\pgfpathmoveto{\pgfqpoint{0.000000in}{0.000000in}}%
\pgfpathlineto{\pgfqpoint{0.000000in}{-0.055556in}}%
\pgfusepath{stroke,fill}%
}%
\begin{pgfscope}%
\pgfsys@transformshift{3.124468in}{1.800000in}%
\pgfsys@useobject{currentmarker}{}%
\end{pgfscope}%
\end{pgfscope}%
\begin{pgfscope}%
\pgftext[left,bottom,x=2.991920in,y=0.039396in,rotate=0.000000]{{\fontsize{10.000000}{12.000000}\selectfont 600}}
%
\end{pgfscope}%
\begin{pgfscope}%
\pgfsetbuttcap%
\pgfsetroundjoin%
\definecolor{currentfill}{rgb}{0.000000,0.000000,0.000000}%
\pgfsetfillcolor{currentfill}%
\pgfsetlinewidth{0.501875pt}%
\definecolor{currentstroke}{rgb}{0.000000,0.000000,0.000000}%
\pgfsetstrokecolor{currentstroke}%
\pgfsetdash{}{0pt}%
\pgfsys@defobject{currentmarker}{\pgfqpoint{0.000000in}{0.000000in}}{\pgfqpoint{0.000000in}{0.055556in}}{%
\pgfpathmoveto{\pgfqpoint{0.000000in}{0.000000in}}%
\pgfpathlineto{\pgfqpoint{0.000000in}{0.055556in}}%
\pgfusepath{stroke,fill}%
}%
\begin{pgfscope}%
\pgfsys@transformshift{4.113830in}{0.200000in}%
\pgfsys@useobject{currentmarker}{}%
\end{pgfscope}%
\end{pgfscope}%
\begin{pgfscope}%
\pgfsetbuttcap%
\pgfsetroundjoin%
\definecolor{currentfill}{rgb}{0.000000,0.000000,0.000000}%
\pgfsetfillcolor{currentfill}%
\pgfsetlinewidth{0.501875pt}%
\definecolor{currentstroke}{rgb}{0.000000,0.000000,0.000000}%
\pgfsetstrokecolor{currentstroke}%
\pgfsetdash{}{0pt}%
\pgfsys@defobject{currentmarker}{\pgfqpoint{0.000000in}{-0.055556in}}{\pgfqpoint{0.000000in}{0.000000in}}{%
\pgfpathmoveto{\pgfqpoint{0.000000in}{0.000000in}}%
\pgfpathlineto{\pgfqpoint{0.000000in}{-0.055556in}}%
\pgfusepath{stroke,fill}%
}%
\begin{pgfscope}%
\pgfsys@transformshift{4.113830in}{1.800000in}%
\pgfsys@useobject{currentmarker}{}%
\end{pgfscope}%
\end{pgfscope}%
\begin{pgfscope}%
\pgftext[left,bottom,x=3.981282in,y=0.039396in,rotate=0.000000]{{\fontsize{10.000000}{12.000000}\selectfont 700}}
%
\end{pgfscope}%
\begin{pgfscope}%
\pgfsetbuttcap%
\pgfsetroundjoin%
\definecolor{currentfill}{rgb}{0.000000,0.000000,0.000000}%
\pgfsetfillcolor{currentfill}%
\pgfsetlinewidth{0.501875pt}%
\definecolor{currentstroke}{rgb}{0.000000,0.000000,0.000000}%
\pgfsetstrokecolor{currentstroke}%
\pgfsetdash{}{0pt}%
\pgfsys@defobject{currentmarker}{\pgfqpoint{0.000000in}{0.000000in}}{\pgfqpoint{0.000000in}{0.055556in}}{%
\pgfpathmoveto{\pgfqpoint{0.000000in}{0.000000in}}%
\pgfpathlineto{\pgfqpoint{0.000000in}{0.055556in}}%
\pgfusepath{stroke,fill}%
}%
\begin{pgfscope}%
\pgfsys@transformshift{5.103191in}{0.200000in}%
\pgfsys@useobject{currentmarker}{}%
\end{pgfscope}%
\end{pgfscope}%
\begin{pgfscope}%
\pgfsetbuttcap%
\pgfsetroundjoin%
\definecolor{currentfill}{rgb}{0.000000,0.000000,0.000000}%
\pgfsetfillcolor{currentfill}%
\pgfsetlinewidth{0.501875pt}%
\definecolor{currentstroke}{rgb}{0.000000,0.000000,0.000000}%
\pgfsetstrokecolor{currentstroke}%
\pgfsetdash{}{0pt}%
\pgfsys@defobject{currentmarker}{\pgfqpoint{0.000000in}{-0.055556in}}{\pgfqpoint{0.000000in}{0.000000in}}{%
\pgfpathmoveto{\pgfqpoint{0.000000in}{0.000000in}}%
\pgfpathlineto{\pgfqpoint{0.000000in}{-0.055556in}}%
\pgfusepath{stroke,fill}%
}%
\begin{pgfscope}%
\pgfsys@transformshift{5.103191in}{1.800000in}%
\pgfsys@useobject{currentmarker}{}%
\end{pgfscope}%
\end{pgfscope}%
\begin{pgfscope}%
\pgftext[left,bottom,x=4.970644in,y=0.039396in,rotate=0.000000]{{\fontsize{10.000000}{12.000000}\selectfont 800}}
%
\end{pgfscope}%
\begin{pgfscope}%
\pgfsetbuttcap%
\pgfsetroundjoin%
\definecolor{currentfill}{rgb}{0.000000,0.000000,0.000000}%
\pgfsetfillcolor{currentfill}%
\pgfsetlinewidth{0.501875pt}%
\definecolor{currentstroke}{rgb}{0.000000,0.000000,0.000000}%
\pgfsetstrokecolor{currentstroke}%
\pgfsetdash{}{0pt}%
\pgfsys@defobject{currentmarker}{\pgfqpoint{0.000000in}{0.000000in}}{\pgfqpoint{0.055556in}{0.000000in}}{%
\pgfpathmoveto{\pgfqpoint{0.000000in}{0.000000in}}%
\pgfpathlineto{\pgfqpoint{0.055556in}{0.000000in}}%
\pgfusepath{stroke,fill}%
}%
\begin{pgfscope}%
\pgfsys@transformshift{0.750000in}{0.200000in}%
\pgfsys@useobject{currentmarker}{}%
\end{pgfscope}%
\end{pgfscope}%
\begin{pgfscope}%
\pgfsetbuttcap%
\pgfsetroundjoin%
\definecolor{currentfill}{rgb}{0.000000,0.000000,0.000000}%
\pgfsetfillcolor{currentfill}%
\pgfsetlinewidth{0.501875pt}%
\definecolor{currentstroke}{rgb}{0.000000,0.000000,0.000000}%
\pgfsetstrokecolor{currentstroke}%
\pgfsetdash{}{0pt}%
\pgfsys@defobject{currentmarker}{\pgfqpoint{-0.055556in}{0.000000in}}{\pgfqpoint{0.000000in}{0.000000in}}{%
\pgfpathmoveto{\pgfqpoint{0.000000in}{0.000000in}}%
\pgfpathlineto{\pgfqpoint{-0.055556in}{0.000000in}}%
\pgfusepath{stroke,fill}%
}%
\begin{pgfscope}%
\pgfsys@transformshift{5.400000in}{0.200000in}%
\pgfsys@useobject{currentmarker}{}%
\end{pgfscope}%
\end{pgfscope}%
\begin{pgfscope}%
\pgftext[left,bottom,x=0.473565in,y=0.147476in,rotate=0.000000]{{\fontsize{10.000000}{12.000000}\selectfont 0.0}}
%
\end{pgfscope}%
\begin{pgfscope}%
\pgfsetbuttcap%
\pgfsetroundjoin%
\definecolor{currentfill}{rgb}{0.000000,0.000000,0.000000}%
\pgfsetfillcolor{currentfill}%
\pgfsetlinewidth{0.501875pt}%
\definecolor{currentstroke}{rgb}{0.000000,0.000000,0.000000}%
\pgfsetstrokecolor{currentstroke}%
\pgfsetdash{}{0pt}%
\pgfsys@defobject{currentmarker}{\pgfqpoint{0.000000in}{0.000000in}}{\pgfqpoint{0.055556in}{0.000000in}}{%
\pgfpathmoveto{\pgfqpoint{0.000000in}{0.000000in}}%
\pgfpathlineto{\pgfqpoint{0.055556in}{0.000000in}}%
\pgfusepath{stroke,fill}%
}%
\begin{pgfscope}%
\pgfsys@transformshift{0.750000in}{0.520000in}%
\pgfsys@useobject{currentmarker}{}%
\end{pgfscope}%
\end{pgfscope}%
\begin{pgfscope}%
\pgfsetbuttcap%
\pgfsetroundjoin%
\definecolor{currentfill}{rgb}{0.000000,0.000000,0.000000}%
\pgfsetfillcolor{currentfill}%
\pgfsetlinewidth{0.501875pt}%
\definecolor{currentstroke}{rgb}{0.000000,0.000000,0.000000}%
\pgfsetstrokecolor{currentstroke}%
\pgfsetdash{}{0pt}%
\pgfsys@defobject{currentmarker}{\pgfqpoint{-0.055556in}{0.000000in}}{\pgfqpoint{0.000000in}{0.000000in}}{%
\pgfpathmoveto{\pgfqpoint{0.000000in}{0.000000in}}%
\pgfpathlineto{\pgfqpoint{-0.055556in}{0.000000in}}%
\pgfusepath{stroke,fill}%
}%
\begin{pgfscope}%
\pgfsys@transformshift{5.400000in}{0.520000in}%
\pgfsys@useobject{currentmarker}{}%
\end{pgfscope}%
\end{pgfscope}%
\begin{pgfscope}%
\pgftext[left,bottom,x=0.473565in,y=0.467476in,rotate=0.000000]{{\fontsize{10.000000}{12.000000}\selectfont 0.2}}
%
\end{pgfscope}%
\begin{pgfscope}%
\pgfsetbuttcap%
\pgfsetroundjoin%
\definecolor{currentfill}{rgb}{0.000000,0.000000,0.000000}%
\pgfsetfillcolor{currentfill}%
\pgfsetlinewidth{0.501875pt}%
\definecolor{currentstroke}{rgb}{0.000000,0.000000,0.000000}%
\pgfsetstrokecolor{currentstroke}%
\pgfsetdash{}{0pt}%
\pgfsys@defobject{currentmarker}{\pgfqpoint{0.000000in}{0.000000in}}{\pgfqpoint{0.055556in}{0.000000in}}{%
\pgfpathmoveto{\pgfqpoint{0.000000in}{0.000000in}}%
\pgfpathlineto{\pgfqpoint{0.055556in}{0.000000in}}%
\pgfusepath{stroke,fill}%
}%
\begin{pgfscope}%
\pgfsys@transformshift{0.750000in}{0.840000in}%
\pgfsys@useobject{currentmarker}{}%
\end{pgfscope}%
\end{pgfscope}%
\begin{pgfscope}%
\pgfsetbuttcap%
\pgfsetroundjoin%
\definecolor{currentfill}{rgb}{0.000000,0.000000,0.000000}%
\pgfsetfillcolor{currentfill}%
\pgfsetlinewidth{0.501875pt}%
\definecolor{currentstroke}{rgb}{0.000000,0.000000,0.000000}%
\pgfsetstrokecolor{currentstroke}%
\pgfsetdash{}{0pt}%
\pgfsys@defobject{currentmarker}{\pgfqpoint{-0.055556in}{0.000000in}}{\pgfqpoint{0.000000in}{0.000000in}}{%
\pgfpathmoveto{\pgfqpoint{0.000000in}{0.000000in}}%
\pgfpathlineto{\pgfqpoint{-0.055556in}{0.000000in}}%
\pgfusepath{stroke,fill}%
}%
\begin{pgfscope}%
\pgfsys@transformshift{5.400000in}{0.840000in}%
\pgfsys@useobject{currentmarker}{}%
\end{pgfscope}%
\end{pgfscope}%
\begin{pgfscope}%
\pgftext[left,bottom,x=0.473565in,y=0.787476in,rotate=0.000000]{{\fontsize{10.000000}{12.000000}\selectfont 0.4}}
%
\end{pgfscope}%
\begin{pgfscope}%
\pgfsetbuttcap%
\pgfsetroundjoin%
\definecolor{currentfill}{rgb}{0.000000,0.000000,0.000000}%
\pgfsetfillcolor{currentfill}%
\pgfsetlinewidth{0.501875pt}%
\definecolor{currentstroke}{rgb}{0.000000,0.000000,0.000000}%
\pgfsetstrokecolor{currentstroke}%
\pgfsetdash{}{0pt}%
\pgfsys@defobject{currentmarker}{\pgfqpoint{0.000000in}{0.000000in}}{\pgfqpoint{0.055556in}{0.000000in}}{%
\pgfpathmoveto{\pgfqpoint{0.000000in}{0.000000in}}%
\pgfpathlineto{\pgfqpoint{0.055556in}{0.000000in}}%
\pgfusepath{stroke,fill}%
}%
\begin{pgfscope}%
\pgfsys@transformshift{0.750000in}{1.160000in}%
\pgfsys@useobject{currentmarker}{}%
\end{pgfscope}%
\end{pgfscope}%
\begin{pgfscope}%
\pgfsetbuttcap%
\pgfsetroundjoin%
\definecolor{currentfill}{rgb}{0.000000,0.000000,0.000000}%
\pgfsetfillcolor{currentfill}%
\pgfsetlinewidth{0.501875pt}%
\definecolor{currentstroke}{rgb}{0.000000,0.000000,0.000000}%
\pgfsetstrokecolor{currentstroke}%
\pgfsetdash{}{0pt}%
\pgfsys@defobject{currentmarker}{\pgfqpoint{-0.055556in}{0.000000in}}{\pgfqpoint{0.000000in}{0.000000in}}{%
\pgfpathmoveto{\pgfqpoint{0.000000in}{0.000000in}}%
\pgfpathlineto{\pgfqpoint{-0.055556in}{0.000000in}}%
\pgfusepath{stroke,fill}%
}%
\begin{pgfscope}%
\pgfsys@transformshift{5.400000in}{1.160000in}%
\pgfsys@useobject{currentmarker}{}%
\end{pgfscope}%
\end{pgfscope}%
\begin{pgfscope}%
\pgftext[left,bottom,x=0.473565in,y=1.107476in,rotate=0.000000]{{\fontsize{10.000000}{12.000000}\selectfont 0.6}}
%
\end{pgfscope}%
\begin{pgfscope}%
\pgfsetbuttcap%
\pgfsetroundjoin%
\definecolor{currentfill}{rgb}{0.000000,0.000000,0.000000}%
\pgfsetfillcolor{currentfill}%
\pgfsetlinewidth{0.501875pt}%
\definecolor{currentstroke}{rgb}{0.000000,0.000000,0.000000}%
\pgfsetstrokecolor{currentstroke}%
\pgfsetdash{}{0pt}%
\pgfsys@defobject{currentmarker}{\pgfqpoint{0.000000in}{0.000000in}}{\pgfqpoint{0.055556in}{0.000000in}}{%
\pgfpathmoveto{\pgfqpoint{0.000000in}{0.000000in}}%
\pgfpathlineto{\pgfqpoint{0.055556in}{0.000000in}}%
\pgfusepath{stroke,fill}%
}%
\begin{pgfscope}%
\pgfsys@transformshift{0.750000in}{1.480000in}%
\pgfsys@useobject{currentmarker}{}%
\end{pgfscope}%
\end{pgfscope}%
\begin{pgfscope}%
\pgfsetbuttcap%
\pgfsetroundjoin%
\definecolor{currentfill}{rgb}{0.000000,0.000000,0.000000}%
\pgfsetfillcolor{currentfill}%
\pgfsetlinewidth{0.501875pt}%
\definecolor{currentstroke}{rgb}{0.000000,0.000000,0.000000}%
\pgfsetstrokecolor{currentstroke}%
\pgfsetdash{}{0pt}%
\pgfsys@defobject{currentmarker}{\pgfqpoint{-0.055556in}{0.000000in}}{\pgfqpoint{0.000000in}{0.000000in}}{%
\pgfpathmoveto{\pgfqpoint{0.000000in}{0.000000in}}%
\pgfpathlineto{\pgfqpoint{-0.055556in}{0.000000in}}%
\pgfusepath{stroke,fill}%
}%
\begin{pgfscope}%
\pgfsys@transformshift{5.400000in}{1.480000in}%
\pgfsys@useobject{currentmarker}{}%
\end{pgfscope}%
\end{pgfscope}%
\begin{pgfscope}%
\pgftext[left,bottom,x=0.473565in,y=1.427476in,rotate=0.000000]{{\fontsize{10.000000}{12.000000}\selectfont 0.8}}
%
\end{pgfscope}%
\begin{pgfscope}%
\pgfsetbuttcap%
\pgfsetroundjoin%
\definecolor{currentfill}{rgb}{0.000000,0.000000,0.000000}%
\pgfsetfillcolor{currentfill}%
\pgfsetlinewidth{0.501875pt}%
\definecolor{currentstroke}{rgb}{0.000000,0.000000,0.000000}%
\pgfsetstrokecolor{currentstroke}%
\pgfsetdash{}{0pt}%
\pgfsys@defobject{currentmarker}{\pgfqpoint{0.000000in}{0.000000in}}{\pgfqpoint{0.055556in}{0.000000in}}{%
\pgfpathmoveto{\pgfqpoint{0.000000in}{0.000000in}}%
\pgfpathlineto{\pgfqpoint{0.055556in}{0.000000in}}%
\pgfusepath{stroke,fill}%
}%
\begin{pgfscope}%
\pgfsys@transformshift{0.750000in}{1.800000in}%
\pgfsys@useobject{currentmarker}{}%
\end{pgfscope}%
\end{pgfscope}%
\begin{pgfscope}%
\pgfsetbuttcap%
\pgfsetroundjoin%
\definecolor{currentfill}{rgb}{0.000000,0.000000,0.000000}%
\pgfsetfillcolor{currentfill}%
\pgfsetlinewidth{0.501875pt}%
\definecolor{currentstroke}{rgb}{0.000000,0.000000,0.000000}%
\pgfsetstrokecolor{currentstroke}%
\pgfsetdash{}{0pt}%
\pgfsys@defobject{currentmarker}{\pgfqpoint{-0.055556in}{0.000000in}}{\pgfqpoint{0.000000in}{0.000000in}}{%
\pgfpathmoveto{\pgfqpoint{0.000000in}{0.000000in}}%
\pgfpathlineto{\pgfqpoint{-0.055556in}{0.000000in}}%
\pgfusepath{stroke,fill}%
}%
\begin{pgfscope}%
\pgfsys@transformshift{5.400000in}{1.800000in}%
\pgfsys@useobject{currentmarker}{}%
\end{pgfscope}%
\end{pgfscope}%
\begin{pgfscope}%
\pgftext[left,bottom,x=0.473565in,y=1.747476in,rotate=0.000000]{{\fontsize{10.000000}{12.000000}\selectfont 1.0}}
%
\end{pgfscope}%
\begin{pgfscope}%
\pgfsetrectcap%
\pgfsetroundjoin%
\pgfsetlinewidth{1.003750pt}%
\definecolor{currentstroke}{rgb}{0.000000,0.000000,0.000000}%
\pgfsetstrokecolor{currentstroke}%
\pgfsetdash{}{0pt}%
\pgfpathmoveto{\pgfqpoint{0.750000in}{1.800000in}}%
\pgfpathlineto{\pgfqpoint{5.400000in}{1.800000in}}%
\pgfusepath{stroke}%
\end{pgfscope}%
\begin{pgfscope}%
\pgfsetrectcap%
\pgfsetroundjoin%
\pgfsetlinewidth{1.003750pt}%
\definecolor{currentstroke}{rgb}{0.000000,0.000000,0.000000}%
\pgfsetstrokecolor{currentstroke}%
\pgfsetdash{}{0pt}%
\pgfpathmoveto{\pgfqpoint{5.400000in}{0.200000in}}%
\pgfpathlineto{\pgfqpoint{5.400000in}{1.800000in}}%
\pgfusepath{stroke}%
\end{pgfscope}%
\begin{pgfscope}%
\pgfsetrectcap%
\pgfsetroundjoin%
\pgfsetlinewidth{1.003750pt}%
\definecolor{currentstroke}{rgb}{0.000000,0.000000,0.000000}%
\pgfsetstrokecolor{currentstroke}%
\pgfsetdash{}{0pt}%
\pgfpathmoveto{\pgfqpoint{0.750000in}{0.200000in}}%
\pgfpathlineto{\pgfqpoint{5.400000in}{0.200000in}}%
\pgfusepath{stroke}%
\end{pgfscope}%
\begin{pgfscope}%
\pgfsetrectcap%
\pgfsetroundjoin%
\pgfsetlinewidth{1.003750pt}%
\definecolor{currentstroke}{rgb}{0.000000,0.000000,0.000000}%
\pgfsetstrokecolor{currentstroke}%
\pgfsetdash{}{0pt}%
\pgfpathmoveto{\pgfqpoint{0.750000in}{0.200000in}}%
\pgfpathlineto{\pgfqpoint{0.750000in}{1.800000in}}%
\pgfusepath{stroke}%
\end{pgfscope}%
\end{pgfpicture}%
\makeatother%
\endgroup%

\caption{CIE LMS2006 Color Space curves}
\label{fig:cielms2006}
%\vskip 1mm
%{\footnotesize\it }
\end{figure}

\begin{displaymath}
C_{LMS} = M_{CAT02} C_{XYZ}  = \left[\begin{array}{r@{.}lr@{.}lr@{.}l}
 0&7328 & 0&4296 & -0&1624 \\
-0&7036 & 1&6975 &  0&0061 \\
 0&0030 & 0&0136 &  0&9834 \\
\end{array}\right] C_{XYZ}
\end{displaymath}
\fi


\begin{pythoncode}
# Computes XYZ coordinates from the given set of
# color matching functions
# l - list of wavelengths
# v - list of values
# xyzcmf an object containing
# .lnms - list of wavelengths
# .xbar - values for $\bar{x}(\lambda)$
# .ybar - values for $\bar{y}(\lambda)$
# .zbar - values for $\bar{z}(\lambda)$
def XYZ_lv(l, v, xyzcmf = CIE.CMF_1931):
    X = _integrate(l, v, xyzcmf.lnms, xyzcmf.xbar)
    Y = _integrate(l, v, xyzcmf.lnms, xyzcmf.ybar)
    Z = _integrate(l, v, xyzcmf.lnms, xyzcmf.zbar)

    return numpy.array([X, Y, Z])
\end{pythoncode}


\section{Texture maps}
\label{sec:implementation:texturemaps}

\subsection{RGB to spectrum conversions}

It's common practice to use \gls{RGB} texture maps to encode various
kind of spatially varying spectral distributions. In order to use them
in a spectral pipeline, a way is needed to try and reconstruct a plausible
spectrum from the data in these files. \cref{ch:lighting} introduced
the idea of discussing spatially varying spectra in terms of $n$-dimensional
vectors in a color basis called generically $col(\lambda)$ for which a
simple example was given for a naive reconstruction from \gls{RGB} as
\begin{displaymath}
M(p,\lambda) = \big\langle M_{rgb}(p), rgb(\lambda) \big\rangle
\end{displaymath}

Although appealing in its simplicity, this way of reconstructing a spectral
distribution from a texture would turn out to be only appropriate to model
emissive objects, for example some kind of \gls{RGB} monitor, but would be
a fairly poor approach to represent richer, more natural reflectance or
transmittance spectra.
In absence of sidecar information regarding what kind or class of spectral
distribution should be use to perform a reconstruction from a given texture,
a more believable way to rebuild spectral distributions from \gls{RGB} triples
is the algorithm proposed by Brian Smits in~\cite{smits99}: given $7$
spectra for white, cyan, magenta, yellow, red, green, blue:
\begin{displaymath}
smits(\lambda) = \big(w(\lambda), c(\lambda), m(\lambda), y(\lambda), r(\lambda), g(\lambda), b(\lambda)\big): \R^+\to\R^7
\end{displaymath}
a simple procedure is described to build a vector
\begin{displaymath}
M_{smits} = \big(M_w, M_c, M_m, M_y, M_r, M_g, M_b\big) \in \R^7
\end{displaymath}
in which at most three coordinates are non-zero.
Then the resulting spectrum would be
\begin{displaymath}
M(p,\lambda) = \big\langle M_{smits}(p), smits(\lambda) \big\rangle
\end{displaymath}
Spectra recovered this way have some desirable guarantees in terms of
smoothness, and the algorithm lends itself to a very efficient implementation.
The data presented here is based on a 10 bin spectral representation from
\SI{380}{\nano\meter} to \SI{720}{\nano\meter}. The bins are all equal size,
the function \Verb|bin()| return the bin index from a given wavelength


\begin{c++code}
const uint32_t c_numBins = 10;
const uint32_t c_numSpectra = 7;
const float    c_minWavelength = 380.f;
const float    c_maxWavelength = 720.f;

uint32_t bin(float wavelength)
{
    using std::min;
    using std::max;

    float binWidth = (c_maxWavelength - c_minWavelength) / c_numBins;

    uint32_t bin = static_cast<unit32_t>(
                        (wavelength - c_minWavelength) / binWidth);

    return min(max(bin, 0u), c_numBins - 1);
}
\end{c++code}


Follows the table of Smits' original data

\begin{c++code}
const float c_table[c_numBins * c_numSpectra] =
{
//  bin    white    cyan     magenta  yellow   red      green    blue
/*  0  */  1.0000f, 0.9710f, 1.0000f, 0.0001f, 0.1000f, 0.0000f, 1.0000f,
/*  1  */  1.0000f, 0.9426f, 1.0000f, 0.0000f, 0.0515f, 0.0000f, 1.0000f,
/*  2  */  0.9999f, 1.0007f, 0.9685f, 0.1088f, 0.0000f, 0.0273f, 0.8916f,
/*  3  */  0.9993f, 1.0007f, 0.2229f, 0.6651f, 0.0000f, 0.7937f, 0.3323f,
/*  4  */  0.9992f, 1.0007f, 0.0000f, 1.0000f, 0.0000f, 1.0000f, 0.0000f,
/*  5  */  0.9998f, 1.0007f, 0.0458f, 1.0000f, 0.0000f, 0.9418f, 0.0000f,
/*  6  */  1.0000f, 0.1564f, 0.8369f, 0.9996f, 0.8325f, 0.1719f, 0.0003f,
/*  7  */  1.0000f, 0.0000f, 1.0000f, 0.9586f, 1.0149f, 0.0000f, 0.0369f,
/*  8  */  1.0000f, 0.0000f, 1.0000f, 0.9685f, 1.0149f, 0.0000f, 0.0483f,
/*  9  */  1.0000f, 0.0000f, 0.9959f, 0.9840f, 1.0149f, 0.0025f, 0.0496f
};
\end{c++code}

and a helper function \Verb|evalBin()| which contains the core of Smits's
algorithm


\begin{c++code}
float evalBin(float r, float g, float b, uint32_t bin)
{
    const float* const row = c_table + bin * c_numSpectra;

    if (r <= g && r <= b) // red is smallest
        return     r * row[0] + ((g <= b) ? // white
            ((g - r) * row[1] +             // cyan
             (b - g) * row[6]) :            // blue
            ((b - r) * row[1] +             // cyan
             (g - b) * row[5]));            // green

    if (g <= r && g <= b) // green is smallest
        return     g * row[0] + ((r <= b) ? // white
            ((r - g) * row[2] +             // magenta
             (b - r) * row[6]) :            // blue
            ((b - g) * row[2] +             // magenta
             (r - b) * row[4]));            // red

    // blue is smallest
    return         b * row[0] + ((r <= g) ? // white
            ((r - b) * row[3] +             // yellow
             (g - r) * row[5]) :            // green
            ((g - b) * row[3] +             // yellow
             (r - g) * row[4]));            // red
}
\end{c++code}


The last remaining function is \Verb|evalSpectrum()| which relies on the seven
basis functions being piecewise constant


\begin{c++code}
float evalSpectrum(float3 rgb, float wavelength)
{
    return evalBin(rgb[0], rgb[1], rgb[2], bin(wavelength));
}
\end{c++code}
