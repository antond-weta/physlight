% SPDX-License-Identifier: Apache-2.0
% Copyright (c) Contributors to the PhysLight Project.

\section{Describing light sources}\label{ch:specification}

Light sources are modeled as emitters characterized by spectral distribution
and angular profile, and further modified by a ``filter'' or ``gel'' (called
\textsl{tint function} in this document). We want to derive a formulation for
spectral exitant radiance $L_{\lambda}^\uparrow(x,\omega,\lambda)$ given
specific light properties. We will do that by means of an \textsl{intrinsic
emission function} $\overline{L}^\uparrow_{\lambda}(x,\omega,\lambda)$
describing the directional and spectral properties of the light source
(inclusive of the tint) and scale this function as needed to obtain exitant
radiance:
\begin{equation}
L_{\lambda}^\uparrow(x, \omega, \lambda) = k_e \cdot
\overline{L}^\uparrow_{\lambda}(x, \omega, \lambda)
\end{equation}
where $k_e$ is a scaling factor called the \textsl{emission constant}.

%\paragraph{Emission constant}

The value of the emission constant is derived \emph{a posteriori} from the
source's power once the emission profile, the angular distribution and the tint
function are fixed, as opposed to a more traditional specification in which it
would be the emission constant to be specified directly as a light parameter and
the power computed from the other quantities.

We propose using luminous power for finite
sources (such as area lights) and illuminance for non-finite sources (such as
\gls{IBL} sources) to specify source brightness and to consider these to be
quantities inclusive of the effects of the tint function.

The rationale behind this choice is that it is desirable to setup lighting
on a scene in terms of perceived brightness of the final rendered image,
abstracting away concerns with such matters as specific spectral or angular
profiles of emitters, in an effort to try and focus attention towards the broad
brightness ratios of a scene.
For example, this arrangement minimizes changes in perceived scene brightness
(and more so brightness ratios) in the event of substitution of a tint function
with another.

\paragraph{Texture maps}
It is common to use texture maps to define some of the emission distribution
functions. A texture map can be seen as a function $M(x,\lambda)$ that specifies
a spatial distribution of color values, the function is dependent on pixel
position $x$ and wavelength $\lambda$.
For later calculations it is useful to separate the spectral dependency from the
spatial dependency. For example, in the case of an \gls{RGB} map, a possibility
could be to use a three-primary decomposition to separate pixel position from
color values:
\begin{equation}
M(x,\lambda) = M_r(x) r(\lambda) + M_g(x) g(\lambda) + M_b(x) b(\lambda)
\end{equation}
where $r(\lambda)$, $g(\lambda)$ and $b(\lambda)$ are the hypothetical spectral
distributions of the map's primaries for red, green and blue and $M_r(p)$,
$M_g(p)$ and $M_b(p)$ are the corresponding values stored in the map for a
given pixel position $x$.

It will be convenient in the upcoming sections to make use of a different
notation: we call $M_{col}(x): X\to\R^n$ the function that maps the position $x$
to its coordinates in a specified $n$ dimensional color space $col$. For example
for the \gls{RGB} color space we would have $n = 3$:
\begin{equation}
M_{rgb}(p) = \big(M_r(p), M_g(p), M_b(p)\big) \in \R^3
\end{equation}
but see \cref{ch:implementation} for an $n=7$ example. A similar thing
can be done for the primaries, joining them into a single function
$col(\lambda): \R^+\to\R^n$ which maps a wavelength to the response of the $n$
primaries at once. For \gls{RGB} we would have $n = 3$ and $rgb(\lambda)$ would be:
\begin{equation}
rgb(\lambda) = \big(r(\lambda), b(\lambda), g(\lambda)\big) \in\R^3
\end{equation}
This notation lets us write an equivalent, more compact definition for
$M(x,\lambda)$ in terms of the scalar product operation in $\R^n$
\begin{equation}
M(x,\lambda) = \big\langle M_{col}(x), col(\lambda) \big\rangle
\end{equation}
or for \gls{RGB} in $\R^3$
\begin{equation}
M(x,\lambda) = \big\langle M_{rgb}(x), rgb(\lambda) \big\rangle
\end{equation}

\Cref{ch:implementation} contains some further details on possible ways
to construct these spectral distributions for the primaries. The following
subsections will describe the functions needed to specify the different types of
lights and show how to calculate the emission constant from the specifications.

\subsection{Area Lights}

The emission for an area light source is specified by:
\begin{itemize}
\item $\Phi_v$ -- luminous power $[\unit{\lumen}]$
\item $A$ -- area $[\unit{\square\meter}]$
\item $\hat{L}(\lambda)$ -- spectral distribution
  \begin{itemize}\small\it
  \item Black body of given temperature $T$: $\hat{L}(\lambda) = B_T(\lambda)$
  \item Standard Illuminant D of given CCT $T$: eg. $\hat{L}(\lambda) = D_{65}(\lambda)$
  \item Standard Illuminant E: $\hat{L}(\lambda) = 1$
  \item Standard Illuminant $F_1$--$F_{12}$: eg. $\hat{L}(\lambda) = F_1(\lambda)$
  \item Tabulated spectral data from file
  \end{itemize}
\item $D(\omega)$ -- angular distribution
  \begin{itemize}\small\it
  \item Lambertian: $D(\omega) = 1$
  \item Powered cosine: $D(\omega) = \cos^n \theta = \langle \hat n,\omega \rangle^n$
  \item IES profile from file
  \end{itemize}
\item $T(x,\lambda)$ -- tint function
  \begin{itemize}\small\it
  \item Color: $T(x,\lambda) = \langle T_{col}, col(\lambda)\rangle$
  \item Pick from list of known gel sets (LEE, Rosco, Wratten, \ldots): $T(x, \lambda) = LEE_{013} (\lambda)$
  \item texture map: $T(x,\lambda) = \langle T_{col}(x), col(\lambda)\rangle$
  \end{itemize}
\end{itemize}

The exitant radiance of an area light source will be specified by a spectral
distribution $\hat{L}(\lambda)$, an angular distribution $D(\omega)$ that
describes the dependence of exitant radiance on outgoing direction and a tint
function $T(x,\lambda)$ that is used to scale exitant radiance depending on
position on the source and wavelength. This leads us to the following
formulation for the intrinsic emission function
\begin{equation}
\overline{L}^\uparrow_{\lambda}(x, \omega, \lambda) = T(x, \lambda)
\hat{L}(\lambda) D(\omega)
\qquad \left[\unit{\watt\per\square\meter\per\steradian\per\meter}\right]
\end{equation}

The luminous power of the source is obtained integrating over the area of the
light source $A$, the angular domain $\Omega$ (typically the front hemisphere
for planar, forward only sources, or an entire sphere for most other sources)
and the wavelength domain
\begin{equation}
\Phi_v = k_e K_{cd} \int_\lambda \int_\Omega \int_A
\overline{L}^\uparrow_{\lambda}(x, \omega, \lambda) \bar y (\lambda) \cos\theta
\d x \d\omega \d\lambda
\qquad [\unit{\lumen}]
\end{equation}
In our case $\Phi_v$ is given as a parameter of the light, and $k_e$ can be
determined by inversion:
\begin{equation}
k_e = \frac{\Phi_v}{K_{cd} \int_\lambda \int_\Omega \int_A T(x, \lambda) \cdot
\hat{L}(\lambda) \cdot D(\omega) \cdot \bar y (\lambda) \; \cos\theta \d x \d\omega
\d\lambda \qquad}
\end{equation}

Depending on the tint function and emission profile the triple integral for
$\Phi_v$ turns out to be separable, and it factorizes into a product of
integrals that can be precomputed easily.
Specifically in the common case in which the tint function is based on a texture
map, the triple integral for $\Phi_v$ separates into a product of integrals as
follows:

\begin{itemize}
\item a tint vector $\|T_{col}\|$
\begin{equation}
\|T_{col}\|  = \frac1A\int_A T_{col}(x) \d x \in \R^n
\end{equation}
\item a reduced luminance vector $\|col \cdot \hat{L}\|_{\bar y}$
\begin{equation}\label{eq:redlum_integral}
\|col \cdot \hat{L}\|_{\bar y} = \int_\lambda col(\lambda) \hat{L}(\lambda)
\bar{y}(\lambda) \d\lambda \in \R^n
\end{equation}
\item an angular norm $\|D\|$
\begin{equation}
 \|D\| = \int_\Omega D(\omega) \cos\theta \d\omega \in \R
\end{equation}
\end{itemize}
with these relations we can rewrite luminous power as
\begin{equation}
\Phi_v = k_e K_{cd} A \|D\| \big\langle \|T_{col}\|,\; \|col\cdot \hat{L}\|_{\bar
y} \big\rangle
\end{equation}
so that the expression for the emission constant is simply
\begin{equation}\label{eq:k_e_closedform}
 k_e = \frac{\Phi_v}{K_{cd} A \|D\| \big\langle \|T_{col}\|,\; \|col\cdot \hat{L}\|_{\bar y} \big\rangle }
\end{equation}

We observe that the tinting vector can be precomputed once for texture maps, the
angular norm has simple closed form solutions for the common cases (and can be
precomputed and stored for tabulated profiles), and the reduced luminous power
vector can be precomputed for each combination of texture map primaries and
spectral emission profiles.
This reduces the computation of $k_e$ to a scalar product and a few
multiplications, once the given inputs of a light are given.
\Cref{ch:implementation} contains tables for various common combinations.

Once the emission constant has been computed, exitant radiance from the source
is simply
\begin{equation}
L^\uparrow_{\lambda}(x, \omega, \lambda) = k_e T(x, \lambda) \hat{L}(\lambda)
D(\omega)
\qquad \left[\unit{\watt\per\square\meter\per\steradian\per\meter}\right]
\end{equation}

% TODO: add the corresponding expression in space col

\subsection{Image-Based Lighting}

The emission for an \gls{IBL} source is specified by (sublists contain possible examples):

\begin{itemize}
\item $E^+_v$ -- illuminance from top hemisphere $[\unit{\lux}]$
\item $\hat{L}(\omega,\lambda)$ -- spherical map of an image (spectral and angular distribution)
\item $T(\lambda)$ -- tint function
  \begin{itemize}\small\it
  \item Color: $T(\lambda) = \langle T_{col}, col(\lambda) \rangle$
  \item Pick from list of known gel sets (LEE, Rosco, Wratten, \ldots): $T(\lambda) = LEE_{013} (\lambda)$
  \end{itemize}
\end{itemize}
\Gls{IBL} sources model light coming from great distance away from the scene. As
such they illuminate the scene geometry in a manner that is dependent on
local orientation, but not position. For this reason it is only possible to
specify their brightness in terms of illuminance (area density of luminous power
at the receiver) rather than luminous power\footnote{Luminous power would be
ill-defined in this case: it's easily seen that an \gls{IBL} source emits in
such a way to have the same finite illuminance at all points in a scene. This
means that its luminous power, that is its illuminance integrated over the scene
area, depends on the amount of geometry in the scene, implying that doubling the
amount of geometry in a scene would double the power of the source, making
such a definition inconsistent}.
This means that for \gls{IBL} sources the formulation is in terms of incident
radiance at an unspecified position in the scene:
\begin{equation}
L_{\lambda}^\downarrow(\omega,\lambda) = k_e T(\lambda) \hat{L}(\omega,\lambda)
\end{equation}
the missing dependence on the position $x$ in the scene outlines how incident
radiance from an \gls{IBL} source is not dependent on the receiver's location
but only on its orientation. The emission constant is the estimated in terms of
illuminance $E^+_v$ from the top hemisphere
\begin{align*}
E^+_v &= K_{cd} \int_{\Omega^+} \int_\lambda L_{\lambda}^\downarrow(\omega,\lambda)
\cdot\bar y(\lambda) \cos\theta \d\omega\d\lambda \\
&= k_e K_{cd}\int_{\Omega^+}\int_{\lambda} T(\lambda) \cdot \hat{L}(\omega,\lambda)
\cdot\bar y(\lambda) \cos\theta \d\omega\d\lambda
\qquad [\unit{\lux}]
\end{align*}
where $\Omega^+$ indicates integration over the top hemisphere, intuitively
modeling the idea of a receiver (such as a grey card or a hemispherical light
meter) set horizontally and facing upwards. This gives us:
\begin{equation}
k_e = \frac{E^+_v}{K_{cd}\int_{\Omega^+}\int_{\lambda} T(\lambda) \cdot
\hat{L}(\omega,\lambda) \cdot \bar y(\lambda)\cos\theta \d\omega\d\lambda}
\end{equation}

The double integral in the denominator can again be factorized into simpler
computations when using an \gls{IBL} texture map in a color space $col(\lambda)$.
This time the spatial component is the directional component of $\hat{L}$:
\begin{equation}
\hat{L}(\omega, \lambda) = \big\langle \hat{L}_{col}(\omega),\; col(\lambda) \big\rangle
\end{equation}

This gives us the possibility to break down the integral into
\begin{itemize}
\item an irradiance coloration vector $\|\hat{L}_{col}\|$
\begin{equation}
\|\hat{L}_{col}\| = \int_{\Omega^+} \hat{L}_{col}(\omega)\cos\theta \d\omega \in \R^n
\end{equation}
\item a tint vector $\|T \cdot col\|_{\bar y}$
\begin{equation}
\|T \cdot col\|_{\bar y} = \int_\lambda T(\lambda)\cdot col(\lambda)\,\bar y(\lambda) \d\lambda \in \R^n
\end{equation}
\end{itemize}
where the tint function itself can be specified as a vector in some color space
as outlined above:
\begin{equation}
T(\lambda) = \langle T_{col},\; col(\lambda)\rangle
\end{equation}
Note that if the map and the tint are specified in color spaces $col_1$ and
$col_2$, the full expression for the tint vector is
\begin{equation}
\|T \cdot col\|_{\bar y} = \int_\lambda T_{col_2}col_2(\lambda)^t col_1(\lambda)\,\bar y(\lambda) \d\lambda
 = T_{col_2} \|col_2 \otimes col_1\|_{\bar y}
 \in \R^n
\end{equation}
where $\|col_2 \otimes col_1\|_{\bar y}$ is the $\bar y$-norm of the outer
product of the color spaces $col_1$ and $col_2$.

This decomposition lets us compute the emission constant for \gls{IBL} sources
as follows
\begin{equation}
k_e = \frac{E^+_v} {K_{cd} \langle \|\hat{L}_{col}\|, \|T \cdot col\|_{\bar y}
\rangle}
\end{equation}

Once the emission constant has been computed, exitant radiance from the source
is simply
\begin{equation}
L^\uparrow_{\lambda}(x, \omega, \lambda) = k_e  T(\lambda)
\hat{L}(\omega,\lambda)
\qquad \left[\unit{\watt\per\square\meter\per\steradian\per\meter}\right]
\end{equation}

% TODO: add the corresponding expression in space col

\subsection{Sun light}

The emission for a Sun light is specified by (sublists contain possible examples):

\begin{itemize}
\item $E^+_v$ -- illuminance of the sun $[\unit{\lux}]$
\item $D_{\omega_c}(\omega)$ -- angular dependent function describing the sun with the center at $\omega_c$
	\begin{itemize} \small\it
	\item Disk of angular size $\alpha$
	\item Gaussian falloff function around $\omega_c$
	\end{itemize}
\item $\hat{L}(\lambda)$ -- emitter (spectral distribution)
  \begin{itemize}\small\it
  \item Black body of given temperature $T$: $\hat L(\lambda) = B_T(\lambda)$
  \item Standard Illuminant D of nominal temperature $T$: eg. $\hat L(\lambda) = D_{65}(\lambda)$
  \item Standard Illuminant E: $\hat L(\lambda) = 1$
  \item Standard Illuminant $F_1$--$F_{12}$: eg. $\hat L(\lambda) = F_1(\lambda)$
  \item Tabulated spectral data from file
  \end{itemize}
\item $T(\lambda)$ -- tint function
  \begin{itemize}\small\it
  \item Color: $T(\lambda) = \langle T_{col}, col(\lambda) \rangle$
  \item Pick from list of known gel sets (LEE, Rosco, Wratten, \ldots): $T(\lambda) = LEE_{013} (\lambda)$
  \end{itemize}
\end{itemize}

Sun lights, like \glspl{IBL}, model a light source located at a great distance from
a scene. Again the irradiance is dependent on the incident angle, but not the
position in the scene.
\begin{equation}
L^\downarrow_{\lambda}(\omega,\lambda) = k_e\cdot T(\lambda)\cdot
D_{\omega_c}(\omega)\cdot \hat{L}(\lambda)
\end{equation}
This time no function is dependent on more than one variable, so a factorization is easily done.
To compute the emission constant we again use the definition of illuminance
$E^+_v$
\begin{equation}
E^+_v = K_{cd} \int_\lambda \int_{\Omega^+} L^\downarrow_{\lambda}(\omega,\lambda) \cdot\bar y(\lambda)\cos\theta\d\omega \d\lambda
\end{equation}
which gives us:
\begin{equation}
k_e = \frac{E^+_v}{K_{cd}\int_{\Omega^+}\int_{\lambda} T(\lambda)\cdot D_{\omega_c}(\omega)\cdot \hat{L}(\lambda) \cdot \bar y(\lambda)\cos\theta\d\omega \d\lambda}
\end{equation}
\\
If $T(\lambda)$ is defined in some color space $col$, we can break the down the double integral into:
\begin{itemize}
\item an angular norm $\|D_{\omega_c}\|$
\begin{equation}
\|D_{\omega_c}\| = \int_\Omega D_{\omega_c}(\omega)\cos\theta\d\omega
\end{equation}
\item a reduced illuminance vector $\|col \cdot \hat{L}\|_{\bar{y}}$
\begin{equation}
\|col \cdot \hat{L}\|_{\bar{y}} = \int_\lambda col(\lambda)\cdot \hat{L}(\lambda)\cdot\bar{y}(\lambda)\d\lambda
\end{equation}
\end{itemize}
This gives us the emission constant $k_e$ as
\begin{equation}
k_e = \frac{E^+_v}{K_{cd}\cdot \|D_{\omega_c}\|\cdot \big\langle T_{col}, \; \|col \cdot \hat{L}\|_{\bar{y}} \big\rangle}
\end{equation}

Once the emission constant has been computed, exitant radiance from the source
is simply
\begin{equation}
L^\uparrow_{\lambda}(x, \omega, \lambda) = k_e  T(\lambda)
D_{\omega_c}(\omega) \hat{L}(\lambda)
\qquad \left[\unit{\watt\per\square\meter\per\steradian\per\meter}\right]
\end{equation}

% TODO: add the corresponding expression in space col
