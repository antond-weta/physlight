% SPDX-License-Identifier: Apache-2.0
% Copyright (c) Contributors to the PhysLight Project.

\chapter{Introduction}

\section{Why physical units?}


One of the reasons physically-based image synthesis has become such an
indispensable ingredient for pushing realism and consistency in very
complex scenarios is predictability. Material modelling is based on
physical properties and thus allows to reproduce increasingly complex
natural phenomena. In traditional rendering systems, control of the
look was often owned by the shading system, often allowing artistic
control knobs for artificial emission or non-physical behavior to
compensate for lighting, for example missing indirect illumination.

The physically-based rendering paradigm dictates that materials,
lighting and camera response are completely independent entities and
must hence always be controlled with this separation in mind. This
on one hand makes digital assets more universally usable, and on the
other hand allows to achieve consistency when integrating digital with
live action footage.

Naturally, since a wide range of different effects interplay with each
other in complex and subtle ways, it is absolutely crucial that all
components are accurately represented by their digital analogues to be
able to reproduce real-world footage realistically. Furthermore, it is
very important that the description of the digital scene is based on
real-world quantities, which enables easier control and verification.

In this document we focus on the specification of illuminants and
sensors and advocate using the same conventions and physical units as
lighters and photographers on set. For example, it is preferable in
terms of predictability to think about brightness in terms of
photometric units, i.e., units which represent perceived brightness as
seen by a human observer, rather than radiometric power (wattage)
which depending on the spectral emission profile can have highly
varying brightness. We specify a system of units and models, such that
provided the digital scene is modelled after a real-world counterpart,
specifying the same illuminants via physical units and using the same
camera settings the renderer will replicate the same pixel values as a
real camera. Naturally, for reproducing color, we will not only need
to develop an accurate understanding of brightness but also be aware
of how spectral data flows through the pipeline.


In short, this document introduces the use of physical units in the
Weta pipeline for the treatment of lighting and imaging.  The
objective is to formalize a system of physical absolute units to be used
when describing lighting, materials and imaging across Weta's
processes of digital image synthesis. This will allow us to bring them in line
with real life measurements in order to facilitate integration of
live action footage and digitally created content. The specification of
how units from the \gls{si} are injected in the pipeline and the
mathematics that corresponds to the quantities described.


This document is divided in parts: \cref{part:models}
contains an introduction on basic quantities that are relevant in
light transport simulation and review the relation of photometric and
radiometric units and describes the modelling aspects. In 
\cref{ch:lighting} we describe imaging and light models and propose the
relevant sets of parameters to specify them accurately.  
\Cref{ch:calcsheets} provides some examples on how to derive some of
the described quantities from real world examples.

\Cref{part:ref} provides a reference of various concrete types of
illuminants (\cref{ch:illuminants}) and cameras (\cref{ch:sensors})
as well as a reference implementation (\cref{ch:implementation}).

\Cref{part:app} contains the appendices: \cref{ch:color} discusses
various considerations on color in the tristimulus versus spectral setting,
\cref{ch:implementation} describes an implementation of a color manipulation library
and \cref{ch:radiance} is an extensive excerpt from Frederic Nicodemus's 
excellent 1963 paper~\cite{nicodemus63} in which the concept of \textsl{radiance}
is introduced: unsurprisingly we found his explanation far clearer than what we 
had been able to come up with, so we decided to retype his text, and limit ourselves
to adjust his notation to contemporary conventions.

\section{Pipeline integration}

As a digital movie making pipeline is comprised of many parts, from content creation, to
scene assembly and asset management, there is a need to
be certain that all interpretations of values match across the various components.

Components that will likely need adaptation or revision are:

\begin{itemize}
\item Host applications (such as Maya, Motion Builder, Katana)
\item Scene translation bridges
\item Light management systems
\item Shader libraries
\item Renderers
\end{itemize}
this is a fairly direct consequence of common custom so far. Until this point, it has been
fairly common to use physical units in simulation systems, but not for lighting descriptions.
Much like cameras have their focal length specified in millimiters in the user interface,
similar adaptations will be needed for various other quantities, such as film back, light 
sensitivity and so on. Similarly, as the renderers will be adapted to receive data
referred to physical units, the bridges would need to make sure they are capable of transporting
and translating such data appropriately, and various shading libraries might need various levels
of adaptation for correctness. These changes are discussed in detail in the rest of this document:
while \cref{ch:imaging} focuses on the theoretical background for these revisions,
\cref{ch:implementation} has code reference for the specifics of a working implementation.

\paragraph{Scene representation}

Scene representations will now need bindings from scene units to \gls{si} units,
so the renderer is aware of the physical dimensions of the entities involved.

For example, in \gls{RIB} this might look like this:

\begin{ribcode}
Option "units" "float length" [0.01]    # scene unit is 1 cm
               "float time"   [0.04166] # 24 fps
\end{ribcode}
where \Verb/units:length/ is the size of a scene unit in meters,
and \Verb/units:time/ is the size of a time unit in seconds.

\section{Quantities in light transport}

A common way of reasoning about image generation is built on the notion of a
light ray, which is the fundamental element of geometrical optics. Light rays
are straight line segments through space along which light travels in the scene.
Thinking about photons, light rays represent paths along which photons flow,
each of them carrying a quantity of energy dependent on its wavelength $\lambda$
and equal to $E = hc/\lambda$ (where $h$ is Planck's constant and $c$ the speed
of light in vacuum).

As much as photons are the elements of light transport, at first we only have an
aggregate measure of their condition in terms of power of a source.
The intention here is to break down the aggregate notion of power of a source
into components so that we can understand what quantity flows through a ``ray of
light''.
For this reason, we start from the amount of power emitted by our source $S$:
the \textsl{emissive radiant power}.
It can be seen as the energy flux through a sphere fully enclosing $S$.
More precisely, we define the \textsl{emissive radiant flux} $\Phi_e$ of a
source $S$ through a given surface $A$ as the amount of energy from $S$ that
crosses $A$ per unit time. This flux will be called \textsl{total} when the surface
$A$ fully encloses the source $S$. Total flux will be called \textsl{power} in
this
document, radiant flux is measured in watts $[\si{\watt}]$.

It is natural to think of radiant power as originating from a region $\delta S$ on our source:
for instance a source could emit different energy over its surface. This leads to an
intuitive notion of radiant power from a point $x$ on the surface of the source
$M_e(x)$, which
we would like defined in such a way that power originated from our source $S$
can be computed
directly by integration over its surface:
\begin{displaymath}
\Phi_e = \int_{S} M_e(x) \d x \qquad [\si{\watt}]
\end{displaymath}
this quantity $M_e$ is called \textsl{radiosity} and represent the power density
of our
source with respect to its surface area, formally
\begin{displaymath}
M_e = \frac{\d\Phi_e}{\d S} \qquad \left[\si{\watt\per\square\meter}\right]
\end{displaymath}
It is immediately verified that for a uniformly emissive source the
radiosity is simply the ratio of emissive power to area
$M_e = \Phi_e / S$.

In a similar way, one could consider a notion of power leaving the source $S$ in
a given
direction $\omega$.
As before we would like this quantity defined so that the power in a range of
directions
$\Omega$ could be directly computed by integration:
\begin{displaymath}
\Phi_e = \int_\Omega I_e(\omega) \d\omega \qquad [\si{\watt}]
\end{displaymath}
the integrand $I_e(\omega)$ is called \textsl{radiant intensity} and represent
the density
of power with respect to directions from the source, formally
\begin{displaymath}
I_e = \frac{\d\Phi_e}{\d\Omega} \qquad \left[\si{\watt\per\steradian}\right]
\end{displaymath}
it is very important to note that radiant intensity should be considered a
``far-field'' quantity, that is to say, a quantity that is only valid when
the source is extremely small in comparison to the other dimensions at play.
The interested reader is invited to consult~\cite{nicodemus63}, a relevant
extract of which is included in \cref{ch:radiance} of this document.

Now that we have a notion of power \textit{from a region} and power \textit{in
a given direction}, we can combine the two to get power \textit{from a region
in a given direction}, which as before is a quantity that lets us compute
power by integration in area and direction:
\begin{displaymath}
\Phi_e = \int_{S} \int_\Omega L_e^\uparrow(x, \omega) \;\cos\theta\d x\d\omega
\qquad [\si{\watt}]
\end{displaymath}
the integrand $L_e^\uparrow(x, \omega)$ is called \textsl{exitant radiance}
and represents the power density with respect to area and directions from
the source, formally
\begin{displaymath}
L_e^\uparrow = \frac{\d\Phi_e}{\cos\theta \d S \d\Omega}
\qquad \left[\si{\watt\per\square\meter\per\steradian}\right]
\end{displaymath}
in some cases it might be convenient to distinguish exitant radiance from
incident radiance and this will be indicated with an arrow in the exponent:
exitant radiance pointing up $L_e^\uparrow$ and incident radiance pointing down
$L_e^\downarrow$.

Here, $\theta$ denotes the angle between the direction $\omega$ and the normal.
The so called \emph{projected solid angle} $\d\omega^\bot := \cos \theta \d \omega$
accounts for the change of flux received per infinitesimal area by a beam of light
according to its orientation towards the surface.


%  \begin{figure}[t]
%  {
%  \centering
%  \begin{minipage}[b]{.3\textwidth}
%  	\centering
%  	\begin{tikzpicture}[line cap=round,line join=round,>=angle 60,x=1.0cm,y=1.0cm, baseline=0]
\draw [domain=0:4] plot(\x,{(--0-0*\x)/1});
\draw [->] (1.78,0) -- (3.71,0.71);
\draw [->] (1.56,0) -- (3.06,1.4);
\draw [->] (1.78,0) -- (2.81,1.78);
\draw [->] (1.28,0) -- (0.86,2.01);
\draw [->] (1.28,0) -- (-0.3,1.31);
\draw [->] (1,0) -- (-0.3,0.69);
\draw [->] (1.28,0) -- (0.1,1.68);
\draw [->] (1.56,0) -- (1.13,2.01);
\draw [->] (1.56,0) -- (0.38,1.68);
\draw [->] (1.56,0) -- (2.59,1.78);
\draw [->] (1.56,0) -- (1.91,2.02);
\draw [->] (1,0) -- (2.5,1.4);
\draw [->] (1,0) -- (1.36,2.02);
\draw [->] (2,0) -- (3.5,1.4);
\begin{scriptsize}
\draw [color=black] (2,0)-- ++(-1.5pt,0 pt) -- ++(3.0pt,0 pt) ++(-1.5pt,-1.5pt) -- ++(0 pt,3.0pt);
\draw [color=black] (1,0)-- ++(-1.5pt,0 pt) -- ++(3.0pt,0 pt) ++(-1.5pt,-1.5pt) -- ++(0 pt,3.0pt);
\draw[color=black] (1.6,-0.2) node {$\delta S$};
\draw (3.27,1) node[anchor=north west] {$\delta$$\Omega$};
\draw[color=black] (0,2) node {$a)$};
\end{scriptsize}
\end{tikzpicture}

%  	\label{fig:P_e}
%  \end{minipage}
%  \begin{minipage}[b]{.3\textwidth}
%  	\centering
%  	% SPDX-License-Identifier: Apache-2.0
% Copyright (c) Contributors to the PhysLight Project.
\begin{tikzpicture}[line cap=round,line join=round,>=angle 60,x=1.0cm,y=1.0cm, baseline=0]
\draw [domain=0:4] plot(\x,{(--0-0*\x)/1});
\draw [->] (2,0) -- (3.93,0.71);
\draw [->] (2,0) -- (3.5,1.4);
\draw [->] (2,0) -- (3.03,1.78);
\draw [->] (2,0) -- (2.36,2.02);
\draw [->] (2,0) -- (1.58,2.01);
\draw [->] (2,0) -- (0.82,1.68);
\draw [->] (2,0) -- (0.42,1.31);
\draw [->] (2,0) -- (0.07,0.69);
\begin{scriptsize}
\draw (3.27,0.96) node[anchor=north west] {$\delta$$\Omega$};
\draw[color=black] (2,-0.2) node {$x$};
\draw[color=black] (0,2) node {$b)$};
\end{scriptsize}
\end{tikzpicture}

%  	\label{fig:M_e}
%  \end{minipage}
%  \begin{minipage}[b]{.3\textwidth}
%  	\centering
%  	\begin{tikzpicture}[line cap=round,line join=round,>=angle 60,x=1.0cm,y=1.0cm, baseline=0]
\draw [->] (1.78,0) -- (3.28,1.4);
\draw [->] (1.28,0) -- (2.78,1.4);
\draw [->] (1.28,0) -- (2.78,1.4);
\draw [->] (1.56,0) -- (3.06,1.4);
\draw [->] (2,0) -- (3.5,1.4);
\draw [->] (1,0) -- (2.5,1.4);
\draw [domain=0:4] plot(\x,{(--0-0*\x)/1});
\begin{scriptsize}
\draw [color=black] (2,0)-- ++(-1.5pt,0 pt) -- ++(3.0pt,0 pt) ++(-1.5pt,-1.5pt) -- ++(0 pt,3.0pt);
\draw [color=black] (1,0)-- ++(-1.5pt,0 pt) -- ++(3.0pt,0 pt) ++(-1.5pt,-1.5pt) -- ++(0 pt,3.0pt);
\draw[color=black] (1.6,-0.2) node {$\delta S$};
\draw (3,0.96) node[anchor=north west] {$\omega$};
\draw[color=black] (0,2) node {$c)$};
\end{scriptsize}
\end{tikzpicture}

%  	\label{fig:I_e}
%  \end{minipage}
%  \caption{Visualization of different light transport quantities}
%  }
%  \vskip 1mm
%  {\footnotesize
%  \begin{inconstruction}[This figure is under construction]
%  \itshape One ray is representing the amount of energy originating at one point
%  and travelling into one direction. Figure $a)$ shows a visualization of the
%  power $\Phi_e$ from a surface $\delta S$. Power is the total energy leaving the
%  source into all directions. Figure $b)$ is a visualization of $M_e$: the energy
%  leaving the source from a specified point $x$ and into all directions.
%  Whereas figure $c)$ visualizes $I_e$: the energy flowing into one specific
%  direction $\omega$ from a surface patch $\delta S$.
%  One ray representing $L_e$, the relations to the other quantities are visualized
%  clearly: $M_e$ is the integration over all directions $\Omega$, $I_e$ is the
%  integral over a surface $S$, but in one specified direction and $\Phi_e$ is
%  the combination of both.
%  \end{inconstruction}
%  }
%  \end{figure}
%  
%  \begin{figure}[htb]
%      \centering
%      \def\svgwidth{0.9\linewidth}
%      \import{figures/}{flux_etc.pdf_tex}
%      \caption{\label{fig:flux_etc}%
%      A sensor for power detects photons across an area regardless of
%      their direction of travel (top left). Restricting the directions to a
%      small cone produces a sensor for intensity (top right). Similarly, 
%      restricting the area to a point yields a sensor for irradiance (bottom left).
%      Radiance is measured by restricting both the sensor area and the incident
%      directions (bottom right). }
%  \end{figure}

\begin{figure}[tb]
    \centering
    \def\svgwidth{\linewidth}
    \import{figures/}{flux_etc.pdf_tex}
    \caption{\label{fig:flux_etc}%
    A visualization of power (top left), intensity (top right), irradiance (bottom left), and radiance (bottom right).
    Power $\Phi_e$ is the total energy per unit time incident everywhere on the area $S$ from all directions $\Omega$. 
    Intensity $I_e(\omega)$ is the energy incident from a specific direction $\omega$ everywhere on $S$, while irradiance $M_e(x)$ is the energy incident at a given point $x$ from all direction $\Omega$.
    Intensity can be obtained by integrating radiance $L_e(x,\omega)$ over $S$, while irradiance can be obtained by integrating radiance over the set of
    all directions $\Omega$. Integrating $L_e$ over both the surface and all directions yields power. }
\end{figure}

One can think of incident radiance as photons along a ray reaching a surface.
This also can vary across a region so it is natural to formulate a notion
of incident power density across an area on the receiver.
The total incident flux of a region $R$ on the receiver can then be computed by
integration over its surface:
\begin{displaymath}
\Phi_e = \int_{R} E_e(x) \d x \qquad [\si{\watt}]
\end{displaymath}
the value $E_e(x)$ is the receiver's area density of incident power and
is called \textit{irradiance}.
Much like radiosity, irradiance as well has a differential formulation:
\begin{displaymath}
E_e = \frac{\d\Phi_e}{\d R} \qquad \left[\si{\watt\per\square\meter}\right]
\end{displaymath}

Comparing all the relations above, we can express radiosity, radiant intensity
and irradiance in terms of radiance:
\begin{displaymath}
M_e(x) = \int_\Omega L_e^\uparrow(x,\omega) \;\cos\theta \d\omega \qquad
I_e(\omega) = \int_S L_e^\uparrow(x,\omega) \d x \qquad
E_e(x) = \int_\Omega L_e^\downarrow(x,\omega) \;\cos\theta \d\omega
\end{displaymath}

Somewhat orthogonally to all the above, we can consider spectral distributions
of all these quantities, which we simply define as the spectral densities of the
quantities above:
\begin{displaymath}
\Phi_{\lambda} = \frac{\d\Phi_e}{\d\lambda} \quad
M_{\lambda} = \frac{\d\Phi_{\lambda}}{\d S} \quad
I_{\lambda} = \frac{\d\Phi_{\lambda}}{\d\Omega} \quad
E_{\lambda} = \frac{\d\Phi_{\lambda}}{\d R} \quad
L_{\lambda}^\uparrow = \frac{\d\Phi_{\lambda}}{\d S \d\Omega} \quad
L_{\lambda}^\downarrow = \frac{\d\Phi_{\lambda}}{\d R \d\Omega}
\end{displaymath}



\section{Measuring light and brightness}

\begin{table}
%\small
\fontsize{9}{10.8}\selectfont
{
\centering
% add some space inbetween lines
\renewcommand{\arraystretch}{1.1}% only valid inside for this table due to scoping
% reduce spacing between columns
\renewcommand{\tabcolsep}{.125em}
\begin{tabular}{l c c c |@{$\;$} l c c c}
\multicolumn{4}{c|@{$\;$}}{Radiometric spectral} & \multicolumn{4}{c}{Photometric} \\
  name	             & \multicolumn{2}{c}{unit} & symbol
& name               & \multicolumn{2}{c}{unit} & symbol \\ \hline
%
  Radiance           &                  &
\si{\watt\per\square\meter\per\steradian\per\meter} & $L_{\lambda}$
& Luminance          & \textsl{nit}     & $\si{\nit} =
\si{\lumen\per\square\meter\per\steradian}$          & $L_v$ \\
%
  Irradiance         &                  & \si{\watt\per\square\meter\per\meter}               & $E_{\lambda}$
& Illuminance        & \textsl{lux}     & $\si{\lux} = \si{\lumen\per\square\meter}$          & $E_v$  \\
%
  Radiosity          &        & \si{\watt\per\square\meter\per\meter}
   & $J_{\lambda}$
& Luminosity         & \textsl{lux}     & $\si{\lux} = \si{\lumen\per\square\meter}$          & $J_v$ \\
%
  Radiant emittance  &                  & \si{\watt\per\square\meter\per\meter}
             & $M_{\lambda}$
& Luminous emittance & \textsl{lux}     & $\si{\lux} = \si{\lumen\per\square\meter}$          & $M_v$ \\
%
  Radiant intensity  &                  & \si{\watt\per\steradian\per\meter}
             & $I_{\lambda}$
& Luminous intensity & \textsl{candela} & $\si{\candela} = \si{\lumen\per\steradian}$         & $I_v$ \\
%
  Radiant power      & \textsl{watt}    & \si{\watt\per\meter}                                & $\Phi_{\lambda}$
& Luminous power     & \textsl{lumen}   & \si{\lumen}                                         & $\Phi_v$ \\
%
  Radiant energy     & \textsl{joule}   & $\si{\joule\per\meter} =
\si{\watt\second\per\meter}$ & $Q_{\lambda}$
& Luminous energy    & \textsl{talbot}  & $\si{\talbot} = \si{\lumen\second}$
               & $Q_v$
\end{tabular}\\[2mm]
\begin{tabular}{l c c}
Throughput (\'etendue) & \si{\square\meter\steradian} & $G$ \\
Reflectance            & & $\rho$ \\
Transmittance          & & $\tau$ \\
\end{tabular}
\caption{Correspondance between radiometric and photometric units \label{tab:radiophoto}}

}
\vskip 1mm
{\footnotesize\it We abbreviate the unit for luminous energy \textit{talbot} as
\si{\talbot} instead of the also common \si{\tesla} to avoid confusion with the
unit for magnetic flux \textit{tesla}.
We also use the convention of subscripting photometric quantities
with $v$ (for \textit{visual}), radiometric quantities with $e$ (for
\textit{energetic}) and spectral radiometric quantities with $\lambda$;
this follows the recommendations in documents such as~\cite{united1967usa}.}
\end{table}

One of the central points of this document is to put units to the pixel
values in a raster image. 
For the purpose of this intoductory section, we will focus on the generation of grayscale images, where each pixel records the \textsl{\gls{brightness}} of a scene at its respective location. In particular, and throughout this whole document, we will use the term \textsl{brightness} to indicate a non-specific, informal, plain-English, everyday meaning that a quantity may feel more or less ``dim'' to a human observer. 
The discussion will then proceed to specify the exact meaning in question more precisely and give it an appropriate name.

As we have just seen, radiometric quantities can be used to measure brightness in terms of \emph{radiative energy}.
However, if our image was generated using values proportional to this quantity,
we would obtain an image that will likely not correspond well to the subjective sense of brightness in the scene as experienced by a person looking at it. 
This is because the human visual system is not equally sensitive to all wavelengths
of light, and may perceive light emanating or reflecting off two objects with different total radiative energy as having the same brightness, as a consequence of differences in the spectral distributions only. 
Conversely, spectral distributions carrying the same amount
of total radiative energy may be perceived as having different brightness levels.

It is therefore useful to reason about the energy of light in two
different ways, one dealing with energy, called \textsl{radiometric}, and the
other tuned to human perception, called \textsl{photometric}. The two approaches
have historically been using a great number of units, a few of which were
included in the \gls{si} at the Eleventh Conference on Weights and Measures
held in 1960.  Many details around various historical and \gls{si} units are
described in~\cite{Meyer-Arendt:68}.

In the \textsl{radiometric} point of view, the \textsl{radiant power} of a
source $S$
(also called \textsl{total radiant flux}) is measured in
\textit{watts} $[\si{\watt}]$ and is the total amount of energy per unit time
emitted by $S$.
%This means that two different light sources emitting the same radiant power and
%having different spectral distributions are likely to appear as having different
%brightness.

On the other hand, in the \textsl{photometric} point of view, the
\textsl{luminous power}
of a source $S$ (called \textsl{luminous flux}) is measured in
\textit{lumens} $[\si{\lumen}]$, and is the amount of energy per unit time
weighted by the \textsl{photopic luminous efficiency function} $\bar y(\lambda)$. 
This function was designed to model the response of the human visual system as a function of wavelength $\lambda$\footnote{
	The photopic luminosity function $\bar y(\lambda)$ is the result of a series of experiments and tabulations first published by the \gls{CIE} in 1924 (the function was called $V(\lambda)$ at the time) and then included in the color matching functions for the \emph{standard 2 degree colorimetric observer}~\citep{cie1931}.}.

This means that two different light sources emitting the same luminous
power will appear as having about the same brightness even when having
different spectral distributions\footnote{This
statement can only be true approximately, as the specific spectral response
of different humans exhibits a certain amount of variation, even among
normal subjects.
Also there are subjects having various different kinds of anomalies in their
visual systems (such as color blindness), for whom this statement ends up being even less accurate}.

More formally, a photometric quantity $X_v(\ldots)$ is obtained from 
the corresponding radiometric spectral quantity $X_{\lambda}(\ldots, \lambda)$
integrating
\begin{equation}
X_v(\ldots) = K_m \int X_{\lambda}(\ldots,\lambda)\; \bar y (\lambda) \d\lambda.
\label{eqn:rad_to_photo}
\end{equation}
For example, given spectral radiant power $\Phi_{\lambda}(\lambda)$, the
corresponding
luminous power $\Phi_v$ is
\begin{displaymath}
\Phi_v = K_m \int \Phi_{\lambda}(\lambda)\; \bar y(\lambda) \d\lambda.
\end{displaymath}

The scaling constant $K_m$ converts watts to lumens. It can be determined
from the 
\gls{si} unit \textsl{candela} $[\si{\candela}]$, which was adopted at
the General Conference on Weights and Measures in 1979 and
there formally defined as follows:
\begin{quote}
\emph{One candela is the amount of luminous intensity of a source of
monochromatic radiation of frequency \SI{540}{\tera\hertz} and
a radiant intensity of \SI{1/683}{\watt\per\steradian}.}
\end{quote}

$\SI{540}{\tera\hertz}$ correspond to about \SI{555.016}{\nano\meter} in
standard air, a wavelength in the greens to which the eye of the standard observer
is most sensitive.
The radiant intensity of $\SI{1/683}{\watt\per\steradian}$ was chosen to make the candela about equal to the
\textit{standard candle}, the unit which it superseded. 

Note that the above definition is an updated version of the original definition, which had been
valid since 1933 and was based on the emission of a blackbody radiator
at the freezing point of platinum, about \SI{2042}{\kelvin}. This 
historical definition is sometimes also found in literature~\citep{Meyer-Arendt:68}.

Using Eqn.~\eqref{eqn:rad_to_photo} and the Dirac delta distribution $\delta$, the
above definition can be written as
\begin{displaymath}
    1 \si{\candela} = 1\frac{\si{\lumen}}{\si{\steradian}} 
    = K_m \int \frac{\delta(\lambda-555.016\si{\nano\meter})}{683} \si{\watt\per\steradian}\;\bar y(\lambda) \d\lambda
    = K_m \cdot \frac{\bar y(555.016\si{\nano\meter})}{683}\si{\watt\per\steradian},
\end{displaymath}
from which it follows that $K_m \approx 683.002 \;\si{\lumen\per\watt}$. In 
practice, however, it is safe to use the value of $683\;\si{\lumen\per\watt}$~\citep{cie1996,wyszecki2000}.

\Cref{tab:radiophoto} lists important radiometric quantities and their photometric equivalents.

