% SPDX-License-Identifier: Apache-2.0
% Copyright (c) Contributors to the PhysLight Project.

\chapter{Imaging and lighting models}\label{ch:lighting}
\section{Describing imaging}\label{ch:imaging}

Imaging is the process of transforming the radiant power of a specific scene
configuration into pixel values. This is done by creating a camera and sensor
or film model and computing their response to the energy flow in the scene.
The imaging parameters are:
\begin{itemize}
\item $f$ -- focal length $[\si{\milli\meter}]$
\item $N$ -- aperture number $[\si{\fnumber}]$  (artificially valid for
non-DoF renders)
\item $o$ -- focus distance $[\si{\meter}]$
\item $\Delta t$ -- exposure time $[\si{\second}]$
\item $S$ -- film speed $[\si{\iso}]$
\item $T$ -- white point as a temperature $[\si{\kelvin}]$  (from black body
chromaticity)
\item filmback size $[\si{\square\meter}]$
\item pixel resolution $[\si{\pixel}]$
\end{itemize}

The pixels in the image being synthesized are typically stored as triples of
floating point numbers, also known as \textsl{tristimulus values}, their
magnitudes being linear in the spectral radiant energy reaching the area of each
pixel.
This process can be modelled by introducing the notion of \textsl{film response
function} $W(x,\lambda): \R^2\times\R \to \R^3$ to model the correspondence
between spectral radiance $L^{\downarrow img}_{\lambda}(x, \omega, \lambda, t)$
reaching a given pixel $p^{img}$ and the tristimulus data $C_{col}(p^{img})$ for
that pixel in color space $col$:
\begin{equation}\label{eqn:imaging1}
C_{col}(p^{img})  = \frac1{A_p^{img}}\int_{\lambda} \int_{A_p^{img}}
\int_{\Omega^{img}_a} \int_{\Delta t} W(x, \lambda) L^{\downarrow
img}_{\lambda}(x, \omega, \lambda, t) \cos\theta\d x\d\omega\d\lambda\d t
\end{equation}
where $A_p^{img}$ is the area spanned by pixel $p^{img}$ on the image plane and
$\Omega^{img}_a$ is the solid angle spanned by the aperture $a$ as seen from
the image side.
Note that this formulation means that the image stores values that are
resolution independent, which is achieved by means of the division by
$A_p^{img}$. In the case in which $W(x, \lambda)$ is the identity function, the
result of the integral would be measured in $[\si{\joule\per\square\meter}]$,
wheras using $W(x, \lambda) = K_m \bar y(\lambda)$ would result in quantities
measured in $[\si{\lux\second}]$.

\begin{inconstruction}
 Details on pixel filtering to be inserted here
\end{inconstruction}

The film response function $W(x, \lambda)$ is typically modeled as a product of
components potentially dependent on wavelength and position, which we formulate
as follows
\begin{equation}
W(x,\lambda) = k_i S W_{loc}(x) W_{col}(\lambda)
\end{equation}
where $k_i$ is the \textsl{imaging constant}, $S$ is the film speed,
$W_{loc}(x)$ describes the \textsl{local response} of the filmback and
$W_{col}(\lambda)$ describes its \textsl{spectral response}.

% TODO: use \sRGBl here
\paragraph{Spectral response}
A common approach in digital image synthesis to construct a spectral sensor
response function is to use the function $W_{XYZ}(\lambda)$ defined as:
\begin{equation}
W_{XYZ}(\lambda) = \big(\bar x(\lambda), \bar y (\lambda), \bar z(\lambda)\big)
\end{equation}
to obtain tristimulus values in \gls{CIE} \gls{XYZ} space.
Multiplication by the $M_{sRGB}$ color space conversion matrix is now all
that's needed to obtain a linear, scene referred \gls{sRGB} image
$W_{sRGB}(\lambda) = M_{sRGB} W_{XYZ}(\lambda)$
(details on this and other color space transformations are available in
\cref{ch:implementation}).

Although using \gls{CIE} \gls{XYZ} spectral response is common practice
in digital image synthesis (see for example \cite{pharr2010, jakob2010,
ward1994}), it is the recommendation of this document to instead use spectral
response functions matching the spectral response of the cameras used on set.
This is done in the interest of trying to match the metamerism of the camera in
order to minimize changes of appearance of photographed versus synthesized
objects under varying lighting conditions. This means that a more desirable
spectral response function $W_{col}(\lambda)$ would actually be
\begin{equation}
W_{camera}(\lambda) = \big(\bar r_{camera}(\lambda), \bar g_{camera}(\lambda),
\bar b_{camera}(\lambda)\big)
\end{equation}
Conversion into \gls{sRGB} or other working spaces would then be done employing
a similar process to that on the raw data from the cameras, further details are
outlined in~\cref{ch:implementation}.


\paragraph{Local response}
As spectral radiance $L^{\downarrow img}_{\lambda}(x, \omega, \lambda, t)$ lands
on a pixel away from the center of the image, it decreases by a factor
$\cos\theta$ due to its angle of incidence on the filmback, a phenomenon that
dominates the appearance of vignetting. In digital image synthesis this is often
compensated away using a position dependent component of the sensor response
function
\begin{equation}
W_{loc}(x) = \frac{1}{\cos\theta_x}
\end{equation}

\paragraph{Imaging constant}

The remaining factor in the sensor response function $W(x,\lambda)$ is the
imaging constant $k_i$, which is defined as
\begin{equation}\label{eqn:imaging_ki}
 k_i = \frac{4K_m}{C} \simeq \frac{4\cdot683}{312.5} = 8.7424.
\end{equation}
The constant $C$ (valued at $312.5$ in this document) is sometimes called the \textsl{incident light meter
calibration constant} and effectively defines the units for the film speed $S$.
Note that as we use exact integration, our value for $C$ is close to what would be used on a meter fitted with a hemispherical receptor and somewhat lower, because our exact implementation doesn't suffer from losses at near-grazing angles as a physical device would.


\section{Luminance and sensor response}

We initially posed the question of how a given pixel value concretely
relates to the measurement. As we will derive below,
Equation~\eqref{eqn:imaging1} can be simplified to the following
compact form.
\begin{equation}\label{eqn:imaging_response}
  Y(p^{img}) \approx \frac{ \pi\;\Delta t\;S}{C\; N^2} L_v^{\uparrow obj},
\end{equation}
which directly relates the luminance emitted towards the camera pupil
with the luminance of the corresponding sensor response value. This approximation
is valid for distant to mid-field objects for which angular variation
is insignificant. For simplicity we confine ourselves here to luminance, although in
reality the sensor response is carried out in \textit{camera RGB}. However, the
analysis can be carried out analogously by applying the camera
response curves instead of $\bar y$. Furthermore, due to physical
constraints and limited storage precision, physical camera sensors
obey the linearity of \eqref{eqn:imaging_response} only within certain
ranges of incoming light: while most of them are very close to being linear for the
majority of their range, they will of course stop responding at the upper limit of brightness (they ``clip'') and often exhibit a mild over-sensitivity in the darkest side of the range. This behaviour is very specific to each device, for example cameras meant for use in astrophotography contain provisions to better linearize the sensitivity to dim light.

From this it directly becomes obvious that if the aperture number $N$,
exposure time $\Delta t$, and film speed $S$ are set to satisfy the
so-called \emph{exposure equation}
\begin{equation}\label{eqn:imaging_Ev}
E_v = C \frac{N^2}{\Delta t\; S} = \frac{C\; 2^{EV}}S
\end{equation}
the image of a Lambertian reflector with albedo $\rho$ (i.e.,
$L_v^{\uparrow obj} = E_v \frac \rho \pi$) at the center of the sensor
will have luminance $Y(p^{img}) = \rho$ if the illuminance incident on
the reflector is $E_v$. Or in other words, given a measured scene
illuminance, camera exposure values chosen to satisfy the exposure
equation thus yields pixel values in a reasonable range.

\begin{figure}[t]
    \centering
    \def\svgwidth{0.9\linewidth}
    \import{figures/}{exposure_equation_setup.pdf_tex}
    \caption{\label{fig:exposure_equation_setup}%
        Setup for our proof about the exposure equation. The Lambertian reflector with albedo $\rho$ is parallel to 
        the image plane, and the incident illumination is $E_v$. }
\end{figure}

To see why Equation~\eqref{eqn:imaging_response} is true, first consider that in the center of the sensor,
$W_{loc}(x) = 1$. The sensitivity then reduces to $W(x,\lambda) = k_i\;S\;\bar y(\lambda)$,
and we can rewrite Equation~\eqref{eqn:imaging1} in terms of luminance
$L_v^{\downarrow img}$ passing through the aperture:
\begin{equation}\label{eqn:imaging_y}
Y(p^{img}) 
           = \frac{k_i\; \Delta t \; S}{K_m\;A^{img}_p} \;
             \int_{A_p^{img}} 
             \int_{\Omega_a^{img}} 
                L^{\downarrow img}_{v}(x, \omega) \;
                \cos\theta \d x \d\omega.
\end{equation}
The integration along the time dimension $t$ reduces to a multiplication
by $\Delta t$ due to the illumination field being constant in time. 

%\begin{figure} 
%% SPDX-License-Identifier: Apache-2.0
% Copyright (c) Contributors to the PhysLight Project.
%\begin{tikzpicture}[line cap=round,line join=round,>=triangle 45,x=1.0cm,y=1.0cm]
%
%\draw [dash pattern=on 1pt off 1pt on 2pt off 4pt] (0,3)-- (14,3);
%\draw [line width=1.6pt] (1,5)-- (1,1);
%\draw [line width=1.6pt,dotted] (4,5)-- (4,1);
%\draw [line width=1.6pt] (5,5)-- (5,4);
%\draw [line width=1.6pt] (5,2)-- (5,1);
%\draw [line width=1.6pt] (13,5)-- (13,1);
%\draw (1,3)-- (13,3);
%\draw [dotted] (5,3.69)-- (1,3);
%\draw [dotted] (5,3.34)-- (1,3);
%\draw [dotted] (5,2.65)-- (1,3);
%\draw [dotted] (5,2.31)-- (1,3);
%\draw [dotted] (5,2)-- (1,3);
%\draw [dotted] (5,4)-- (1,3);
%\draw [dotted] (5,4)-- (13,3);
%\draw [dotted] (5,3.69)-- (13,3);
%\draw [dotted] (5,3.34)-- (13,3);
%\draw [dotted] (5,2.65)-- (13,3);
%\draw [dotted] (5,2.31)-- (13,3);
%\draw [dotted] (5,2)-- (13,3);
%\draw (1,0.4)-- (1,-0.8);
%\draw [dotted] (1,0.9)-- (1,0.5);
%\draw [->] (1,0.2) -- (4,0.2);
%\draw [->] (1,-0.2) -- (5,-0.2);
%\draw [->] (1,-0.6) -- (13,-0.6);
%\draw (4,0.4)-- (4,0);
%\draw (5,0.4)-- (5,-0.4);
%\draw (13,0.4)-- (13,-0.8);
%\draw [dotted] (4,0.9)-- (4,0.5);
%\draw [dotted] (5,0.9)-- (5,0.5);
%\draw [dotted] (13,0.9)-- (13,0.5);
%\begin{scriptsize}
%\draw[color=black] (8.98,2.9) node {optical axis};
%\draw [fill=xdxdff] (13,3) circle (1.5pt);
%\draw[color=xdxdff] (13.26,3.1) node {$P_{obj}$};
%\draw [fill=uuuuuu] (1,3) circle (1.5pt);
%\draw[color=uuuuuu] (0.61,2.8) node {$P_{img}$};
%\draw[color=black] (2.54,0.4) node {$f$};
%\draw[color=black] (3.04,0) node {$a$};
%\draw[color=black] (7.04,-0.39) node {$o$};
%\end{scriptsize}
%\end{tikzpicture}

\begin{tikzpicture}[line cap=round,line join=round,>=triangle 45,x=1.0cm,y=1.0cm]
\clip(-0.5,-1.5) rectangle (14.5,5.5);
\draw [dash pattern=on 1pt off 1pt on 2pt off 4pt] (0,3)-- (14,3);
\draw [line width=1.6pt] (1,5)-- (1,1);
\draw [line width=1.6pt,dotted] (4,5)-- (4,1);
\draw [line width=1.6pt] (5,5)-- (5,4);
\draw [line width=1.6pt] (5,2)-- (5,1);
\draw [line width=1.6pt] (13,5)-- (13,1);
\draw (1,3)-- (13,3);
\draw [dotted] (5,3.69)-- (1,3);
\draw [dotted] (5,3.34)-- (1,3);
\draw [dotted] (5,2.65)-- (1,3);
\draw [dotted] (5,2.31)-- (1,3);
\draw [dotted] (5,2)-- (1,3);
\draw [dotted] (5,4)-- (1,3);
\draw [dotted] (5,4)-- (13,3);
\draw [dotted] (5,3.69)-- (13,3);
\draw [dotted] (5,3.34)-- (13,3);
\draw [dotted] (5,2.65)-- (13,3);
\draw [dotted] (5,2.31)-- (13,3);
\draw [dotted] (5,2)-- (13,3);
\draw (1,0.4)-- (1,-0.8);
\draw [dotted] (1,0.9)-- (1,0.5);
\draw [->] (1,0.2) -- (4,0.2);
\draw [->] (1,-0.2) -- (5,-0.2);
\draw [->] (1,-0.6) -- (13,-0.6);
\draw (4,0.4)-- (4,0);
\draw (5,0.4)-- (5,-0.4);
\draw (13,0.4)-- (13,-0.8);
\draw [dotted] (4,0.9)-- (4,0.5);
\draw [dotted] (5,0.9)-- (5,0.5);
\draw [dotted] (13,0.9)-- (13,0.5);
\begin{scriptsize}
\draw[color=black] (8.41,2.93) node {optical axis};
\draw [fill=xdxdff] (13,3) circle (1.0pt);
\draw[color=xdxdff] (13.4,3.1) node {$P_{obj}$};
\draw [fill=uuuuuu] (1,3) circle (0.5pt);
\draw[color=uuuuuu] (0.73,2.8) node {$P_{img}$};
\draw[color=black] (2.53,0.34) node {$f$};
\draw[color=black] (3.03,-0.05) node {$a$};
\draw[color=black] (7.03,-0.46) node {$o$};
\draw [color=black] (1,3.04)-- ++(-2.5pt,0 pt) -- ++(5.0pt,0 pt) ++(-2.5pt,-2.5pt) -- ++(0 pt,5.0pt);
\draw [color=black] (1,2.96)-- ++(-2.5pt,0 pt) -- ++(5.0pt,0 pt) ++(-2.5pt,-2.5pt) -- ++(0 pt,5.0pt);
\draw [color=black] (13,2.85)-- ++(-2.5pt,0 pt) -- ++(5.0pt,0 pt) ++(-2.5pt,-2.5pt) -- ++(0 pt,5.0pt);
\draw [color=black] (13,3.15)-- ++(-2.5pt,0 pt) -- ++(5.0pt,0 pt) ++(-2.5pt,-2.5pt) -- ++(0 pt,5.0pt);
\end{scriptsize}
\end{tikzpicture}

%\caption{Entrance pupil distance}
%\label{fig:aperture_distance}
%\end{figure}

Each point $p^{img}$ on the filmback corresponds to a point $p^{obj}$ on the
focus plane in object space. Energy conservation demands that the luminous flux
$\Phi_{av}$ through the aperture must have the same value when computed from the
sensor side or the object side. Assuming the aperture is a disk parallel to 
the filmback plane, and using the luminance $L_v^{\uparrow obj}$ reflected off
the Lambertian surface into the imaging system,
\begin{equation}\label{eqn:imaging_energyconservation}
    \begin{split}
    \Phi_{av} &= \int_{A_p^{img}} \int_{\Omega_a^{img}} L_v^{\downarrow img}(x,\omega) \cos\theta \d\omega \d x \\
              &= \int_{A_p^{obj}} \int_{\Omega_a^{obj}} L_v^{\uparrow obj}(x,\omega) \cos\theta \d\omega \d x  \\
              &= A_p^{obj} \Omega_a^{obj} L_v^{\uparrow obj} 
    \end{split}
\end{equation}
The last equality assumed that $L_v^{\uparrow obj}(x,\omega)$ has
negligible variation over $(x,\omega)\in A_p^{obj}\times\Omega^{obj}_a$
and $\cos\theta \simeq 1$ for all $\omega\in\Omega^{obj}_a$. 
In other words, the aperture is small as seen from $p^{obj}$
and $p^{obj}$ is very near the optical axis of the system. 
Consequently,
\begin{equation}
    \int_{A_p^{img}} \int_{\Omega_a^{img}} L_v^{\downarrow img}(x,\omega) \cos\theta \d\omega \d x 
     = A_p^{obj} \Omega_a^{obj}L_v^{\uparrow obj}.
\end{equation}
Substituting into Equation~\eqref{eqn:imaging_y}, we obtain
\begin{equation}\label{eqn:pixel_value_area_solidangle}
  Y(p^{img}) = \frac{k_i\; \Delta t \; S}{K_m\;A^{img}_p} \; A_p^{obj} \Omega_a^{obj}L_v^{\uparrow obj}.
\end{equation}

\begin{figure}[t]
    \centering
    \def\svgwidth{0.9\linewidth}
    \import{figures/}{a_img_obj.pdf_tex} 
    \caption{\label{fig:aperture_distance}%
        Relating image and object size in a camera where the aperture is small compared to the distance
        $o-a$ between the aperture and the object.}
\end{figure}

At this point, in order to compute $Y(p^{img})$ we need to compute the ratio of
areas $A_p^{obj}/A_p^{img}$, and the solid angle $\Omega_a^{obj}$. Let $a$ be the distance between the filmback and the
aperture on the optical axis and $o$ the distance between filmback and focus
plane. From \cref{fig:aperture_distance}, we can see that
\begin{equation*}
    \frac{A_p^{obj}}{A^{img}_p}
    = \frac{r_o^2}{r_i^2}
    = \frac{t_o^2 \sin^2\theta}{t_i^2 \sin^2\theta}
    = \frac{t_o^2}{t_i^2} \cdot 1
    = \frac{t_o^2 \cos^2\theta}{t_i^2 \cos^2\theta}
    = \frac{(o-a)^2}{a^2}.
\end{equation*}
The solid angle $\Omega_p^{obj}$ can be computed assuming that the radius of the
aperture 
\begin{equation}\label{eqn:aperture_radius}
    r = \frac{f}{2\;N}
\end{equation}
is much smaller than the distance $o-a$ between the object
and the center of the aperture. In this configuration we have simply
\begin{equation}\label{eqn:imaging_omegapobj}
\Omega_p^{obj} \simeq \frac{\pi r^2}{(o-a)^2} \qquad r \ll o-a.
\end{equation}
Substituting both results into Equation~\eqref{eqn:pixel_value_area_solidangle}, we obtain
\begin{equation}\label{eqn:imaging_y2}
Y(p^{img}) =  \frac{k_i\; \Delta t \; S}{K_m\;} \frac{(o-a)^2}{a^2} \frac{\pi\;r^2}{(o-a)^2}\; L_v^{\uparrow obj}.
\end{equation}
If the object distance $o$ is far greater than the focal length $f$ we have
\begin{equation}
f \ll o \Rightarrow  f \simeq a
\end{equation}
and thus
\begin{equation}
  Y(p^{img}) =  \frac{k_i\; \Delta t \; S}{K_m\;} \frac{\pi\;r^2}{a^2}\; L_v^{\uparrow obj} \stackrel{\eqref{eqn:aperture_radius}}{=}
   \frac{\pi\; k_i\; \Delta t \; S}{4\;N^2\;K_m} L_v^{\uparrow obj} 
\end{equation}
which using Equation~\eqref{eqn:imaging_ki} concludes our proof. \hfill $\square$

%%
%   \begin{inconstruction}
%   
%   \paragraph{Physical units of pixel values}
%   An expression similar to the formula for $Y$ lets us determine illuminance at a pixel:
%   \begin{equation}
%   Y_v(p^{img}) = \frac{1}{A_p^{img}} \int_{A_p^{img}} \int_{\Omega_a^{img}}
%   K_m \int_\lambda L_{\lambda}^{\downarrow img}(x,\omega,\lambda) \bar y (\lambda) \cos\theta \d\omega \d x \d\lambda
%   \qquad [\si{\lux}]
%   \end{equation}
%   \end{inconstruction}

\paragraph{Applying white balance}
White balance is a color correction step intended to give a certain spectral distribution 
 neutral gray coordinates in the target color space.
The whitepoint is often specified using the \gls{CCT} of the given color,
sometimes augmented by a second value called \textsl{tint} that is intended to represent a color shift 
in some other direction in the chromaticity plane. 
    
    Conceptually the idea is that
color temperature correction yields an adjustment along an orange-blue axis, and tint provides
a mean to correct along the remaining dimension of the chromaticity space, which is a green-magenta shift.

Unfortunately there is little agreement in the implementations as to how exactly this should be performed:
the color correction temperature seems to be mostly understood as the color of a blackbody emitter for
temperatures below $4000\si{\kelvin}$ and of \gls{CIE} Illuminant D above that. 

The case of the tint correction is somewhat more complicated with various interpretations 
of how such an axis should be oriented: in some descriptions the axis is orthogonal to the 
spectral locus of the blackbody emitter, so that the orientation depends on the specific 
\gls{CCT} chosen. In others, it has a constant orientation. 

Further, the amount of color shift along the tint axis has various ranges such as $[0,1]$, 
$[-100,100]$ and various interpretation as to whether the amount of apparent adjustment should be 
uniform along the range or not.

On the other hand, there seems to be consensus that white balance be implemented as a component-wise
division performed in camera \gls{RGB} coordinates, meant to divide out the coloration of the target
whitepoint from the image. In order to not introduce excessive brightness alterations
to the image, the correction color is adjusted to have either unit luminance $Y$ in \gls{CIE} \gls{XYZ} 
coordinates or unit camera \gls{RGB} green component, an approximation of unit brightness, 
useful for compute-constrained devices such as digital cameras. 

This means that the image would be computed in camera \gls{RGB} coordinates, divided by the color of 
the whitepoint in camera \gls{RGB} coordinates, and then multiplied by the camera \gls{RGB} values 
obtained by inverse transforming $(1,1,1)$ from the target \gls{RGB} space to the camera \gls{RGB} 
space.

