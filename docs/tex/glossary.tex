% SPDX-License-Identifier: Apache-2.0
% Copyright (c) Contributors to the PhysLight Project.

%\newacronym{ddye}{D$_{\text{dye}}$}{donor dye, ex. Alexa 488}
%\newacronym[description={\glslink{r0}{F\"{o}rster distance}}]{R0}{$R_{0}$}{F\"{o}rster distance}
%\newglossaryentry{r0}{name=\glslink{R0}{\ensuremath{R_{0}}},text=F\"{o}rster distance,description={F\"{o}rster distance, where 50\% ...}, sort=R}
%\newglossaryentry{kdeac}{name=\glslink{R0}{\ensuremath{k_{DEAC}}},text=$k_{DEAC}$, description={is the rate of deactivation from ... and emission)}, sort=k}

\newacronym{ADC}{adc}{Analog to Digital Converter}

\newacronym[seealso=ANSI]{ASA}{asa}{American Standards Association}

\newacronym{ANSI}{ansi}{American National Standards Institute}

\newacronym[
	description={An integrated device containing an array of linked capacitors,
		these can be charged upon receiving photons and their charge later read out to
		recover what is effectively a photon count. 
		Used in this document to indicate a specific class of photosensors used in contemporary
		digital cameras}, seealso={CMOS}
]{CCD}{ccd}{Charge-Coupled Device}

\newacronym{CCT}{cct}{Correlated Color Temperature}

\newacronym[description={The traditional system of units in use in Europe for physics
	and related fields, proposed in 1873 by Maxwell and Thmoson among many others. 
	It was supplanted by the \glsname{MKS} system in the 1940s which then turned into
	the \glsname{si} system in the 1960s}, seealso={si,MKS}]{CGS}{cgs}{centimeter-gram-second}

\newacronym{DCC}{dcc}{Digital Content Creation}

\newacronym[description={Commission Internationale de l'\'Eclairage 
	(International Commission on Illumination) is the international standards body that regulates over quantities related to 
	illumination}]{CIE}{cie}{International
	Commission on Illumination}

\newacronym[
	description={Currently one of the main technologies used in integrated circuit construction,
		this is used in this document to indicate a specific class of photosensors used in contemporary
		digital cameras}, seealso={CCD}
]{CMOS}{cmos}{Complementary Metal Oxide Semconductor}

\newacronym{CRI}{cri}{Color Rendering Index}

\newacronym{CUDA}{cuda}{Compute Unified Device Architecture}

\newacronym{DIN}{din}{Deutsches Institut f\"ur Normung (German Institute for Standardisation)}

\newacronym[longplural=Frames per Second]{FPS}{fps}{Frame per Second}

\newacronym{IBL}{ibl}{Image based lighting}

\newacronym{IES}{ies}{Illuminating Engineering Society of North America}

\newacronym{IEC}{iec}{International Electrotechnical Commission}

\newacronym{IESNA}{iesna}{Illuminating Engineering Society of North America}

\newacronym{ISO}{iso}{International Organization for Standardization}

\newacronym{LED}{led}{Light emitting diode}

\newacronym[description={The traditional name for the International System of Units,
after meter/kilogram/second}, seealso={si,CGS}]{MKS}{mks}{meter-kilogram-second}

\newacronym{RGB}{rgb}{RGB color space of generic primaries}

\newacronym{RIB}{rib}{RenderMan Interface Bytestream}

\newacronym{RSL}{rsl}{RenderMan Shading Language}

\newacronym[description={The \textsl{International System of Units}
(Syst\`eme International d'Unit\'es) is a coherent system of
units of measurement (based on the \gls{MKS} system) built around seven base
units: \textit{kelvin} $[\unit{\kelvin}]$ (temperature), \textsl{second}
$[\unit{\second}]$ (time), \textsl{meter} $[\unit{\meter}]$ (length),
\textsl{kilogram} $[\unit{\kilogram}]$ (mass), \textsl{candela} $[\unit{\candela}]$
(luminous intensity), \textsl{mole} $[\unit{\mole}]$ (amount of substance) and
\textsl{ampere} $[\unit{\ampere}]$ (electric current)}]{si}{si}{International
System of Units}

\newacronym[see={spectral distribution}]{SPD}{spd}{Spectral Power Distribution}

\newacronym{sRGB}{srgb}{sRGB color space}

\newacronym{USD}{usd}{Universal Scene Description}

\newacronym{XYZ}{xyz}{CIE XYZ color space}

%%%%%%%%%%%%%%%%%%%%%%%%%%%%%%%%%%%%%%%%%%%%%%%%%%%%%%%%%%%%%%%%%%%%%%%%%%

\newglossaryentry{aperture}
{
	name=aperture,
	seealso={aperture number},
	description={
		An opening through which light can pass. 
		In an optical system such as a camera lens this is also called 
		\textsl{entrance pupil} or \textsl{diaphragm} and is normally
		mounted inside the barrel of the lens to admit light through it. 
		When the aperture is well approximated by a circle, it is normally measured 
		in one of two ways: either the diameter is specified directly, 
		usually in $[\unit{\milli\meter}]$, or when discussing camera lenses, the ratio 
		of the aperture's diameter to the focal length is given, 
		and is called the \textsl{aperture number}}
}

\newglossaryentry{aperture number}
{
	name={aperture number},
	seealso={aperture,f-number,T-stop},
	description={In a camera system, the aperture is usually given as a ratio to the
		focal length, and marked such as $\sfrac{f}{2.8}$: on a $50\unit{\milli\meter}$ 
		camera lens, this would mean the diameter of the lens's aperture 
		is $50 / 2.8 \approx 17.8\unit{\milli\meter}$. 
		Like other series used in photography, the
		numbers found on most lenses are taken from this standard series:
		$\sfrac{f}{1.4}$, $\sfrac{f}{2}$, $\sfrac{f}{2.8}$, $\sfrac{f}{4}$, $\sfrac{f}{5.6}$,
		$\sfrac{f}{8}$, $\sfrac{f}{11}$, $\sfrac{f}{16}$, $\sfrac{f}{22}$, $\sfrac{f}{32}$,
		each in an approximate ratio of $\sqrt{2}$ to the next one, corresponding to a doubling/halving of admitted light from the previous one.
	    The aperture number if also called \textsl{f-number} and indicated with the symbol
    	$N$, where $N=8$ indicates an $\sfrac{f}{8}$ aperture}
}

\newglossaryentry{black body}
{
	name={black body},
	description=\nopostdesc
}

\newglossaryentry{brightness}
{
  name=brightness,
  description={The strength of the visual perception caused by the luminance emitted or
    reflected by an object. It is a subjective property of the object being
    observed and it is used in this document in its loose, 
    plain spoken-English sense. 
    Other terms are used to designate specific properties}
}

\newglossaryentry{camera lens}
{
	name={camera lens},
	seealso={aperture number},
	description={An assembly of optical lenses meant for use in conjunction with a 
		camera body to make images of objects by focusing the light arriving from the scene
		onto the camera's filmback. 
		The various optical lenses in a camera lens are called \textsl{elements} and 
		are held in a \textsl{barrel} or approximately cylindrical shape.
		Around the barrel are normally \textsl{rings} which the photographer can
		operate to achieve focus, to choose the aperture number and, in zoom lenses, to
		control the focal length.
		Some high quality lenses also include a shutter, suitably places inbetween the
		elements in the barrel and very near to the aperture. In some cases the shutter 
		can serve double duty and precisely open just to the desired aperture size}
}

\newglossaryentry{candela}
{
	name={candela},
	description={TODO}
}

\newglossaryentry{color}
{
	name={color},
	description={The visual perception based on the distribution of photon wavelengths arriving
		at an observer's eye. Care should be taken to keep in mind that color is a property of
		light, as opposed to materials. All materials do is reflect, transmit or absorb light
		thereby \emph{altering} its color}
}

\newglossaryentry{diffuse}
{
	name={diffuse},
	seealso={material model},
	description={A material model (or component thereof) where light bounces off a material
		so that equal luminance is received by all observers in the front half-space of the
		surface (in the case of \textsl{diffuse reflection}) or the back half-space (in the
		case of \textsl{diffuse transmission})}
}

\newglossaryentry{emission}
{
	name={emission},
	description={TODO}
}

\newglossaryentry{entrance pupil}
{
	name={entrance pupil},
	see={aperture},
	description=\nopostdesc
}

\newglossaryentry{exitant}
{
	name={exitant},
	description={In radiometry and photometry, and adjective used to characterize the direction of
		flow as leaving the object}
}

\newglossaryentry{exposure}{
	name={exposure},
	description={A measure of the area density of incident energy}
}

\newglossaryentry{exposure meter}
{
	name={exposure meter},
	seealso={illuminance},
	description={A device used to read out the illuminance at a location. The value is often
		available both in \unit{\lux} as well as \textsl{foot-candle}, being one \unit{\lumen}
		per \textsl{square foot}. Being a photographer's tool, an exposure meter normally also
		has provisions to quickly solve the exposure equation for the current lighting
		conditions: a dial might be use to choose the film speed }
}

\newglossaryentry{exposure time}
{
	name={exposure time},
	seealso={shutter},
	description={The amount of time for which a camera shutter is kept open in order to
		admit light onto the filmback, this quantity is measured in seconds and uses the
		symbol $t$. 
		On photographic cameras, the value is calles \textsl{shutter speed} and is 
		indicated in fractions of a second, standard timings being
		taken from the series $1$, $\sfrac12$, $\sfrac14$, $\sfrac18$, 
		$\sfrac1{15}$, $\sfrac1{30}$, $\sfrac1{60}$, $\sfrac1{125}$, 
		$\sfrac1{250}$, $\sfrac1{500}$, $\sfrac1{1000}$, $\sfrac1{2000}$.
		In motion pictures the convention is to speak of \textsl{shutter angle}:
		in a motion picture film camera the shutter is made of two half circle ``blades'' 
		each covering 180 degrees and set up to complete one revolution per frame. 
		During setup these can be set to leave an opening between 0 and the
		full 180 degrees during which light is admitted onto the film stock.
		The remaining time, called \textsl{blanking}, is used to advance the film
		to the position of the following frame. The exposure time resulting from this 
		approach on a given angle $\alpha$ in degrees is then $\alpha/360/24 \unit{\second}$
		when the camera is running at $24$ \glsname{FPS}}
}

\newglossaryentry{exposure value}
{
	name={exposure value},
	seealso={shutter speed,aperture number},
	description={A number used to combine the shutter speed and aperture number into an
		exposure setting. 
		Symbols used are $EV$, $Ev$ and $E_v$, although when used to indicate an exposure 
		setting it would be most correct to include the film speed associate with the value, 
		such as $EV_{100}$ for ISO 100 stock.
		For an exposure time of $t$ and aperture number $N$, the relation is
		$EV_{100} = \log_2\frac{N^2}{t}$}
}

\newglossaryentry{f-number}
{
	name={f-number},
	seealso={aperture number,T-stop},
	description=\nopostdesc
}

\newglossaryentry{filmback}
{
	name=filmback,
	seealso={sensor},
	description={The area in a camera body onto which an image is to be formed. 
		In modern cameras the filmback contains a photosensitive digital sensor,
		whereas in non-digital models, the filmback would be covered by 
		photosensitive film}
}


\newglossaryentry{film speed}
{
	name={film speed},
	seealso={stop},
	description={A measure for the sensitivity of photosensitive film stock. 
		The \glsname{ISO} scale is commonly used, with values in powers of two such 
		as 100, 200, 400, 800, 1600, 3200, each corresponding to a one stop increment. 
		The standard had previously been proposed by \glsname{ASA}
		(now called \glsname{ANSI}) and is often found on film stock as 
		\textsl{\glsname{ISO}/\glsname{ASA}}. 
		Also common was the German system standardized by \glsname{DIN}, which instead 
		used a scaled logarithmic scale where a film speed of \glsname{ISO} 100 was equivalent 
		to DIN 21\degree. 
		The sensitivity marking on film stock would often include both values, 
		such as ``\glsname{ISO}/\glsname{ASA} 100/21\degree''. 
		For some higher end film stock, intermediate
		sensitivity rating were also available, usually along this series
		100/21\degree, 125/22\degree, 160/23\degree, 200/24\degree\ and so on}
}

\newglossaryentry{footage}
{
	name=footage,
	description={
		In cinema photography, a segment of motion picture material. The name derives from
	    the days of physical film: at 24 frames per second, being the standard frame rate for
    	35\unit{\milli\meter} physical film, one second of film is exactly 1.5ft.
    	Today the term is used to indicate motion picture content, be that on physical
        media or stored in digital files}
}

\newglossaryentry{highlight}
{
	name=highligh,
	description={TODO}
}

\newglossaryentry{human observer}
{    
	name={human observer},
	description={An observer with a light sensitivity equal to the \textsl{standard human},
		being some kind of a mythical creature.
		The \glsname{CIE} has spent considerable effort to characterize the response to light 
		of the visual system in humans, producing several tabulations each suitable for
		different uses. 
		It is very important to remember that human subjects exhibit a very wide range 
		in how they perceive light, even when the observations are restricted to only 
		consider subjects not affected by color blindness.
		Combined with the difficulty of making objective measurements on a wide range 
		of subjects, the width of this variation is one of the reasons why color science is
		such a complex subject}
}

\newglossaryentry{illuminance}
{
	name=illuminance,
	seealso={brightness},
	description=\nopostdesc
}

\newglossaryentry{illuminant}
{
	name=illuminant,
	seealso={virtual scene},
	description={
		A source of light, meant to focus on the part of a light-emitting object that
	    specifically emits the light, as opposed to other elements of it. All together
        these constitute a \textsl{light fixture}}
}

\newglossaryentry{incident}
{
	name={incident},
	description={In radiometry and photometry, and adjective used to characterize the direction of
		flow as arriving at an object}
}

\newglossaryentry{irradiance}
{
	name=irradiance,
	seealso={brightness},
	description=\nopostdesc
}

\newglossaryentry{luminous efficacy}
{
	name={luminous efficacy},
	description={The ratio of light emitted by a source to its absorbed power.
		More precisely, the ratio of luminous power to radiant power for the source,
		a quantity measured in \unit{\lumen\per\watt}}
}

\newglossaryentry{luminous flux}
{
	name={luminous flux},
	description=\nopostdesc
}

\newglossaryentry{luminance}
{
	name={luminance},
	description=\nopostdesc
}

\newglossaryentry{material model}
{
	name={material model},
	seealso={shader},
	description={TODO}
}

\newglossaryentry{photometry}
{
	name={photometry},
	seealso={radiometry},
	description={The science of measurement of the strength of visible light, 
		through which are defined units that capture its perceived brightness 
		to the human eye. In photometry, light has \textsl{luminous power} 
		measured in \unit{\lumen}}
}


\newglossaryentry{pipeline}
{
	name={pipeline},
	description={The set of processes and software tools used for the production of 
		digital images, especially in digital movie-making. The word is used to draw
		attention to the large number of connections that must be put in place to 
		connect the various major tools in a production workflow, where many kinds
		of files flow together and get combined to produce the result images}
}

\newglossaryentry{radiance}
{
	name=radiance,
	description={The fundamental elemental quantity for light transport, radiance
		is the subject of~\cref{ch:radiance}}
}

\newglossaryentry{radiant power}
{
	name={radiant power},
	seealso={spectral},
	description={The power $\Phi_e$ emitted by a source as electromagnetic waves,
		also known as \textsl{radiant flux} and measured in 
		$\unit{\watt} = \unit{\joule\per\second}$. 
		Sometimes there is discussion of \textsl{spectral radiant power},
		which is the radiant power per unit frequency or wavelength.
		In most cases the radiant power is the integral of spectral radiant power
		over all wavelengths, but for the purposes of this document it is occasionally 
		a useful thought excercise to restrict the integration domain to just the
		visible wavelengths instead}
}

\newglossaryentry{radiant intensity}
{
	name={radiant intensity},
	description=\nopostdesc
}

\newglossaryentry{radiometry}
{
	name={radiometry},
	seealso={photometry},
	description={The science of measurement of the strength of electromagnetic radiation,
		measuring its total energy, or in some cases the energy corresponding to a part of
		its spectrum. In radiometry, visible light has \textsl{radiant power},
		measured in \unit{\watt}}
}

\newglossaryentry{radiosity}
{
	name={radiosity},
	description=\nopostdesc
}

\newglossaryentry{reflection}
{
	name={reflection},
	description={TODO}
}

\newglossaryentry{rendering}
{
	name={rendering},
	seealso={shader,material model,virtual scene},
	description={The process of generating an image from a virtual model, or virtual scene by 
		means of a computer program. This is also called \textsl{image synthesis}. A scene file
		will normally contain information about the shapes of the objects to be rendered, their
		appearance, also called \textsl{shading}, as well as descriptions of the camera or cameras and lighting configuration
		to use}
}

\newglossaryentry{scene}
{
    name={scene},
	description={In filmmaking and theater, a scene is a sequence of actions that compose
		a small narrative unit (in cinema a scene is composed by several 
		\textsl{shots}). 
		The word takes of an additional meaning in virtual cinematography, 
		where a \textsl{scene} is also a set of objects intended to be rendered.
		These virtual scenes are kept in files edited with 
		Digital Content Creation software.
		In rendering, the scene will comprise all the objects to be rendered for one frame,
		including their material models, the lights and the cameras}
}

\newglossaryentry{sensor}
{
	name=sensor,
	seealso={filmback},
	description={
		In a digital camera, the device installed at the filmback onto which the image is 
		formed. Photons hitting the sensor displace electrons or otherwise build an electric
		field proportional to the photon count. An \glsname{ADC} then provides a 
		digital readout of the magnitude of this field}
}

\newglossaryentry{shader}
{
	name={shader},
	description={A program meant to compute the visual appearance of an object. 
		This is usually achieved implementing one or more material models in 
		software which is then fed to a renderer along with geometric descriptions
		of the object to be represented.
		Shaders often read data from files called \textsl{textures} to provide 
		surface detail such as patterns or surface decorations, 
		and also often use procedural approaches to synthesize other surface
		details such as rust marks, stains or scuffs.
		In many cases shaders are written in domain specific languages, such as 
		\glsname{RSL} or \glsname{CUDA} that facilitate certain aspects of the 
		implementation or control of specific hardware}
}

\newglossaryentry{shot}
{
	name={shot},
	seealso={scene},
	description={In filmmaking, a shot is a series of frames that runs for an uninterrupted
		period of time, typically a few seconds to a few minutes. Shots are then composed
		to form scenes through the editing process, using cuts and other kinds of transitions}
}

\newglossaryentry{shutter}
{
	name={shutter},
	seealso={aperture,camera lens,exposure time,filmback},
	description={A component in a camera design to open for a brief period of time, called the 
		\textsl{exposure time}, to permit light to arrive at the filmback. 
		Most often the shutter is mounted on the camera body between the camera lens 
		and the filmback, but in large format photography leaf shutters are available as 
		independent parts of the kit, and are sometimes installed ``sandwiched'' in the
		middle of a so-called two-part lens, so it sits as close as possible to the
		aperture}
}

\newglossaryentry{shutter speed}
{    
	name={shutter speed},
	see={exposure time},
	description=\nopostdesc
}

\newglossaryentry{spectral}
{    
	name={spectral},
	description={An adjective to indicate the value in question (usually a radiometric quantity)
		is to be intended as per unit wavelength. For example, \textsl{spectral radiant power}
		$\Phi_\lambda = \partial\Phi_e / \partial\lambda$ is the partial derivative of 
		radiant power with respect to the wavelength. This is measured in 
		\unit{\watt\per\meter} or sometimes \unit{\watt\per\nano\meter}.
		It is also common (but not used in this document), to use the partial derivative
		with respect to frequency, for which a $\nu$ subscript is used, as in $\Phi_\nu$}
}

\newglossaryentry{spectral distribution}
{
	name={spectral distribution},
	see={spectral},
	description={A function (\textsl{distribution}) relating the spectral density of a 
		quantity to a given wavelength. For example, the spectral power distribution \glsname{SPD} is the 
		derivative of radiant power with respect to wavelength. 
		Distributions are a key concept in the field of probability, 
		where continuous distributions are called \textsl{densities}}
}

\newglossaryentry{standard air}
{
	name={standard air},
	description={Standard air is defined in~\cite[clause 3]{cie:018.2019} as \textsl{dry air at $15\degree\;[\unit\celsius]$ and $101\.325\;[\unit\pascal]$, containing
		$0.045\%$ by volume of carbon dioxide}}
}

\newglossaryentry{stop}
{
	name={stop},
	seealso={film speed,exposure time,aperture},
	description={In photography, stops are changes to the exposure settings that
		double or halve the amount of light admitted during one exposure. 
		They are so called because they correspond to the action of going to the 
		next detent position on the shutter timing gear on the camera or on the 
		aperture ring on the lens barrel, which respectively doubles the 
		exposure time or the aperture area.
		For consistency, film stock is usally rated also in stops, 
		resulting in the base ratings as described in \gls{film speed}. 
		Stops tend to be subdivided in thirds, which correspond
		to an increment of 1 unit on the \glsname{DIN} sensisitivity scale}
}

\newglossaryentry{T-stop}
{
	name={T-stop},
	seealso={f-number,aperture number,camera lens},
	description={A \textsl{transmission stop} is a way to mark aperture numbers on a camera lens
		where the physical size of the aperture is calibrated at manufacturing time 
		to compensate for the light lost through the various elements of the lens, 
		so to admit as much light as it would be admitted if these elements had $100\%$
		transmittance.
		Camera lenses intended for cinema use are normally calibrated with T-stops, 
		because this enables producing footage of identical brightness independent of
		which one camera lens is used from the cinematographer's lens kit selected for the show}
}

\newglossaryentry{tilt-shift lens}
{
	name={tilt-shift lens},
	description={A camera lens built in such a manner that its barrel can be \emph{shifted},
		(moved parallel to its axis), and \emph{tilted} (moved so make the lens axis 
		lay at some angle with respect to the filmback). These camera lenses are sometimes called
		\emph{perspective-control lenses}}
}

\newglossaryentry{transmission}
{
	name={transmission},
	description={TODO}
}

\newglossaryentry{virtual}
{    
	name={virtual},
	description={We use the word \textsl{virtual} to indicate a digital counterpart to some
		real-world process, object or set of objects}
}

\newglossaryentry{virtual scene}
{
	name={virtual scene},
	alias={scene},
	description=\nopostdesc
}

\newglossaryentry{wattage}
{    
	name={wattage},
	alias={radiant power},
	description=\nopostdesc
}

\newglossaryentry{wavelength}
{    
	name={wavelength},
	description={TODO}
}
